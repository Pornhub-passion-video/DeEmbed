\section{Calibration}
SOLTI Calibration is a commonly used calibration method in network analyzers. SOL stands for Short-Open-Load, the standards that are commonly used for a single port calibration. In order to calibrate the s-parameters between different ports, also Through-Isolation (TI) can be used to calibrate S21 and S12.




\subsection{SOL Calibration}
\label{sec:solcal}

\begin{figure}[H]
	\centering
	% Graphic for TeX using PGF
% Title: /home/franss/DeEmbed/deembed/doc/figures/OnePortModel.dia
% Creator: Dia v0.97.3
% CreationDate: Tue Jan 31 14:42:01 2017
% For: franss
% \usepackage{tikz}
% The following commands are not supported in PSTricks at present
% We define them conditionally, so when they are implemented,
% this pgf file will use them.
\ifx\du\undefined
  \newlength{\du}
\fi
\setlength{\du}{15\unitlength}
\begin{tikzpicture}
\pgftransformxscale{1.000000}
\pgftransformyscale{-1.000000}
\definecolor{dialinecolor}{rgb}{0.000000, 0.000000, 0.000000}
\pgfsetstrokecolor{dialinecolor}
\definecolor{dialinecolor}{rgb}{1.000000, 1.000000, 1.000000}
\pgfsetfillcolor{dialinecolor}
\definecolor{dialinecolor}{rgb}{1.000000, 0.996078, 0.870588}
\pgfsetfillcolor{dialinecolor}
\fill (34.000000\du,59.000000\du)--(34.000000\du,65.000000\du)--(37.000000\du,65.000000\du)--(37.000000\du,59.000000\du)--cycle;
\pgfsetlinewidth{0.100000\du}
\pgfsetdash{}{0pt}
\pgfsetdash{}{0pt}
\pgfsetmiterjoin
\definecolor{dialinecolor}{rgb}{0.000000, 0.000000, 0.000000}
\pgfsetstrokecolor{dialinecolor}
\draw (34.000000\du,59.000000\du)--(34.000000\du,65.000000\du)--(37.000000\du,65.000000\du)--(37.000000\du,59.000000\du)--cycle;
% setfont left to latex
\definecolor{dialinecolor}{rgb}{0.000000, 0.000000, 0.000000}
\pgfsetstrokecolor{dialinecolor}
\node at (35.500000\du,62.195000\du){};
\pgfsetlinewidth{0.100000\du}
\pgfsetdash{}{0pt}
\pgfsetdash{}{0pt}
\pgfsetbuttcap
{
\definecolor{dialinecolor}{rgb}{0.000000, 0.000000, 0.000000}
\pgfsetfillcolor{dialinecolor}
% was here!!!
\pgfsetarrowsend{latex}
\definecolor{dialinecolor}{rgb}{0.000000, 0.000000, 0.000000}
\pgfsetstrokecolor{dialinecolor}
\draw (22.750000\du,60.000000\du)--(25.250000\du,60.000000\du);
}
\definecolor{dialinecolor}{rgb}{0.000000, 0.000000, 0.000000}
\pgfsetstrokecolor{dialinecolor}
\draw (23.300000\du,60.000000\du)--(25.250000\du,60.000000\du);
\pgfsetlinewidth{0.100000\du}
\pgfsetdash{}{0pt}
\pgfsetmiterjoin
\pgfsetbuttcap
\definecolor{dialinecolor}{rgb}{1.000000, 1.000000, 1.000000}
\pgfsetfillcolor{dialinecolor}
\pgfpathmoveto{\pgfpoint{22.800000\du}{60.000000\du}}
\pgfpathcurveto{\pgfpoint{22.800000\du}{59.875000\du}}{\pgfpoint{22.925000\du}{59.750000\du}}{\pgfpoint{23.050000\du}{59.750000\du}}
\pgfpathcurveto{\pgfpoint{23.175000\du}{59.750000\du}}{\pgfpoint{23.300000\du}{59.875000\du}}{\pgfpoint{23.300000\du}{60.000000\du}}
\pgfpathcurveto{\pgfpoint{23.300000\du}{60.125000\du}}{\pgfpoint{23.175000\du}{60.250000\du}}{\pgfpoint{23.050000\du}{60.250000\du}}
\pgfpathcurveto{\pgfpoint{22.925000\du}{60.250000\du}}{\pgfpoint{22.800000\du}{60.125000\du}}{\pgfpoint{22.800000\du}{60.000000\du}}
\pgfusepath{fill}
\definecolor{dialinecolor}{rgb}{0.000000, 0.000000, 0.000000}
\pgfsetstrokecolor{dialinecolor}
\pgfpathmoveto{\pgfpoint{22.800000\du}{60.000000\du}}
\pgfpathcurveto{\pgfpoint{22.800000\du}{59.875000\du}}{\pgfpoint{22.925000\du}{59.750000\du}}{\pgfpoint{23.050000\du}{59.750000\du}}
\pgfpathcurveto{\pgfpoint{23.175000\du}{59.750000\du}}{\pgfpoint{23.300000\du}{59.875000\du}}{\pgfpoint{23.300000\du}{60.000000\du}}
\pgfpathcurveto{\pgfpoint{23.300000\du}{60.125000\du}}{\pgfpoint{23.175000\du}{60.250000\du}}{\pgfpoint{23.050000\du}{60.250000\du}}
\pgfpathcurveto{\pgfpoint{22.925000\du}{60.250000\du}}{\pgfpoint{22.800000\du}{60.125000\du}}{\pgfpoint{22.800000\du}{60.000000\du}}
\pgfusepath{stroke}
\pgfsetlinewidth{0.100000\du}
\pgfsetdash{}{0pt}
\pgfsetdash{}{0pt}
\pgfsetbuttcap
{
\definecolor{dialinecolor}{rgb}{0.000000, 0.000000, 0.000000}
\pgfsetfillcolor{dialinecolor}
% was here!!!
\definecolor{dialinecolor}{rgb}{0.000000, 0.000000, 0.000000}
\pgfsetstrokecolor{dialinecolor}
\draw (25.000000\du,60.000000\du)--(27.250000\du,60.000000\du);
}
\definecolor{dialinecolor}{rgb}{0.000000, 0.000000, 0.000000}
\pgfsetstrokecolor{dialinecolor}
\draw (25.000000\du,60.000000\du)--(27.250000\du,60.000000\du);
\pgfsetlinewidth{0.100000\du}
\pgfsetdash{}{0pt}
\pgfsetmiterjoin
\pgfsetbuttcap
\definecolor{dialinecolor}{rgb}{0.000000, 0.000000, 0.000000}
\pgfsetfillcolor{dialinecolor}
\pgfpathmoveto{\pgfpoint{27.250000\du}{60.000000\du}}
\pgfpathcurveto{\pgfpoint{27.250000\du}{60.125000\du}}{\pgfpoint{27.125000\du}{60.250000\du}}{\pgfpoint{27.000000\du}{60.250000\du}}
\pgfpathcurveto{\pgfpoint{26.875000\du}{60.250000\du}}{\pgfpoint{26.750000\du}{60.125000\du}}{\pgfpoint{26.750000\du}{60.000000\du}}
\pgfpathcurveto{\pgfpoint{26.750000\du}{59.875000\du}}{\pgfpoint{26.875000\du}{59.750000\du}}{\pgfpoint{27.000000\du}{59.750000\du}}
\pgfpathcurveto{\pgfpoint{27.125000\du}{59.750000\du}}{\pgfpoint{27.250000\du}{59.875000\du}}{\pgfpoint{27.250000\du}{60.000000\du}}
\pgfusepath{fill}
\definecolor{dialinecolor}{rgb}{0.000000, 0.000000, 0.000000}
\pgfsetstrokecolor{dialinecolor}
\pgfpathmoveto{\pgfpoint{27.250000\du}{60.000000\du}}
\pgfpathcurveto{\pgfpoint{27.250000\du}{60.125000\du}}{\pgfpoint{27.125000\du}{60.250000\du}}{\pgfpoint{27.000000\du}{60.250000\du}}
\pgfpathcurveto{\pgfpoint{26.875000\du}{60.250000\du}}{\pgfpoint{26.750000\du}{60.125000\du}}{\pgfpoint{26.750000\du}{60.000000\du}}
\pgfpathcurveto{\pgfpoint{26.750000\du}{59.875000\du}}{\pgfpoint{26.875000\du}{59.750000\du}}{\pgfpoint{27.000000\du}{59.750000\du}}
\pgfpathcurveto{\pgfpoint{27.125000\du}{59.750000\du}}{\pgfpoint{27.250000\du}{59.875000\du}}{\pgfpoint{27.250000\du}{60.000000\du}}
\pgfusepath{stroke}
\pgfsetlinewidth{0.100000\du}
\pgfsetdash{}{0pt}
\pgfsetdash{}{0pt}
\pgfsetbuttcap
{
\definecolor{dialinecolor}{rgb}{0.000000, 0.000000, 0.000000}
\pgfsetfillcolor{dialinecolor}
% was here!!!
\definecolor{dialinecolor}{rgb}{0.000000, 0.000000, 0.000000}
\pgfsetstrokecolor{dialinecolor}
\draw (22.750000\du,64.000000\du)--(25.000000\du,64.000000\du);
}
\definecolor{dialinecolor}{rgb}{0.000000, 0.000000, 0.000000}
\pgfsetstrokecolor{dialinecolor}
\draw (23.300000\du,64.000000\du)--(25.000000\du,64.000000\du);
\pgfsetlinewidth{0.100000\du}
\pgfsetdash{}{0pt}
\pgfsetmiterjoin
\pgfsetbuttcap
\definecolor{dialinecolor}{rgb}{1.000000, 1.000000, 1.000000}
\pgfsetfillcolor{dialinecolor}
\pgfpathmoveto{\pgfpoint{22.800000\du}{64.000000\du}}
\pgfpathcurveto{\pgfpoint{22.800000\du}{63.875000\du}}{\pgfpoint{22.925000\du}{63.750000\du}}{\pgfpoint{23.050000\du}{63.750000\du}}
\pgfpathcurveto{\pgfpoint{23.175000\du}{63.750000\du}}{\pgfpoint{23.300000\du}{63.875000\du}}{\pgfpoint{23.300000\du}{64.000000\du}}
\pgfpathcurveto{\pgfpoint{23.300000\du}{64.125000\du}}{\pgfpoint{23.175000\du}{64.250000\du}}{\pgfpoint{23.050000\du}{64.250000\du}}
\pgfpathcurveto{\pgfpoint{22.925000\du}{64.250000\du}}{\pgfpoint{22.800000\du}{64.125000\du}}{\pgfpoint{22.800000\du}{64.000000\du}}
\pgfusepath{fill}
\definecolor{dialinecolor}{rgb}{0.000000, 0.000000, 0.000000}
\pgfsetstrokecolor{dialinecolor}
\pgfpathmoveto{\pgfpoint{22.800000\du}{64.000000\du}}
\pgfpathcurveto{\pgfpoint{22.800000\du}{63.875000\du}}{\pgfpoint{22.925000\du}{63.750000\du}}{\pgfpoint{23.050000\du}{63.750000\du}}
\pgfpathcurveto{\pgfpoint{23.175000\du}{63.750000\du}}{\pgfpoint{23.300000\du}{63.875000\du}}{\pgfpoint{23.300000\du}{64.000000\du}}
\pgfpathcurveto{\pgfpoint{23.300000\du}{64.125000\du}}{\pgfpoint{23.175000\du}{64.250000\du}}{\pgfpoint{23.050000\du}{64.250000\du}}
\pgfpathcurveto{\pgfpoint{22.925000\du}{64.250000\du}}{\pgfpoint{22.800000\du}{64.125000\du}}{\pgfpoint{22.800000\du}{64.000000\du}}
\pgfusepath{stroke}
\pgfsetlinewidth{0.100000\du}
\pgfsetdash{}{0pt}
\pgfsetdash{}{0pt}
\pgfsetbuttcap
{
\definecolor{dialinecolor}{rgb}{0.000000, 0.000000, 0.000000}
\pgfsetfillcolor{dialinecolor}
% was here!!!
\pgfsetarrowsstart{latex}
\definecolor{dialinecolor}{rgb}{0.000000, 0.000000, 0.000000}
\pgfsetstrokecolor{dialinecolor}
\draw (24.750000\du,64.000000\du)--(27.250000\du,64.000000\du);
}
\definecolor{dialinecolor}{rgb}{0.000000, 0.000000, 0.000000}
\pgfsetstrokecolor{dialinecolor}
\draw (24.750000\du,64.000000\du)--(27.250000\du,64.000000\du);
\pgfsetlinewidth{0.100000\du}
\pgfsetdash{}{0pt}
\pgfsetmiterjoin
\pgfsetbuttcap
\definecolor{dialinecolor}{rgb}{0.000000, 0.000000, 0.000000}
\pgfsetfillcolor{dialinecolor}
\pgfpathmoveto{\pgfpoint{27.250000\du}{64.000000\du}}
\pgfpathcurveto{\pgfpoint{27.250000\du}{64.125000\du}}{\pgfpoint{27.125000\du}{64.250000\du}}{\pgfpoint{27.000000\du}{64.250000\du}}
\pgfpathcurveto{\pgfpoint{26.875000\du}{64.250000\du}}{\pgfpoint{26.750000\du}{64.125000\du}}{\pgfpoint{26.750000\du}{64.000000\du}}
\pgfpathcurveto{\pgfpoint{26.750000\du}{63.875000\du}}{\pgfpoint{26.875000\du}{63.750000\du}}{\pgfpoint{27.000000\du}{63.750000\du}}
\pgfpathcurveto{\pgfpoint{27.125000\du}{63.750000\du}}{\pgfpoint{27.250000\du}{63.875000\du}}{\pgfpoint{27.250000\du}{64.000000\du}}
\pgfusepath{fill}
\definecolor{dialinecolor}{rgb}{0.000000, 0.000000, 0.000000}
\pgfsetstrokecolor{dialinecolor}
\pgfpathmoveto{\pgfpoint{27.250000\du}{64.000000\du}}
\pgfpathcurveto{\pgfpoint{27.250000\du}{64.125000\du}}{\pgfpoint{27.125000\du}{64.250000\du}}{\pgfpoint{27.000000\du}{64.250000\du}}
\pgfpathcurveto{\pgfpoint{26.875000\du}{64.250000\du}}{\pgfpoint{26.750000\du}{64.125000\du}}{\pgfpoint{26.750000\du}{64.000000\du}}
\pgfpathcurveto{\pgfpoint{26.750000\du}{63.875000\du}}{\pgfpoint{26.875000\du}{63.750000\du}}{\pgfpoint{27.000000\du}{63.750000\du}}
\pgfpathcurveto{\pgfpoint{27.125000\du}{63.750000\du}}{\pgfpoint{27.250000\du}{63.875000\du}}{\pgfpoint{27.250000\du}{64.000000\du}}
\pgfusepath{stroke}
\pgfsetlinewidth{0.100000\du}
\pgfsetdash{}{0pt}
\pgfsetdash{}{0pt}
\pgfsetbuttcap
{
\definecolor{dialinecolor}{rgb}{0.000000, 0.000000, 0.000000}
\pgfsetfillcolor{dialinecolor}
% was here!!!
\pgfsetarrowsend{latex}
\definecolor{dialinecolor}{rgb}{0.000000, 0.000000, 0.000000}
\pgfsetstrokecolor{dialinecolor}
\draw (27.000000\du,60.000000\du)--(27.000000\du,62.250000\du);
}
\pgfsetlinewidth{0.100000\du}
\pgfsetdash{}{0pt}
\pgfsetdash{}{0pt}
\pgfsetbuttcap
{
\definecolor{dialinecolor}{rgb}{0.000000, 0.000000, 0.000000}
\pgfsetfillcolor{dialinecolor}
% was here!!!
\definecolor{dialinecolor}{rgb}{0.000000, 0.000000, 0.000000}
\pgfsetstrokecolor{dialinecolor}
\draw (27.000000\du,62.000000\du)--(27.000000\du,64.000000\du);
}
\pgfsetlinewidth{0.100000\du}
\pgfsetdash{}{0pt}
\pgfsetdash{}{0pt}
\pgfsetbuttcap
{
\definecolor{dialinecolor}{rgb}{0.000000, 0.000000, 0.000000}
\pgfsetfillcolor{dialinecolor}
% was here!!!
\pgfsetarrowsend{latex}
\definecolor{dialinecolor}{rgb}{0.000000, 0.000000, 0.000000}
\pgfsetstrokecolor{dialinecolor}
\draw (27.000000\du,60.000000\du)--(29.250000\du,60.000000\du);
}
\pgfsetlinewidth{0.100000\du}
\pgfsetdash{}{0pt}
\pgfsetdash{}{0pt}
\pgfsetbuttcap
{
\definecolor{dialinecolor}{rgb}{0.000000, 0.000000, 0.000000}
\pgfsetfillcolor{dialinecolor}
% was here!!!
\definecolor{dialinecolor}{rgb}{0.000000, 0.000000, 0.000000}
\pgfsetstrokecolor{dialinecolor}
\draw (29.000000\du,60.000000\du)--(31.250000\du,60.000000\du);
}
\definecolor{dialinecolor}{rgb}{0.000000, 0.000000, 0.000000}
\pgfsetstrokecolor{dialinecolor}
\draw (29.000000\du,60.000000\du)--(31.250000\du,60.000000\du);
\pgfsetlinewidth{0.100000\du}
\pgfsetdash{}{0pt}
\pgfsetmiterjoin
\pgfsetbuttcap
\definecolor{dialinecolor}{rgb}{0.000000, 0.000000, 0.000000}
\pgfsetfillcolor{dialinecolor}
\pgfpathmoveto{\pgfpoint{31.250000\du}{60.000000\du}}
\pgfpathcurveto{\pgfpoint{31.250000\du}{60.125000\du}}{\pgfpoint{31.125000\du}{60.250000\du}}{\pgfpoint{31.000000\du}{60.250000\du}}
\pgfpathcurveto{\pgfpoint{30.875000\du}{60.250000\du}}{\pgfpoint{30.750000\du}{60.125000\du}}{\pgfpoint{30.750000\du}{60.000000\du}}
\pgfpathcurveto{\pgfpoint{30.750000\du}{59.875000\du}}{\pgfpoint{30.875000\du}{59.750000\du}}{\pgfpoint{31.000000\du}{59.750000\du}}
\pgfpathcurveto{\pgfpoint{31.125000\du}{59.750000\du}}{\pgfpoint{31.250000\du}{59.875000\du}}{\pgfpoint{31.250000\du}{60.000000\du}}
\pgfusepath{fill}
\definecolor{dialinecolor}{rgb}{0.000000, 0.000000, 0.000000}
\pgfsetstrokecolor{dialinecolor}
\pgfpathmoveto{\pgfpoint{31.250000\du}{60.000000\du}}
\pgfpathcurveto{\pgfpoint{31.250000\du}{60.125000\du}}{\pgfpoint{31.125000\du}{60.250000\du}}{\pgfpoint{31.000000\du}{60.250000\du}}
\pgfpathcurveto{\pgfpoint{30.875000\du}{60.250000\du}}{\pgfpoint{30.750000\du}{60.125000\du}}{\pgfpoint{30.750000\du}{60.000000\du}}
\pgfpathcurveto{\pgfpoint{30.750000\du}{59.875000\du}}{\pgfpoint{30.875000\du}{59.750000\du}}{\pgfpoint{31.000000\du}{59.750000\du}}
\pgfpathcurveto{\pgfpoint{31.125000\du}{59.750000\du}}{\pgfpoint{31.250000\du}{59.875000\du}}{\pgfpoint{31.250000\du}{60.000000\du}}
\pgfusepath{stroke}
\pgfsetlinewidth{0.100000\du}
\pgfsetdash{}{0pt}
\pgfsetdash{}{0pt}
\pgfsetbuttcap
{
\definecolor{dialinecolor}{rgb}{0.000000, 0.000000, 0.000000}
\pgfsetfillcolor{dialinecolor}
% was here!!!
\pgfsetarrowsend{latex}
\definecolor{dialinecolor}{rgb}{0.000000, 0.000000, 0.000000}
\pgfsetstrokecolor{dialinecolor}
\draw (31.000000\du,60.000000\du)--(33.250000\du,60.000000\du);
}
\pgfsetlinewidth{0.100000\du}
\pgfsetdash{}{0pt}
\pgfsetdash{}{0pt}
\pgfsetbuttcap
{
\definecolor{dialinecolor}{rgb}{0.000000, 0.000000, 0.000000}
\pgfsetfillcolor{dialinecolor}
% was here!!!
\definecolor{dialinecolor}{rgb}{0.000000, 0.000000, 0.000000}
\pgfsetstrokecolor{dialinecolor}
\draw (33.000000\du,60.000000\du)--(35.250000\du,60.000000\du);
}
\definecolor{dialinecolor}{rgb}{0.000000, 0.000000, 0.000000}
\pgfsetstrokecolor{dialinecolor}
\draw (33.000000\du,60.000000\du)--(35.250000\du,60.000000\du);
\pgfsetlinewidth{0.100000\du}
\pgfsetdash{}{0pt}
\pgfsetmiterjoin
\pgfsetbuttcap
\definecolor{dialinecolor}{rgb}{0.000000, 0.000000, 0.000000}
\pgfsetfillcolor{dialinecolor}
\pgfpathmoveto{\pgfpoint{35.250000\du}{60.000000\du}}
\pgfpathcurveto{\pgfpoint{35.250000\du}{60.125000\du}}{\pgfpoint{35.125000\du}{60.250000\du}}{\pgfpoint{35.000000\du}{60.250000\du}}
\pgfpathcurveto{\pgfpoint{34.875000\du}{60.250000\du}}{\pgfpoint{34.750000\du}{60.125000\du}}{\pgfpoint{34.750000\du}{60.000000\du}}
\pgfpathcurveto{\pgfpoint{34.750000\du}{59.875000\du}}{\pgfpoint{34.875000\du}{59.750000\du}}{\pgfpoint{35.000000\du}{59.750000\du}}
\pgfpathcurveto{\pgfpoint{35.125000\du}{59.750000\du}}{\pgfpoint{35.250000\du}{59.875000\du}}{\pgfpoint{35.250000\du}{60.000000\du}}
\pgfusepath{fill}
\definecolor{dialinecolor}{rgb}{0.000000, 0.000000, 0.000000}
\pgfsetstrokecolor{dialinecolor}
\pgfpathmoveto{\pgfpoint{35.250000\du}{60.000000\du}}
\pgfpathcurveto{\pgfpoint{35.250000\du}{60.125000\du}}{\pgfpoint{35.125000\du}{60.250000\du}}{\pgfpoint{35.000000\du}{60.250000\du}}
\pgfpathcurveto{\pgfpoint{34.875000\du}{60.250000\du}}{\pgfpoint{34.750000\du}{60.125000\du}}{\pgfpoint{34.750000\du}{60.000000\du}}
\pgfpathcurveto{\pgfpoint{34.750000\du}{59.875000\du}}{\pgfpoint{34.875000\du}{59.750000\du}}{\pgfpoint{35.000000\du}{59.750000\du}}
\pgfpathcurveto{\pgfpoint{35.125000\du}{59.750000\du}}{\pgfpoint{35.250000\du}{59.875000\du}}{\pgfpoint{35.250000\du}{60.000000\du}}
\pgfusepath{stroke}
\pgfsetlinewidth{0.100000\du}
\pgfsetdash{}{0pt}
\pgfsetdash{}{0pt}
\pgfsetbuttcap
{
\definecolor{dialinecolor}{rgb}{0.000000, 0.000000, 0.000000}
\pgfsetfillcolor{dialinecolor}
% was here!!!
\pgfsetarrowsend{latex}
\definecolor{dialinecolor}{rgb}{0.000000, 0.000000, 0.000000}
\pgfsetstrokecolor{dialinecolor}
\draw (31.000000\du,64.000000\du)--(31.000000\du,61.750000\du);
}
\pgfsetlinewidth{0.100000\du}
\pgfsetdash{}{0pt}
\pgfsetdash{}{0pt}
\pgfsetbuttcap
{
\definecolor{dialinecolor}{rgb}{0.000000, 0.000000, 0.000000}
\pgfsetfillcolor{dialinecolor}
% was here!!!
\definecolor{dialinecolor}{rgb}{0.000000, 0.000000, 0.000000}
\pgfsetstrokecolor{dialinecolor}
\draw (31.000000\du,60.000000\du)--(31.000000\du,62.000000\du);
}
\pgfsetlinewidth{0.100000\du}
\pgfsetdash{}{0pt}
\pgfsetdash{}{0pt}
\pgfsetbuttcap
{
\definecolor{dialinecolor}{rgb}{0.000000, 0.000000, 0.000000}
\pgfsetfillcolor{dialinecolor}
% was here!!!
\pgfsetarrowsend{latex}
\definecolor{dialinecolor}{rgb}{0.000000, 0.000000, 0.000000}
\pgfsetstrokecolor{dialinecolor}
\draw (31.250000\du,64.000000\du)--(28.750000\du,64.000000\du);
}
\definecolor{dialinecolor}{rgb}{0.000000, 0.000000, 0.000000}
\pgfsetstrokecolor{dialinecolor}
\draw (31.250000\du,64.000000\du)--(28.750000\du,64.000000\du);
\pgfsetlinewidth{0.100000\du}
\pgfsetdash{}{0pt}
\pgfsetmiterjoin
\pgfsetbuttcap
\definecolor{dialinecolor}{rgb}{0.000000, 0.000000, 0.000000}
\pgfsetfillcolor{dialinecolor}
\pgfpathmoveto{\pgfpoint{31.250000\du}{64.000000\du}}
\pgfpathcurveto{\pgfpoint{31.250000\du}{64.125000\du}}{\pgfpoint{31.125000\du}{64.250000\du}}{\pgfpoint{31.000000\du}{64.250000\du}}
\pgfpathcurveto{\pgfpoint{30.875000\du}{64.250000\du}}{\pgfpoint{30.750000\du}{64.125000\du}}{\pgfpoint{30.750000\du}{64.000000\du}}
\pgfpathcurveto{\pgfpoint{30.750000\du}{63.875000\du}}{\pgfpoint{30.875000\du}{63.750000\du}}{\pgfpoint{31.000000\du}{63.750000\du}}
\pgfpathcurveto{\pgfpoint{31.125000\du}{63.750000\du}}{\pgfpoint{31.250000\du}{63.875000\du}}{\pgfpoint{31.250000\du}{64.000000\du}}
\pgfusepath{fill}
\definecolor{dialinecolor}{rgb}{0.000000, 0.000000, 0.000000}
\pgfsetstrokecolor{dialinecolor}
\pgfpathmoveto{\pgfpoint{31.250000\du}{64.000000\du}}
\pgfpathcurveto{\pgfpoint{31.250000\du}{64.125000\du}}{\pgfpoint{31.125000\du}{64.250000\du}}{\pgfpoint{31.000000\du}{64.250000\du}}
\pgfpathcurveto{\pgfpoint{30.875000\du}{64.250000\du}}{\pgfpoint{30.750000\du}{64.125000\du}}{\pgfpoint{30.750000\du}{64.000000\du}}
\pgfpathcurveto{\pgfpoint{30.750000\du}{63.875000\du}}{\pgfpoint{30.875000\du}{63.750000\du}}{\pgfpoint{31.000000\du}{63.750000\du}}
\pgfpathcurveto{\pgfpoint{31.125000\du}{63.750000\du}}{\pgfpoint{31.250000\du}{63.875000\du}}{\pgfpoint{31.250000\du}{64.000000\du}}
\pgfusepath{stroke}
\pgfsetlinewidth{0.100000\du}
\pgfsetdash{}{0pt}
\pgfsetdash{}{0pt}
\pgfsetbuttcap
{
\definecolor{dialinecolor}{rgb}{0.000000, 0.000000, 0.000000}
\pgfsetfillcolor{dialinecolor}
% was here!!!
\definecolor{dialinecolor}{rgb}{0.000000, 0.000000, 0.000000}
\pgfsetstrokecolor{dialinecolor}
\draw (27.000000\du,64.000000\du)--(29.000000\du,64.000000\du);
}
\pgfsetlinewidth{0.100000\du}
\pgfsetdash{}{0pt}
\pgfsetdash{}{0pt}
\pgfsetbuttcap
{
\definecolor{dialinecolor}{rgb}{0.000000, 0.000000, 0.000000}
\pgfsetfillcolor{dialinecolor}
% was here!!!
\pgfsetarrowsend{latex}
\definecolor{dialinecolor}{rgb}{0.000000, 0.000000, 0.000000}
\pgfsetstrokecolor{dialinecolor}
\draw (35.250000\du,64.000000\du)--(32.750000\du,64.000000\du);
}
\definecolor{dialinecolor}{rgb}{0.000000, 0.000000, 0.000000}
\pgfsetstrokecolor{dialinecolor}
\draw (35.250000\du,64.000000\du)--(32.750000\du,64.000000\du);
\pgfsetlinewidth{0.100000\du}
\pgfsetdash{}{0pt}
\pgfsetmiterjoin
\pgfsetbuttcap
\definecolor{dialinecolor}{rgb}{0.000000, 0.000000, 0.000000}
\pgfsetfillcolor{dialinecolor}
\pgfpathmoveto{\pgfpoint{35.250000\du}{64.000000\du}}
\pgfpathcurveto{\pgfpoint{35.250000\du}{64.125000\du}}{\pgfpoint{35.125000\du}{64.250000\du}}{\pgfpoint{35.000000\du}{64.250000\du}}
\pgfpathcurveto{\pgfpoint{34.875000\du}{64.250000\du}}{\pgfpoint{34.750000\du}{64.125000\du}}{\pgfpoint{34.750000\du}{64.000000\du}}
\pgfpathcurveto{\pgfpoint{34.750000\du}{63.875000\du}}{\pgfpoint{34.875000\du}{63.750000\du}}{\pgfpoint{35.000000\du}{63.750000\du}}
\pgfpathcurveto{\pgfpoint{35.125000\du}{63.750000\du}}{\pgfpoint{35.250000\du}{63.875000\du}}{\pgfpoint{35.250000\du}{64.000000\du}}
\pgfusepath{fill}
\definecolor{dialinecolor}{rgb}{0.000000, 0.000000, 0.000000}
\pgfsetstrokecolor{dialinecolor}
\pgfpathmoveto{\pgfpoint{35.250000\du}{64.000000\du}}
\pgfpathcurveto{\pgfpoint{35.250000\du}{64.125000\du}}{\pgfpoint{35.125000\du}{64.250000\du}}{\pgfpoint{35.000000\du}{64.250000\du}}
\pgfpathcurveto{\pgfpoint{34.875000\du}{64.250000\du}}{\pgfpoint{34.750000\du}{64.125000\du}}{\pgfpoint{34.750000\du}{64.000000\du}}
\pgfpathcurveto{\pgfpoint{34.750000\du}{63.875000\du}}{\pgfpoint{34.875000\du}{63.750000\du}}{\pgfpoint{35.000000\du}{63.750000\du}}
\pgfpathcurveto{\pgfpoint{35.125000\du}{63.750000\du}}{\pgfpoint{35.250000\du}{63.875000\du}}{\pgfpoint{35.250000\du}{64.000000\du}}
\pgfusepath{stroke}
\pgfsetlinewidth{0.100000\du}
\pgfsetdash{}{0pt}
\pgfsetdash{}{0pt}
\pgfsetbuttcap
{
\definecolor{dialinecolor}{rgb}{0.000000, 0.000000, 0.000000}
\pgfsetfillcolor{dialinecolor}
% was here!!!
\definecolor{dialinecolor}{rgb}{0.000000, 0.000000, 0.000000}
\pgfsetstrokecolor{dialinecolor}
\draw (31.000000\du,64.000000\du)--(33.000000\du,64.000000\du);
}
\pgfsetlinewidth{0.100000\du}
\pgfsetdash{}{0pt}
\pgfsetdash{}{0pt}
\pgfsetbuttcap
{
\definecolor{dialinecolor}{rgb}{0.000000, 0.000000, 0.000000}
\pgfsetfillcolor{dialinecolor}
% was here!!!
\pgfsetarrowsend{latex}
\definecolor{dialinecolor}{rgb}{0.000000, 0.000000, 0.000000}
\pgfsetstrokecolor{dialinecolor}
\draw (35.000000\du,60.000000\du)--(35.000000\du,62.250000\du);
}
\pgfsetlinewidth{0.100000\du}
\pgfsetdash{}{0pt}
\pgfsetdash{}{0pt}
\pgfsetbuttcap
{
\definecolor{dialinecolor}{rgb}{0.000000, 0.000000, 0.000000}
\pgfsetfillcolor{dialinecolor}
% was here!!!
\definecolor{dialinecolor}{rgb}{0.000000, 0.000000, 0.000000}
\pgfsetstrokecolor{dialinecolor}
\draw (35.000000\du,62.000000\du)--(35.000000\du,64.000000\du);
}
% setfont left to latex
\definecolor{dialinecolor}{rgb}{0.000000, 0.000000, 0.000000}
\pgfsetstrokecolor{dialinecolor}
\node[anchor=west] at (23.000000\du,59.000000\du){};
% setfont left to latex
\definecolor{dialinecolor}{rgb}{0.000000, 0.000000, 0.000000}
\pgfsetstrokecolor{dialinecolor}
\node at (22.000000\du,60.213188\du){$a_{0}$};
% setfont left to latex
\definecolor{dialinecolor}{rgb}{0.000000, 0.000000, 0.000000}
\pgfsetstrokecolor{dialinecolor}
\node at (22.000000\du,64.213188\du){$b_{0}$};
% setfont left to latex
\definecolor{dialinecolor}{rgb}{0.000000, 0.000000, 0.000000}
\pgfsetstrokecolor{dialinecolor}
\node at (29.000000\du,61.222500\du){$e_{10}$};
% setfont left to latex
\definecolor{dialinecolor}{rgb}{0.000000, 0.000000, 0.000000}
\pgfsetstrokecolor{dialinecolor}
\node at (29.000000\du,63.222500\du){$e_{01}$};
% setfont left to latex
\definecolor{dialinecolor}{rgb}{0.000000, 0.000000, 0.000000}
\pgfsetstrokecolor{dialinecolor}
\node at (28.000000\du,62.222500\du){$e_{00}$};
% setfont left to latex
\definecolor{dialinecolor}{rgb}{0.000000, 0.000000, 0.000000}
\pgfsetstrokecolor{dialinecolor}
\node at (30.000000\du,62.213188\du){$e_{11}$};
% setfont left to latex
\definecolor{dialinecolor}{rgb}{0.000000, 0.000000, 0.000000}
\pgfsetstrokecolor{dialinecolor}
\node at (36.000000\du,60.222500\du){$a_{1}$};
% setfont left to latex
\definecolor{dialinecolor}{rgb}{0.000000, 0.000000, 0.000000}
\pgfsetstrokecolor{dialinecolor}
\node at (36.000000\du,64.222500\du){$b_{1}$};
% setfont left to latex
\definecolor{dialinecolor}{rgb}{0.000000, 0.000000, 0.000000}
\pgfsetstrokecolor{dialinecolor}
\node at (36.000000\du,62.222500\du){$\Gamma$};
% setfont left to latex
\definecolor{dialinecolor}{rgb}{0.000000, 0.000000, 0.000000}
\pgfsetstrokecolor{dialinecolor}
\node at (35.000000\du,58.222500\du){DUT};
% setfont left to latex
\definecolor{dialinecolor}{rgb}{0.000000, 0.000000, 0.000000}
\pgfsetstrokecolor{dialinecolor}
\node at (31.000000\du,59.222500\du){Port 1};
% setfont left to latex
\definecolor{dialinecolor}{rgb}{0.000000, 0.000000, 0.000000}
\pgfsetstrokecolor{dialinecolor}
\node at (23.000000\du,62.213188\du){$\Gamma_{M}$};
\end{tikzpicture}

	\caption{One port model}
	\label{fig:oneportmodel}
\end{figure}


The measured S-parameter data can be de-embedded to the actual DUT with the lengths of the cables and the error coefficients of the network analyzer taken out of the data as if the DUT was measured directly.
	\begin{equation}
	\label{eqn:solcal}
	S11=\frac{S11_{M}-e_{00}}{(S11_{M}e_{11})-\Delta_{e}}
	\end{equation}
	\begin{itemize}
		\item $e_{00}$ is the Directivity
		\item $e_{11}$ is the port match
		\item $\Delta_{e} = e_{00}e_{11}-(e_{10}e_{01})$, of which $(e_{10}e_{01})$ is the tracking.
		\item $S11$ is the one port S-parameter that you want to display (De-embedded)
		\item 	$S11_M$ is the measured S-parameter including the cable and the errors of the port
	\end{itemize}

	The 3 error coefficients can be obtained from 3 independent measurements of known standards. The commonly used standards are Short, Open and Load, but any known standard can be used instead. (see section \ref{sec:obtainingerrorcoefsSOL})
\newpage
\subsubsection{Obtaining error coefficients for SOL calibration}
\label{sec:obtainingerrorcoefsSOL}

Equation \ref{eqn:solcal} contains 3 error coefficients $e_{00}$, $e_{11}$ and $\Delta_{e}$. From \cite{agilent_calibration} the equations \ref{eqn:obtaining1}, \ref{eqn:obtaining2} and \ref{eqn:obtaining3} are obtained. We see 3 times the same equations, but with different measurements. $\Gamma_1$, $\Gamma_2$ and $\Gamma_3$ are the known independent calibration standards, in this case Short, Open and Load. The standards don't have to be perfect though, a short can for instance have some series inductance or loss (see \ref{sec:calstds}). $\Gamma_{M1}$, $\Gamma_{M2}$ and $\Gamma_{M3}$ are the measured traces, this data is obtained by connecting the well-defined calibration standard to the network analyzer, through the cable that is also used in the measurement and measure the reflection (S11). Equations \ref{eqn:obtaining1}, \ref{eqn:obtaining2} and \ref{eqn:obtaining3} still contain our 3 unknown error coefficients $e_{00}$, $e_{11}$ and $\Delta_{e}$ that we need to solve equation \ref{eqn:solcal}.

	
	\small
	\begin{equation}
	\label{eqn:obtaining1}
	\Gamma_{M{1}} = e_{00}+\Gamma_{1}\Gamma_{M{1}}e{11}-\Gamma_{1}\Delta_{e}
	\end{equation}

	\begin{equation}
	\label{eqn:obtaining2}
	\Gamma_{M{2}} = e_{00}+\Gamma_{2}\Gamma_{M{2}}e{11}-\Gamma_{2}\Delta_{e}
	\end{equation}

	\begin{equation}
	\label{eqn:obtaining3}
	\Gamma_{M{3}} = e_{00}+\Gamma_{3}\Gamma_{M{3}}e{11}-\Gamma_{3}\Delta_{e}
	\end{equation}
	\normalsize

In order to solve $e_{00}$, $e_{11}$ and $\Delta_{e}$ we need to substitute equations  \ref{eqn:obtaining1}, \ref{eqn:obtaining2} and \ref{eqn:obtaining3} into one equation and extract the 3 error coefficients. The result is 3 lengthy equations, but with modern computers they can easily be computed for all the data points in our measurement.

	\small
	\begin{equation}
	\label{eqn:e00}
		e_{00} = -\frac{{\left(\Gamma_{2} \Gamma_{M_{3}} - \Gamma_{3} \Gamma_{M_{3}}\right)} \Gamma_{1} \Gamma_{M_{2}} - {\left(\Gamma_{2} \Gamma_{3} \Gamma_{M_{2}} - \Gamma_{2} \Gamma_{3} \Gamma_{M_{3}} - {\left(\Gamma_{3} \Gamma_{M_{2}} - \Gamma_{2} \Gamma_{M_{3}}\right)} \Gamma_{1}\right)} \Gamma_{M_{1}}}{\Gamma_{1} {\left(\Gamma_{2} - \Gamma_{3}\right)} \Gamma_{M_{1}} + \Gamma_{2} \Gamma_{3} \Gamma_{M_{2}} - \Gamma_{2} \Gamma_{3} \Gamma_{M_{3}} - {\left(\Gamma_{2} \Gamma_{M_{2}} - \Gamma_{3} \Gamma_{M_{3}}\right)} \Gamma_{1}}		
	\end{equation}
	\begin{equation}
	\label{eqn:e11}
	e_{11} = \frac{{\left(\Gamma_{2} - \Gamma_{3}\right)} \Gamma_{M_{1}} - \Gamma_{1} {\left(\Gamma_{M_{2}} - \Gamma_{M_{3}}\right)} + \Gamma_{3} \Gamma_{M_{2}} - \Gamma_{2} \Gamma_{M_{3}}}{\Gamma_{1} {\left(\Gamma_{2} - \Gamma_{3}\right)} \Gamma_{M_{1}} + \Gamma_{2} \Gamma_{3} \Gamma_{M_{2}} - \Gamma_{2} \Gamma_{3} \Gamma_{M_{3}} - {\left(\Gamma_{2} \Gamma_{M_{2}} - \Gamma_{3} \Gamma_{M_{3}}\right)} \Gamma_{1}}
	\end{equation}
	\begin{equation}
	\label{eqn:deltae}
	 \Delta_{e} = -\frac{{\left(\Gamma_{1} {\left(\Gamma_{M_{2}} - \Gamma_{M_{3}}\right)} - \Gamma_{2} \Gamma_{M_{2}} + \Gamma_{3} \Gamma_{M_{3}}\right)} \Gamma_{M_{1}} + {\left(\Gamma_{2} \Gamma_{M_{3}} - \Gamma_{3} \Gamma_{M_{3}}\right)} \Gamma_{M_{2}}}{\Gamma_{1} {\left(\Gamma_{2} - \Gamma_{3}\right)} \Gamma_{M_{1}} + \Gamma_{2} \Gamma_{3} \Gamma_{M_{2}} - \Gamma_{2} \Gamma_{3} \Gamma_{M_{3}} - {\left(\Gamma_{2} \Gamma_{M_{2}} - \Gamma_{3} \Gamma_{M_{3}}\right)} \Gamma_{1}}
	\end{equation}
	\normalsize
	
\newpage
\subsection{Full two port calibration}
\label{sec:soltical}
\begin{figure}[H]
	\centering
	% Graphic for TeX using PGF
% Title: /home/frans/DeEmbed/deembed/doc/figures/TwoPortModel.dia
% Creator: Dia v0.97.3
% CreationDate: Wed Feb 15 21:11:42 2017
% For: frans
% \usepackage{tikz}
% The following commands are not supported in PSTricks at present
% We define them conditionally, so when they are implemented,
% this pgf file will use them.
\ifx\du\undefined
  \newlength{\du}
\fi
\setlength{\du}{15\unitlength}
\begin{tikzpicture}[scale=0.75]
\pgftransformxscale{1.000000}
\pgftransformyscale{-1.000000}
\definecolor{dialinecolor}{rgb}{0.000000, 0.000000, 0.000000}
\pgfsetstrokecolor{dialinecolor}
\definecolor{dialinecolor}{rgb}{1.000000, 1.000000, 1.000000}
\pgfsetfillcolor{dialinecolor}
\definecolor{dialinecolor}{rgb}{1.000000, 0.996078, 0.870588}
\pgfsetfillcolor{dialinecolor}
\fill (34.000000\du,59.000000\du)--(34.000000\du,65.000000\du)--(40.000000\du,65.000000\du)--(40.000000\du,59.000000\du)--cycle;
\pgfsetlinewidth{0.100000\du}
\pgfsetdash{}{0pt}
\pgfsetdash{}{0pt}
\pgfsetmiterjoin
\definecolor{dialinecolor}{rgb}{0.000000, 0.000000, 0.000000}
\pgfsetstrokecolor{dialinecolor}
\draw (34.000000\du,59.000000\du)--(34.000000\du,65.000000\du)--(40.000000\du,65.000000\du)--(40.000000\du,59.000000\du)--cycle;
% setfont left to latex
\definecolor{dialinecolor}{rgb}{0.000000, 0.000000, 0.000000}
\pgfsetstrokecolor{dialinecolor}
\node at (37.000000\du,62.195000\du){};
\pgfsetlinewidth{0.100000\du}
\pgfsetdash{}{0pt}
\pgfsetdash{}{0pt}
\pgfsetbuttcap
{
\definecolor{dialinecolor}{rgb}{0.000000, 0.000000, 0.000000}
\pgfsetfillcolor{dialinecolor}
% was here!!!
\pgfsetarrowsend{latex}
\definecolor{dialinecolor}{rgb}{0.000000, 0.000000, 0.000000}
\pgfsetstrokecolor{dialinecolor}
\draw (22.750000\du,60.000000\du)--(25.250000\du,60.000000\du);
}
\definecolor{dialinecolor}{rgb}{0.000000, 0.000000, 0.000000}
\pgfsetstrokecolor{dialinecolor}
\draw (23.300000\du,60.000000\du)--(25.250000\du,60.000000\du);
\pgfsetlinewidth{0.100000\du}
\pgfsetdash{}{0pt}
\pgfsetmiterjoin
\pgfsetbuttcap
\definecolor{dialinecolor}{rgb}{1.000000, 1.000000, 1.000000}
\pgfsetfillcolor{dialinecolor}
\pgfpathmoveto{\pgfpoint{22.800000\du}{60.000000\du}}
\pgfpathcurveto{\pgfpoint{22.800000\du}{59.875000\du}}{\pgfpoint{22.925000\du}{59.750000\du}}{\pgfpoint{23.050000\du}{59.750000\du}}
\pgfpathcurveto{\pgfpoint{23.175000\du}{59.750000\du}}{\pgfpoint{23.300000\du}{59.875000\du}}{\pgfpoint{23.300000\du}{60.000000\du}}
\pgfpathcurveto{\pgfpoint{23.300000\du}{60.125000\du}}{\pgfpoint{23.175000\du}{60.250000\du}}{\pgfpoint{23.050000\du}{60.250000\du}}
\pgfpathcurveto{\pgfpoint{22.925000\du}{60.250000\du}}{\pgfpoint{22.800000\du}{60.125000\du}}{\pgfpoint{22.800000\du}{60.000000\du}}
\pgfusepath{fill}
\definecolor{dialinecolor}{rgb}{0.000000, 0.000000, 0.000000}
\pgfsetstrokecolor{dialinecolor}
\pgfpathmoveto{\pgfpoint{22.800000\du}{60.000000\du}}
\pgfpathcurveto{\pgfpoint{22.800000\du}{59.875000\du}}{\pgfpoint{22.925000\du}{59.750000\du}}{\pgfpoint{23.050000\du}{59.750000\du}}
\pgfpathcurveto{\pgfpoint{23.175000\du}{59.750000\du}}{\pgfpoint{23.300000\du}{59.875000\du}}{\pgfpoint{23.300000\du}{60.000000\du}}
\pgfpathcurveto{\pgfpoint{23.300000\du}{60.125000\du}}{\pgfpoint{23.175000\du}{60.250000\du}}{\pgfpoint{23.050000\du}{60.250000\du}}
\pgfpathcurveto{\pgfpoint{22.925000\du}{60.250000\du}}{\pgfpoint{22.800000\du}{60.125000\du}}{\pgfpoint{22.800000\du}{60.000000\du}}
\pgfusepath{stroke}
\pgfsetlinewidth{0.100000\du}
\pgfsetdash{}{0pt}
\pgfsetdash{}{0pt}
\pgfsetbuttcap
{
\definecolor{dialinecolor}{rgb}{0.000000, 0.000000, 0.000000}
\pgfsetfillcolor{dialinecolor}
% was here!!!
\definecolor{dialinecolor}{rgb}{0.000000, 0.000000, 0.000000}
\pgfsetstrokecolor{dialinecolor}
\draw (25.000000\du,60.000000\du)--(27.250000\du,60.000000\du);
}
\definecolor{dialinecolor}{rgb}{0.000000, 0.000000, 0.000000}
\pgfsetstrokecolor{dialinecolor}
\draw (25.000000\du,60.000000\du)--(27.250000\du,60.000000\du);
\pgfsetlinewidth{0.100000\du}
\pgfsetdash{}{0pt}
\pgfsetmiterjoin
\pgfsetbuttcap
\definecolor{dialinecolor}{rgb}{0.000000, 0.000000, 0.000000}
\pgfsetfillcolor{dialinecolor}
\pgfpathmoveto{\pgfpoint{27.250000\du}{60.000000\du}}
\pgfpathcurveto{\pgfpoint{27.250000\du}{60.125000\du}}{\pgfpoint{27.125000\du}{60.250000\du}}{\pgfpoint{27.000000\du}{60.250000\du}}
\pgfpathcurveto{\pgfpoint{26.875000\du}{60.250000\du}}{\pgfpoint{26.750000\du}{60.125000\du}}{\pgfpoint{26.750000\du}{60.000000\du}}
\pgfpathcurveto{\pgfpoint{26.750000\du}{59.875000\du}}{\pgfpoint{26.875000\du}{59.750000\du}}{\pgfpoint{27.000000\du}{59.750000\du}}
\pgfpathcurveto{\pgfpoint{27.125000\du}{59.750000\du}}{\pgfpoint{27.250000\du}{59.875000\du}}{\pgfpoint{27.250000\du}{60.000000\du}}
\pgfusepath{fill}
\definecolor{dialinecolor}{rgb}{0.000000, 0.000000, 0.000000}
\pgfsetstrokecolor{dialinecolor}
\pgfpathmoveto{\pgfpoint{27.250000\du}{60.000000\du}}
\pgfpathcurveto{\pgfpoint{27.250000\du}{60.125000\du}}{\pgfpoint{27.125000\du}{60.250000\du}}{\pgfpoint{27.000000\du}{60.250000\du}}
\pgfpathcurveto{\pgfpoint{26.875000\du}{60.250000\du}}{\pgfpoint{26.750000\du}{60.125000\du}}{\pgfpoint{26.750000\du}{60.000000\du}}
\pgfpathcurveto{\pgfpoint{26.750000\du}{59.875000\du}}{\pgfpoint{26.875000\du}{59.750000\du}}{\pgfpoint{27.000000\du}{59.750000\du}}
\pgfpathcurveto{\pgfpoint{27.125000\du}{59.750000\du}}{\pgfpoint{27.250000\du}{59.875000\du}}{\pgfpoint{27.250000\du}{60.000000\du}}
\pgfusepath{stroke}
\pgfsetlinewidth{0.100000\du}
\pgfsetdash{}{0pt}
\pgfsetdash{}{0pt}
\pgfsetbuttcap
{
\definecolor{dialinecolor}{rgb}{0.000000, 0.000000, 0.000000}
\pgfsetfillcolor{dialinecolor}
% was here!!!
\definecolor{dialinecolor}{rgb}{0.000000, 0.000000, 0.000000}
\pgfsetstrokecolor{dialinecolor}
\draw (22.750000\du,64.000000\du)--(25.000000\du,64.000000\du);
}
\definecolor{dialinecolor}{rgb}{0.000000, 0.000000, 0.000000}
\pgfsetstrokecolor{dialinecolor}
\draw (23.300000\du,64.000000\du)--(25.000000\du,64.000000\du);
\pgfsetlinewidth{0.100000\du}
\pgfsetdash{}{0pt}
\pgfsetmiterjoin
\pgfsetbuttcap
\definecolor{dialinecolor}{rgb}{1.000000, 1.000000, 1.000000}
\pgfsetfillcolor{dialinecolor}
\pgfpathmoveto{\pgfpoint{22.800000\du}{64.000000\du}}
\pgfpathcurveto{\pgfpoint{22.800000\du}{63.875000\du}}{\pgfpoint{22.925000\du}{63.750000\du}}{\pgfpoint{23.050000\du}{63.750000\du}}
\pgfpathcurveto{\pgfpoint{23.175000\du}{63.750000\du}}{\pgfpoint{23.300000\du}{63.875000\du}}{\pgfpoint{23.300000\du}{64.000000\du}}
\pgfpathcurveto{\pgfpoint{23.300000\du}{64.125000\du}}{\pgfpoint{23.175000\du}{64.250000\du}}{\pgfpoint{23.050000\du}{64.250000\du}}
\pgfpathcurveto{\pgfpoint{22.925000\du}{64.250000\du}}{\pgfpoint{22.800000\du}{64.125000\du}}{\pgfpoint{22.800000\du}{64.000000\du}}
\pgfusepath{fill}
\definecolor{dialinecolor}{rgb}{0.000000, 0.000000, 0.000000}
\pgfsetstrokecolor{dialinecolor}
\pgfpathmoveto{\pgfpoint{22.800000\du}{64.000000\du}}
\pgfpathcurveto{\pgfpoint{22.800000\du}{63.875000\du}}{\pgfpoint{22.925000\du}{63.750000\du}}{\pgfpoint{23.050000\du}{63.750000\du}}
\pgfpathcurveto{\pgfpoint{23.175000\du}{63.750000\du}}{\pgfpoint{23.300000\du}{63.875000\du}}{\pgfpoint{23.300000\du}{64.000000\du}}
\pgfpathcurveto{\pgfpoint{23.300000\du}{64.125000\du}}{\pgfpoint{23.175000\du}{64.250000\du}}{\pgfpoint{23.050000\du}{64.250000\du}}
\pgfpathcurveto{\pgfpoint{22.925000\du}{64.250000\du}}{\pgfpoint{22.800000\du}{64.125000\du}}{\pgfpoint{22.800000\du}{64.000000\du}}
\pgfusepath{stroke}
\pgfsetlinewidth{0.100000\du}
\pgfsetdash{}{0pt}
\pgfsetdash{}{0pt}
\pgfsetbuttcap
{
\definecolor{dialinecolor}{rgb}{0.000000, 0.000000, 0.000000}
\pgfsetfillcolor{dialinecolor}
% was here!!!
\pgfsetarrowsstart{latex}
\definecolor{dialinecolor}{rgb}{0.000000, 0.000000, 0.000000}
\pgfsetstrokecolor{dialinecolor}
\draw (24.750000\du,64.000000\du)--(27.250000\du,64.000000\du);
}
\definecolor{dialinecolor}{rgb}{0.000000, 0.000000, 0.000000}
\pgfsetstrokecolor{dialinecolor}
\draw (24.750000\du,64.000000\du)--(27.250000\du,64.000000\du);
\pgfsetlinewidth{0.100000\du}
\pgfsetdash{}{0pt}
\pgfsetmiterjoin
\pgfsetbuttcap
\definecolor{dialinecolor}{rgb}{0.000000, 0.000000, 0.000000}
\pgfsetfillcolor{dialinecolor}
\pgfpathmoveto{\pgfpoint{27.250000\du}{64.000000\du}}
\pgfpathcurveto{\pgfpoint{27.250000\du}{64.125000\du}}{\pgfpoint{27.125000\du}{64.250000\du}}{\pgfpoint{27.000000\du}{64.250000\du}}
\pgfpathcurveto{\pgfpoint{26.875000\du}{64.250000\du}}{\pgfpoint{26.750000\du}{64.125000\du}}{\pgfpoint{26.750000\du}{64.000000\du}}
\pgfpathcurveto{\pgfpoint{26.750000\du}{63.875000\du}}{\pgfpoint{26.875000\du}{63.750000\du}}{\pgfpoint{27.000000\du}{63.750000\du}}
\pgfpathcurveto{\pgfpoint{27.125000\du}{63.750000\du}}{\pgfpoint{27.250000\du}{63.875000\du}}{\pgfpoint{27.250000\du}{64.000000\du}}
\pgfusepath{fill}
\definecolor{dialinecolor}{rgb}{0.000000, 0.000000, 0.000000}
\pgfsetstrokecolor{dialinecolor}
\pgfpathmoveto{\pgfpoint{27.250000\du}{64.000000\du}}
\pgfpathcurveto{\pgfpoint{27.250000\du}{64.125000\du}}{\pgfpoint{27.125000\du}{64.250000\du}}{\pgfpoint{27.000000\du}{64.250000\du}}
\pgfpathcurveto{\pgfpoint{26.875000\du}{64.250000\du}}{\pgfpoint{26.750000\du}{64.125000\du}}{\pgfpoint{26.750000\du}{64.000000\du}}
\pgfpathcurveto{\pgfpoint{26.750000\du}{63.875000\du}}{\pgfpoint{26.875000\du}{63.750000\du}}{\pgfpoint{27.000000\du}{63.750000\du}}
\pgfpathcurveto{\pgfpoint{27.125000\du}{63.750000\du}}{\pgfpoint{27.250000\du}{63.875000\du}}{\pgfpoint{27.250000\du}{64.000000\du}}
\pgfusepath{stroke}
\pgfsetlinewidth{0.100000\du}
\pgfsetdash{}{0pt}
\pgfsetdash{}{0pt}
\pgfsetbuttcap
{
\definecolor{dialinecolor}{rgb}{0.000000, 0.000000, 0.000000}
\pgfsetfillcolor{dialinecolor}
% was here!!!
\pgfsetarrowsend{latex}
\definecolor{dialinecolor}{rgb}{0.000000, 0.000000, 0.000000}
\pgfsetstrokecolor{dialinecolor}
\draw (27.000000\du,60.000000\du)--(27.000000\du,62.250000\du);
}
\pgfsetlinewidth{0.100000\du}
\pgfsetdash{}{0pt}
\pgfsetdash{}{0pt}
\pgfsetbuttcap
{
\definecolor{dialinecolor}{rgb}{0.000000, 0.000000, 0.000000}
\pgfsetfillcolor{dialinecolor}
% was here!!!
\definecolor{dialinecolor}{rgb}{0.000000, 0.000000, 0.000000}
\pgfsetstrokecolor{dialinecolor}
\draw (27.000000\du,62.000000\du)--(27.000000\du,64.000000\du);
}
\pgfsetlinewidth{0.100000\du}
\pgfsetdash{}{0pt}
\pgfsetdash{}{0pt}
\pgfsetbuttcap
{
\definecolor{dialinecolor}{rgb}{0.000000, 0.000000, 0.000000}
\pgfsetfillcolor{dialinecolor}
% was here!!!
\pgfsetarrowsend{latex}
\definecolor{dialinecolor}{rgb}{0.000000, 0.000000, 0.000000}
\pgfsetstrokecolor{dialinecolor}
\draw (27.000000\du,60.000000\du)--(29.250000\du,60.000000\du);
}
\pgfsetlinewidth{0.100000\du}
\pgfsetdash{}{0pt}
\pgfsetdash{}{0pt}
\pgfsetbuttcap
{
\definecolor{dialinecolor}{rgb}{0.000000, 0.000000, 0.000000}
\pgfsetfillcolor{dialinecolor}
% was here!!!
\definecolor{dialinecolor}{rgb}{0.000000, 0.000000, 0.000000}
\pgfsetstrokecolor{dialinecolor}
\draw (29.000000\du,60.000000\du)--(31.250000\du,60.000000\du);
}
\definecolor{dialinecolor}{rgb}{0.000000, 0.000000, 0.000000}
\pgfsetstrokecolor{dialinecolor}
\draw (29.000000\du,60.000000\du)--(31.250000\du,60.000000\du);
\pgfsetlinewidth{0.100000\du}
\pgfsetdash{}{0pt}
\pgfsetmiterjoin
\pgfsetbuttcap
\definecolor{dialinecolor}{rgb}{0.000000, 0.000000, 0.000000}
\pgfsetfillcolor{dialinecolor}
\pgfpathmoveto{\pgfpoint{31.250000\du}{60.000000\du}}
\pgfpathcurveto{\pgfpoint{31.250000\du}{60.125000\du}}{\pgfpoint{31.125000\du}{60.250000\du}}{\pgfpoint{31.000000\du}{60.250000\du}}
\pgfpathcurveto{\pgfpoint{30.875000\du}{60.250000\du}}{\pgfpoint{30.750000\du}{60.125000\du}}{\pgfpoint{30.750000\du}{60.000000\du}}
\pgfpathcurveto{\pgfpoint{30.750000\du}{59.875000\du}}{\pgfpoint{30.875000\du}{59.750000\du}}{\pgfpoint{31.000000\du}{59.750000\du}}
\pgfpathcurveto{\pgfpoint{31.125000\du}{59.750000\du}}{\pgfpoint{31.250000\du}{59.875000\du}}{\pgfpoint{31.250000\du}{60.000000\du}}
\pgfusepath{fill}
\definecolor{dialinecolor}{rgb}{0.000000, 0.000000, 0.000000}
\pgfsetstrokecolor{dialinecolor}
\pgfpathmoveto{\pgfpoint{31.250000\du}{60.000000\du}}
\pgfpathcurveto{\pgfpoint{31.250000\du}{60.125000\du}}{\pgfpoint{31.125000\du}{60.250000\du}}{\pgfpoint{31.000000\du}{60.250000\du}}
\pgfpathcurveto{\pgfpoint{30.875000\du}{60.250000\du}}{\pgfpoint{30.750000\du}{60.125000\du}}{\pgfpoint{30.750000\du}{60.000000\du}}
\pgfpathcurveto{\pgfpoint{30.750000\du}{59.875000\du}}{\pgfpoint{30.875000\du}{59.750000\du}}{\pgfpoint{31.000000\du}{59.750000\du}}
\pgfpathcurveto{\pgfpoint{31.125000\du}{59.750000\du}}{\pgfpoint{31.250000\du}{59.875000\du}}{\pgfpoint{31.250000\du}{60.000000\du}}
\pgfusepath{stroke}
\pgfsetlinewidth{0.100000\du}
\pgfsetdash{}{0pt}
\pgfsetdash{}{0pt}
\pgfsetbuttcap
{
\definecolor{dialinecolor}{rgb}{0.000000, 0.000000, 0.000000}
\pgfsetfillcolor{dialinecolor}
% was here!!!
\pgfsetarrowsend{latex}
\definecolor{dialinecolor}{rgb}{0.000000, 0.000000, 0.000000}
\pgfsetstrokecolor{dialinecolor}
\draw (31.000000\du,60.000000\du)--(33.250000\du,60.000000\du);
}
\pgfsetlinewidth{0.100000\du}
\pgfsetdash{}{0pt}
\pgfsetdash{}{0pt}
\pgfsetbuttcap
{
\definecolor{dialinecolor}{rgb}{0.000000, 0.000000, 0.000000}
\pgfsetfillcolor{dialinecolor}
% was here!!!
\definecolor{dialinecolor}{rgb}{0.000000, 0.000000, 0.000000}
\pgfsetstrokecolor{dialinecolor}
\draw (33.000000\du,60.000000\du)--(35.250000\du,60.000000\du);
}
\definecolor{dialinecolor}{rgb}{0.000000, 0.000000, 0.000000}
\pgfsetstrokecolor{dialinecolor}
\draw (33.000000\du,60.000000\du)--(35.250000\du,60.000000\du);
\pgfsetlinewidth{0.100000\du}
\pgfsetdash{}{0pt}
\pgfsetmiterjoin
\pgfsetbuttcap
\definecolor{dialinecolor}{rgb}{0.000000, 0.000000, 0.000000}
\pgfsetfillcolor{dialinecolor}
\pgfpathmoveto{\pgfpoint{35.250000\du}{60.000000\du}}
\pgfpathcurveto{\pgfpoint{35.250000\du}{60.125000\du}}{\pgfpoint{35.125000\du}{60.250000\du}}{\pgfpoint{35.000000\du}{60.250000\du}}
\pgfpathcurveto{\pgfpoint{34.875000\du}{60.250000\du}}{\pgfpoint{34.750000\du}{60.125000\du}}{\pgfpoint{34.750000\du}{60.000000\du}}
\pgfpathcurveto{\pgfpoint{34.750000\du}{59.875000\du}}{\pgfpoint{34.875000\du}{59.750000\du}}{\pgfpoint{35.000000\du}{59.750000\du}}
\pgfpathcurveto{\pgfpoint{35.125000\du}{59.750000\du}}{\pgfpoint{35.250000\du}{59.875000\du}}{\pgfpoint{35.250000\du}{60.000000\du}}
\pgfusepath{fill}
\definecolor{dialinecolor}{rgb}{0.000000, 0.000000, 0.000000}
\pgfsetstrokecolor{dialinecolor}
\pgfpathmoveto{\pgfpoint{35.250000\du}{60.000000\du}}
\pgfpathcurveto{\pgfpoint{35.250000\du}{60.125000\du}}{\pgfpoint{35.125000\du}{60.250000\du}}{\pgfpoint{35.000000\du}{60.250000\du}}
\pgfpathcurveto{\pgfpoint{34.875000\du}{60.250000\du}}{\pgfpoint{34.750000\du}{60.125000\du}}{\pgfpoint{34.750000\du}{60.000000\du}}
\pgfpathcurveto{\pgfpoint{34.750000\du}{59.875000\du}}{\pgfpoint{34.875000\du}{59.750000\du}}{\pgfpoint{35.000000\du}{59.750000\du}}
\pgfpathcurveto{\pgfpoint{35.125000\du}{59.750000\du}}{\pgfpoint{35.250000\du}{59.875000\du}}{\pgfpoint{35.250000\du}{60.000000\du}}
\pgfusepath{stroke}
\pgfsetlinewidth{0.100000\du}
\pgfsetdash{}{0pt}
\pgfsetdash{}{0pt}
\pgfsetbuttcap
{
\definecolor{dialinecolor}{rgb}{0.000000, 0.000000, 0.000000}
\pgfsetfillcolor{dialinecolor}
% was here!!!
\pgfsetarrowsend{latex}
\definecolor{dialinecolor}{rgb}{0.000000, 0.000000, 0.000000}
\pgfsetstrokecolor{dialinecolor}
\draw (31.000000\du,64.000000\du)--(31.000000\du,61.750000\du);
}
\pgfsetlinewidth{0.100000\du}
\pgfsetdash{}{0pt}
\pgfsetdash{}{0pt}
\pgfsetbuttcap
{
\definecolor{dialinecolor}{rgb}{0.000000, 0.000000, 0.000000}
\pgfsetfillcolor{dialinecolor}
% was here!!!
\definecolor{dialinecolor}{rgb}{0.000000, 0.000000, 0.000000}
\pgfsetstrokecolor{dialinecolor}
\draw (31.000000\du,60.000000\du)--(31.000000\du,62.000000\du);
}
\pgfsetlinewidth{0.100000\du}
\pgfsetdash{}{0pt}
\pgfsetdash{}{0pt}
\pgfsetbuttcap
{
\definecolor{dialinecolor}{rgb}{0.000000, 0.000000, 0.000000}
\pgfsetfillcolor{dialinecolor}
% was here!!!
\pgfsetarrowsend{latex}
\definecolor{dialinecolor}{rgb}{0.000000, 0.000000, 0.000000}
\pgfsetstrokecolor{dialinecolor}
\draw (31.250000\du,64.000000\du)--(28.750000\du,64.000000\du);
}
\definecolor{dialinecolor}{rgb}{0.000000, 0.000000, 0.000000}
\pgfsetstrokecolor{dialinecolor}
\draw (31.250000\du,64.000000\du)--(28.750000\du,64.000000\du);
\pgfsetlinewidth{0.100000\du}
\pgfsetdash{}{0pt}
\pgfsetmiterjoin
\pgfsetbuttcap
\definecolor{dialinecolor}{rgb}{0.000000, 0.000000, 0.000000}
\pgfsetfillcolor{dialinecolor}
\pgfpathmoveto{\pgfpoint{31.250000\du}{64.000000\du}}
\pgfpathcurveto{\pgfpoint{31.250000\du}{64.125000\du}}{\pgfpoint{31.125000\du}{64.250000\du}}{\pgfpoint{31.000000\du}{64.250000\du}}
\pgfpathcurveto{\pgfpoint{30.875000\du}{64.250000\du}}{\pgfpoint{30.750000\du}{64.125000\du}}{\pgfpoint{30.750000\du}{64.000000\du}}
\pgfpathcurveto{\pgfpoint{30.750000\du}{63.875000\du}}{\pgfpoint{30.875000\du}{63.750000\du}}{\pgfpoint{31.000000\du}{63.750000\du}}
\pgfpathcurveto{\pgfpoint{31.125000\du}{63.750000\du}}{\pgfpoint{31.250000\du}{63.875000\du}}{\pgfpoint{31.250000\du}{64.000000\du}}
\pgfusepath{fill}
\definecolor{dialinecolor}{rgb}{0.000000, 0.000000, 0.000000}
\pgfsetstrokecolor{dialinecolor}
\pgfpathmoveto{\pgfpoint{31.250000\du}{64.000000\du}}
\pgfpathcurveto{\pgfpoint{31.250000\du}{64.125000\du}}{\pgfpoint{31.125000\du}{64.250000\du}}{\pgfpoint{31.000000\du}{64.250000\du}}
\pgfpathcurveto{\pgfpoint{30.875000\du}{64.250000\du}}{\pgfpoint{30.750000\du}{64.125000\du}}{\pgfpoint{30.750000\du}{64.000000\du}}
\pgfpathcurveto{\pgfpoint{30.750000\du}{63.875000\du}}{\pgfpoint{30.875000\du}{63.750000\du}}{\pgfpoint{31.000000\du}{63.750000\du}}
\pgfpathcurveto{\pgfpoint{31.125000\du}{63.750000\du}}{\pgfpoint{31.250000\du}{63.875000\du}}{\pgfpoint{31.250000\du}{64.000000\du}}
\pgfusepath{stroke}
\pgfsetlinewidth{0.100000\du}
\pgfsetdash{}{0pt}
\pgfsetdash{}{0pt}
\pgfsetbuttcap
{
\definecolor{dialinecolor}{rgb}{0.000000, 0.000000, 0.000000}
\pgfsetfillcolor{dialinecolor}
% was here!!!
\definecolor{dialinecolor}{rgb}{0.000000, 0.000000, 0.000000}
\pgfsetstrokecolor{dialinecolor}
\draw (27.000000\du,64.000000\du)--(29.000000\du,64.000000\du);
}
\pgfsetlinewidth{0.100000\du}
\pgfsetdash{}{0pt}
\pgfsetdash{}{0pt}
\pgfsetbuttcap
{
\definecolor{dialinecolor}{rgb}{0.000000, 0.000000, 0.000000}
\pgfsetfillcolor{dialinecolor}
% was here!!!
\pgfsetarrowsend{latex}
\definecolor{dialinecolor}{rgb}{0.000000, 0.000000, 0.000000}
\pgfsetstrokecolor{dialinecolor}
\draw (35.250000\du,64.000000\du)--(32.750000\du,64.000000\du);
}
\definecolor{dialinecolor}{rgb}{0.000000, 0.000000, 0.000000}
\pgfsetstrokecolor{dialinecolor}
\draw (35.250000\du,64.000000\du)--(32.750000\du,64.000000\du);
\pgfsetlinewidth{0.100000\du}
\pgfsetdash{}{0pt}
\pgfsetmiterjoin
\pgfsetbuttcap
\definecolor{dialinecolor}{rgb}{0.000000, 0.000000, 0.000000}
\pgfsetfillcolor{dialinecolor}
\pgfpathmoveto{\pgfpoint{35.250000\du}{64.000000\du}}
\pgfpathcurveto{\pgfpoint{35.250000\du}{64.125000\du}}{\pgfpoint{35.125000\du}{64.250000\du}}{\pgfpoint{35.000000\du}{64.250000\du}}
\pgfpathcurveto{\pgfpoint{34.875000\du}{64.250000\du}}{\pgfpoint{34.750000\du}{64.125000\du}}{\pgfpoint{34.750000\du}{64.000000\du}}
\pgfpathcurveto{\pgfpoint{34.750000\du}{63.875000\du}}{\pgfpoint{34.875000\du}{63.750000\du}}{\pgfpoint{35.000000\du}{63.750000\du}}
\pgfpathcurveto{\pgfpoint{35.125000\du}{63.750000\du}}{\pgfpoint{35.250000\du}{63.875000\du}}{\pgfpoint{35.250000\du}{64.000000\du}}
\pgfusepath{fill}
\definecolor{dialinecolor}{rgb}{0.000000, 0.000000, 0.000000}
\pgfsetstrokecolor{dialinecolor}
\pgfpathmoveto{\pgfpoint{35.250000\du}{64.000000\du}}
\pgfpathcurveto{\pgfpoint{35.250000\du}{64.125000\du}}{\pgfpoint{35.125000\du}{64.250000\du}}{\pgfpoint{35.000000\du}{64.250000\du}}
\pgfpathcurveto{\pgfpoint{34.875000\du}{64.250000\du}}{\pgfpoint{34.750000\du}{64.125000\du}}{\pgfpoint{34.750000\du}{64.000000\du}}
\pgfpathcurveto{\pgfpoint{34.750000\du}{63.875000\du}}{\pgfpoint{34.875000\du}{63.750000\du}}{\pgfpoint{35.000000\du}{63.750000\du}}
\pgfpathcurveto{\pgfpoint{35.125000\du}{63.750000\du}}{\pgfpoint{35.250000\du}{63.875000\du}}{\pgfpoint{35.250000\du}{64.000000\du}}
\pgfusepath{stroke}
\pgfsetlinewidth{0.100000\du}
\pgfsetdash{}{0pt}
\pgfsetdash{}{0pt}
\pgfsetbuttcap
{
\definecolor{dialinecolor}{rgb}{0.000000, 0.000000, 0.000000}
\pgfsetfillcolor{dialinecolor}
% was here!!!
\definecolor{dialinecolor}{rgb}{0.000000, 0.000000, 0.000000}
\pgfsetstrokecolor{dialinecolor}
\draw (31.000000\du,64.000000\du)--(33.000000\du,64.000000\du);
}
\pgfsetlinewidth{0.100000\du}
\pgfsetdash{}{0pt}
\pgfsetdash{}{0pt}
\pgfsetbuttcap
{
\definecolor{dialinecolor}{rgb}{0.000000, 0.000000, 0.000000}
\pgfsetfillcolor{dialinecolor}
% was here!!!
\pgfsetarrowsend{latex}
\definecolor{dialinecolor}{rgb}{0.000000, 0.000000, 0.000000}
\pgfsetstrokecolor{dialinecolor}
\draw (35.000000\du,60.000000\du)--(35.000000\du,62.250000\du);
}
\pgfsetlinewidth{0.100000\du}
\pgfsetdash{}{0pt}
\pgfsetdash{}{0pt}
\pgfsetbuttcap
{
\definecolor{dialinecolor}{rgb}{0.000000, 0.000000, 0.000000}
\pgfsetfillcolor{dialinecolor}
% was here!!!
\definecolor{dialinecolor}{rgb}{0.000000, 0.000000, 0.000000}
\pgfsetstrokecolor{dialinecolor}
\draw (35.000000\du,62.000000\du)--(35.000000\du,64.000000\du);
}
% setfont left to latex
\definecolor{dialinecolor}{rgb}{0.000000, 0.000000, 0.000000}
\pgfsetstrokecolor{dialinecolor}
\node[anchor=west] at (23.000000\du,59.000000\du){};
% setfont left to latex
\definecolor{dialinecolor}{rgb}{0.000000, 0.000000, 0.000000}
\pgfsetstrokecolor{dialinecolor}
\node at (22.000000\du,60.222500\du){$a_{0}$};
% setfont left to latex
\definecolor{dialinecolor}{rgb}{0.000000, 0.000000, 0.000000}
\pgfsetstrokecolor{dialinecolor}
\node at (22.000000\du,64.222500\du){$b_{0}$};
% setfont left to latex
\definecolor{dialinecolor}{rgb}{0.000000, 0.000000, 0.000000}
\pgfsetstrokecolor{dialinecolor}
\node at (29.000000\du,60.949900\du){$e_{10}$};
% setfont left to latex
\definecolor{dialinecolor}{rgb}{0.000000, 0.000000, 0.000000}
\pgfsetstrokecolor{dialinecolor}
\node at (29.000000\du,63.495100\du){$e_{01}$};
% setfont left to latex
\definecolor{dialinecolor}{rgb}{0.000000, 0.000000, 0.000000}
\pgfsetstrokecolor{dialinecolor}
\node at (27.809200\du,62.222500\du){$e_{00}$};
% setfont left to latex
\definecolor{dialinecolor}{rgb}{0.000000, 0.000000, 0.000000}
\pgfsetstrokecolor{dialinecolor}
\node at (30.190800\du,62.222500\du){$e_{11}$};
% setfont left to latex
\definecolor{dialinecolor}{rgb}{0.000000, 0.000000, 0.000000}
\pgfsetstrokecolor{dialinecolor}
\node at (35.000000\du,59.444800\du){$a_{1}$};
% setfont left to latex
\definecolor{dialinecolor}{rgb}{0.000000, 0.000000, 0.000000}
\pgfsetstrokecolor{dialinecolor}
\node at (35.000000\du,64.712500\du){$b_{1}$};
% setfont left to latex
\definecolor{dialinecolor}{rgb}{0.000000, 0.000000, 0.000000}
\pgfsetstrokecolor{dialinecolor}
\node at (36.000000\du,62.222500\du){$S_{11}$};
% setfont left to latex
\definecolor{dialinecolor}{rgb}{0.000000, 0.000000, 0.000000}
\pgfsetstrokecolor{dialinecolor}
\node at (37.000000\du,58.222500\du){DUT};
% setfont left to latex
\definecolor{dialinecolor}{rgb}{0.000000, 0.000000, 0.000000}
\pgfsetstrokecolor{dialinecolor}
\node at (31.000000\du,59.222500\du){Port 1};
\pgfsetlinewidth{0.100000\du}
\pgfsetdash{}{0pt}
\pgfsetdash{}{0pt}
\pgfsetbuttcap
{
\definecolor{dialinecolor}{rgb}{0.000000, 0.000000, 0.000000}
\pgfsetfillcolor{dialinecolor}
% was here!!!
\pgfsetarrowsend{latex}
\definecolor{dialinecolor}{rgb}{0.000000, 0.000000, 0.000000}
\pgfsetstrokecolor{dialinecolor}
\draw (35.000000\du,60.000000\du)--(37.250000\du,60.000000\du);
}
\pgfsetlinewidth{0.100000\du}
\pgfsetdash{}{0pt}
\pgfsetdash{}{0pt}
\pgfsetbuttcap
{
\definecolor{dialinecolor}{rgb}{0.000000, 0.000000, 0.000000}
\pgfsetfillcolor{dialinecolor}
% was here!!!
\definecolor{dialinecolor}{rgb}{0.000000, 0.000000, 0.000000}
\pgfsetstrokecolor{dialinecolor}
\draw (37.000000\du,60.000000\du)--(39.250000\du,60.000000\du);
}
\definecolor{dialinecolor}{rgb}{0.000000, 0.000000, 0.000000}
\pgfsetstrokecolor{dialinecolor}
\draw (37.000000\du,60.000000\du)--(39.250000\du,60.000000\du);
\pgfsetlinewidth{0.100000\du}
\pgfsetdash{}{0pt}
\pgfsetmiterjoin
\pgfsetbuttcap
\definecolor{dialinecolor}{rgb}{0.000000, 0.000000, 0.000000}
\pgfsetfillcolor{dialinecolor}
\pgfpathmoveto{\pgfpoint{39.250000\du}{60.000000\du}}
\pgfpathcurveto{\pgfpoint{39.250000\du}{60.125000\du}}{\pgfpoint{39.125000\du}{60.250000\du}}{\pgfpoint{39.000000\du}{60.250000\du}}
\pgfpathcurveto{\pgfpoint{38.875000\du}{60.250000\du}}{\pgfpoint{38.750000\du}{60.125000\du}}{\pgfpoint{38.750000\du}{60.000000\du}}
\pgfpathcurveto{\pgfpoint{38.750000\du}{59.875000\du}}{\pgfpoint{38.875000\du}{59.750000\du}}{\pgfpoint{39.000000\du}{59.750000\du}}
\pgfpathcurveto{\pgfpoint{39.125000\du}{59.750000\du}}{\pgfpoint{39.250000\du}{59.875000\du}}{\pgfpoint{39.250000\du}{60.000000\du}}
\pgfusepath{fill}
\definecolor{dialinecolor}{rgb}{0.000000, 0.000000, 0.000000}
\pgfsetstrokecolor{dialinecolor}
\pgfpathmoveto{\pgfpoint{39.250000\du}{60.000000\du}}
\pgfpathcurveto{\pgfpoint{39.250000\du}{60.125000\du}}{\pgfpoint{39.125000\du}{60.250000\du}}{\pgfpoint{39.000000\du}{60.250000\du}}
\pgfpathcurveto{\pgfpoint{38.875000\du}{60.250000\du}}{\pgfpoint{38.750000\du}{60.125000\du}}{\pgfpoint{38.750000\du}{60.000000\du}}
\pgfpathcurveto{\pgfpoint{38.750000\du}{59.875000\du}}{\pgfpoint{38.875000\du}{59.750000\du}}{\pgfpoint{39.000000\du}{59.750000\du}}
\pgfpathcurveto{\pgfpoint{39.125000\du}{59.750000\du}}{\pgfpoint{39.250000\du}{59.875000\du}}{\pgfpoint{39.250000\du}{60.000000\du}}
\pgfusepath{stroke}
\pgfsetlinewidth{0.100000\du}
\pgfsetdash{}{0pt}
\pgfsetdash{}{0pt}
\pgfsetbuttcap
{
\definecolor{dialinecolor}{rgb}{0.000000, 0.000000, 0.000000}
\pgfsetfillcolor{dialinecolor}
% was here!!!
\pgfsetarrowsend{latex}
\definecolor{dialinecolor}{rgb}{0.000000, 0.000000, 0.000000}
\pgfsetstrokecolor{dialinecolor}
\draw (39.000000\du,64.000000\du)--(39.000000\du,61.750000\du);
}
\pgfsetlinewidth{0.100000\du}
\pgfsetdash{}{0pt}
\pgfsetdash{}{0pt}
\pgfsetbuttcap
{
\definecolor{dialinecolor}{rgb}{0.000000, 0.000000, 0.000000}
\pgfsetfillcolor{dialinecolor}
% was here!!!
\definecolor{dialinecolor}{rgb}{0.000000, 0.000000, 0.000000}
\pgfsetstrokecolor{dialinecolor}
\draw (39.000000\du,60.000000\du)--(39.000000\du,62.000000\du);
}
\pgfsetlinewidth{0.100000\du}
\pgfsetdash{}{0pt}
\pgfsetdash{}{0pt}
\pgfsetbuttcap
{
\definecolor{dialinecolor}{rgb}{0.000000, 0.000000, 0.000000}
\pgfsetfillcolor{dialinecolor}
% was here!!!
\pgfsetarrowsend{latex}
\definecolor{dialinecolor}{rgb}{0.000000, 0.000000, 0.000000}
\pgfsetstrokecolor{dialinecolor}
\draw (39.250000\du,64.000000\du)--(36.750000\du,64.000000\du);
}
\definecolor{dialinecolor}{rgb}{0.000000, 0.000000, 0.000000}
\pgfsetstrokecolor{dialinecolor}
\draw (39.250000\du,64.000000\du)--(36.750000\du,64.000000\du);
\pgfsetlinewidth{0.100000\du}
\pgfsetdash{}{0pt}
\pgfsetmiterjoin
\pgfsetbuttcap
\definecolor{dialinecolor}{rgb}{0.000000, 0.000000, 0.000000}
\pgfsetfillcolor{dialinecolor}
\pgfpathmoveto{\pgfpoint{39.250000\du}{64.000000\du}}
\pgfpathcurveto{\pgfpoint{39.250000\du}{64.125000\du}}{\pgfpoint{39.125000\du}{64.250000\du}}{\pgfpoint{39.000000\du}{64.250000\du}}
\pgfpathcurveto{\pgfpoint{38.875000\du}{64.250000\du}}{\pgfpoint{38.750000\du}{64.125000\du}}{\pgfpoint{38.750000\du}{64.000000\du}}
\pgfpathcurveto{\pgfpoint{38.750000\du}{63.875000\du}}{\pgfpoint{38.875000\du}{63.750000\du}}{\pgfpoint{39.000000\du}{63.750000\du}}
\pgfpathcurveto{\pgfpoint{39.125000\du}{63.750000\du}}{\pgfpoint{39.250000\du}{63.875000\du}}{\pgfpoint{39.250000\du}{64.000000\du}}
\pgfusepath{fill}
\definecolor{dialinecolor}{rgb}{0.000000, 0.000000, 0.000000}
\pgfsetstrokecolor{dialinecolor}
\pgfpathmoveto{\pgfpoint{39.250000\du}{64.000000\du}}
\pgfpathcurveto{\pgfpoint{39.250000\du}{64.125000\du}}{\pgfpoint{39.125000\du}{64.250000\du}}{\pgfpoint{39.000000\du}{64.250000\du}}
\pgfpathcurveto{\pgfpoint{38.875000\du}{64.250000\du}}{\pgfpoint{38.750000\du}{64.125000\du}}{\pgfpoint{38.750000\du}{64.000000\du}}
\pgfpathcurveto{\pgfpoint{38.750000\du}{63.875000\du}}{\pgfpoint{38.875000\du}{63.750000\du}}{\pgfpoint{39.000000\du}{63.750000\du}}
\pgfpathcurveto{\pgfpoint{39.125000\du}{63.750000\du}}{\pgfpoint{39.250000\du}{63.875000\du}}{\pgfpoint{39.250000\du}{64.000000\du}}
\pgfusepath{stroke}
\pgfsetlinewidth{0.100000\du}
\pgfsetdash{}{0pt}
\pgfsetdash{}{0pt}
\pgfsetbuttcap
{
\definecolor{dialinecolor}{rgb}{0.000000, 0.000000, 0.000000}
\pgfsetfillcolor{dialinecolor}
% was here!!!
\definecolor{dialinecolor}{rgb}{0.000000, 0.000000, 0.000000}
\pgfsetstrokecolor{dialinecolor}
\draw (35.000000\du,64.000000\du)--(37.000000\du,64.000000\du);
}
% setfont left to latex
\definecolor{dialinecolor}{rgb}{0.000000, 0.000000, 0.000000}
\pgfsetstrokecolor{dialinecolor}
\node at (39.000000\du,64.714400\du){$a_{2}$};
% setfont left to latex
\definecolor{dialinecolor}{rgb}{0.000000, 0.000000, 0.000000}
\pgfsetstrokecolor{dialinecolor}
\node at (39.000000\du,59.440700\du){$b_{2}$};
% setfont left to latex
\definecolor{dialinecolor}{rgb}{0.000000, 0.000000, 0.000000}
\pgfsetstrokecolor{dialinecolor}
\node at (37.000000\du,61.222500\du){$S_{21}$};
\pgfsetlinewidth{0.100000\du}
\pgfsetdash{}{0pt}
\pgfsetdash{}{0pt}
\pgfsetbuttcap
{
\definecolor{dialinecolor}{rgb}{0.000000, 0.000000, 0.000000}
\pgfsetfillcolor{dialinecolor}
% was here!!!
\pgfsetarrowsend{latex}
\definecolor{dialinecolor}{rgb}{0.000000, 0.000000, 0.000000}
\pgfsetstrokecolor{dialinecolor}
\draw (39.000000\du,60.000000\du)--(41.250000\du,60.000000\du);
}
\pgfsetlinewidth{0.100000\du}
\pgfsetdash{}{0pt}
\pgfsetdash{}{0pt}
\pgfsetbuttcap
{
\definecolor{dialinecolor}{rgb}{0.000000, 0.000000, 0.000000}
\pgfsetfillcolor{dialinecolor}
% was here!!!
\definecolor{dialinecolor}{rgb}{0.000000, 0.000000, 0.000000}
\pgfsetstrokecolor{dialinecolor}
\draw (41.000000\du,60.000000\du)--(43.250000\du,60.000000\du);
}
\definecolor{dialinecolor}{rgb}{0.000000, 0.000000, 0.000000}
\pgfsetstrokecolor{dialinecolor}
\draw (41.000000\du,60.000000\du)--(43.250000\du,60.000000\du);
\pgfsetlinewidth{0.100000\du}
\pgfsetdash{}{0pt}
\pgfsetmiterjoin
\pgfsetbuttcap
\definecolor{dialinecolor}{rgb}{0.000000, 0.000000, 0.000000}
\pgfsetfillcolor{dialinecolor}
\pgfpathmoveto{\pgfpoint{43.250000\du}{60.000000\du}}
\pgfpathcurveto{\pgfpoint{43.250000\du}{60.125000\du}}{\pgfpoint{43.125000\du}{60.250000\du}}{\pgfpoint{43.000000\du}{60.250000\du}}
\pgfpathcurveto{\pgfpoint{42.875000\du}{60.250000\du}}{\pgfpoint{42.750000\du}{60.125000\du}}{\pgfpoint{42.750000\du}{60.000000\du}}
\pgfpathcurveto{\pgfpoint{42.750000\du}{59.875000\du}}{\pgfpoint{42.875000\du}{59.750000\du}}{\pgfpoint{43.000000\du}{59.750000\du}}
\pgfpathcurveto{\pgfpoint{43.125000\du}{59.750000\du}}{\pgfpoint{43.250000\du}{59.875000\du}}{\pgfpoint{43.250000\du}{60.000000\du}}
\pgfusepath{fill}
\definecolor{dialinecolor}{rgb}{0.000000, 0.000000, 0.000000}
\pgfsetstrokecolor{dialinecolor}
\pgfpathmoveto{\pgfpoint{43.250000\du}{60.000000\du}}
\pgfpathcurveto{\pgfpoint{43.250000\du}{60.125000\du}}{\pgfpoint{43.125000\du}{60.250000\du}}{\pgfpoint{43.000000\du}{60.250000\du}}
\pgfpathcurveto{\pgfpoint{42.875000\du}{60.250000\du}}{\pgfpoint{42.750000\du}{60.125000\du}}{\pgfpoint{42.750000\du}{60.000000\du}}
\pgfpathcurveto{\pgfpoint{42.750000\du}{59.875000\du}}{\pgfpoint{42.875000\du}{59.750000\du}}{\pgfpoint{43.000000\du}{59.750000\du}}
\pgfpathcurveto{\pgfpoint{43.125000\du}{59.750000\du}}{\pgfpoint{43.250000\du}{59.875000\du}}{\pgfpoint{43.250000\du}{60.000000\du}}
\pgfusepath{stroke}
\pgfsetlinewidth{0.100000\du}
\pgfsetdash{}{0pt}
\pgfsetdash{}{0pt}
\pgfsetbuttcap
{
\definecolor{dialinecolor}{rgb}{0.000000, 0.000000, 0.000000}
\pgfsetfillcolor{dialinecolor}
% was here!!!
\pgfsetarrowsend{latex}
\definecolor{dialinecolor}{rgb}{0.000000, 0.000000, 0.000000}
\pgfsetstrokecolor{dialinecolor}
\draw (43.000000\du,64.000000\du)--(43.000000\du,61.750000\du);
}
\pgfsetlinewidth{0.100000\du}
\pgfsetdash{}{0pt}
\pgfsetdash{}{0pt}
\pgfsetbuttcap
{
\definecolor{dialinecolor}{rgb}{0.000000, 0.000000, 0.000000}
\pgfsetfillcolor{dialinecolor}
% was here!!!
\definecolor{dialinecolor}{rgb}{0.000000, 0.000000, 0.000000}
\pgfsetstrokecolor{dialinecolor}
\draw (43.000000\du,60.000000\du)--(43.000000\du,62.000000\du);
}
\pgfsetlinewidth{0.100000\du}
\pgfsetdash{}{0pt}
\pgfsetdash{}{0pt}
\pgfsetbuttcap
{
\definecolor{dialinecolor}{rgb}{0.000000, 0.000000, 0.000000}
\pgfsetfillcolor{dialinecolor}
% was here!!!
\pgfsetarrowsend{latex}
\definecolor{dialinecolor}{rgb}{0.000000, 0.000000, 0.000000}
\pgfsetstrokecolor{dialinecolor}
\draw (43.250000\du,64.000000\du)--(40.750000\du,64.000000\du);
}
\definecolor{dialinecolor}{rgb}{0.000000, 0.000000, 0.000000}
\pgfsetstrokecolor{dialinecolor}
\draw (43.250000\du,64.000000\du)--(40.750000\du,64.000000\du);
\pgfsetlinewidth{0.100000\du}
\pgfsetdash{}{0pt}
\pgfsetmiterjoin
\pgfsetbuttcap
\definecolor{dialinecolor}{rgb}{0.000000, 0.000000, 0.000000}
\pgfsetfillcolor{dialinecolor}
\pgfpathmoveto{\pgfpoint{43.250000\du}{64.000000\du}}
\pgfpathcurveto{\pgfpoint{43.250000\du}{64.125000\du}}{\pgfpoint{43.125000\du}{64.250000\du}}{\pgfpoint{43.000000\du}{64.250000\du}}
\pgfpathcurveto{\pgfpoint{42.875000\du}{64.250000\du}}{\pgfpoint{42.750000\du}{64.125000\du}}{\pgfpoint{42.750000\du}{64.000000\du}}
\pgfpathcurveto{\pgfpoint{42.750000\du}{63.875000\du}}{\pgfpoint{42.875000\du}{63.750000\du}}{\pgfpoint{43.000000\du}{63.750000\du}}
\pgfpathcurveto{\pgfpoint{43.125000\du}{63.750000\du}}{\pgfpoint{43.250000\du}{63.875000\du}}{\pgfpoint{43.250000\du}{64.000000\du}}
\pgfusepath{fill}
\definecolor{dialinecolor}{rgb}{0.000000, 0.000000, 0.000000}
\pgfsetstrokecolor{dialinecolor}
\pgfpathmoveto{\pgfpoint{43.250000\du}{64.000000\du}}
\pgfpathcurveto{\pgfpoint{43.250000\du}{64.125000\du}}{\pgfpoint{43.125000\du}{64.250000\du}}{\pgfpoint{43.000000\du}{64.250000\du}}
\pgfpathcurveto{\pgfpoint{42.875000\du}{64.250000\du}}{\pgfpoint{42.750000\du}{64.125000\du}}{\pgfpoint{42.750000\du}{64.000000\du}}
\pgfpathcurveto{\pgfpoint{42.750000\du}{63.875000\du}}{\pgfpoint{42.875000\du}{63.750000\du}}{\pgfpoint{43.000000\du}{63.750000\du}}
\pgfpathcurveto{\pgfpoint{43.125000\du}{63.750000\du}}{\pgfpoint{43.250000\du}{63.875000\du}}{\pgfpoint{43.250000\du}{64.000000\du}}
\pgfusepath{stroke}
\pgfsetlinewidth{0.100000\du}
\pgfsetdash{}{0pt}
\pgfsetdash{}{0pt}
\pgfsetbuttcap
{
\definecolor{dialinecolor}{rgb}{0.000000, 0.000000, 0.000000}
\pgfsetfillcolor{dialinecolor}
% was here!!!
\definecolor{dialinecolor}{rgb}{0.000000, 0.000000, 0.000000}
\pgfsetstrokecolor{dialinecolor}
\draw (39.000000\du,64.000000\du)--(41.000000\du,64.000000\du);
}
\pgfsetlinewidth{0.100000\du}
\pgfsetdash{}{0pt}
\pgfsetdash{}{0pt}
\pgfsetbuttcap
{
\definecolor{dialinecolor}{rgb}{0.000000, 0.000000, 0.000000}
\pgfsetfillcolor{dialinecolor}
% was here!!!
\pgfsetarrowsend{latex}
\definecolor{dialinecolor}{rgb}{0.000000, 0.000000, 0.000000}
\pgfsetstrokecolor{dialinecolor}
\draw (43.000000\du,60.000000\du)--(45.250000\du,60.000000\du);
}
\pgfsetlinewidth{0.100000\du}
\pgfsetdash{}{0pt}
\pgfsetdash{}{0pt}
\pgfsetbuttcap
{
\definecolor{dialinecolor}{rgb}{0.000000, 0.000000, 0.000000}
\pgfsetfillcolor{dialinecolor}
% was here!!!
\definecolor{dialinecolor}{rgb}{0.000000, 0.000000, 0.000000}
\pgfsetstrokecolor{dialinecolor}
\draw (45.000000\du,60.000000\du)--(47.250000\du,60.000000\du);
}
\definecolor{dialinecolor}{rgb}{0.000000, 0.000000, 0.000000}
\pgfsetstrokecolor{dialinecolor}
\draw (45.000000\du,60.000000\du)--(47.250000\du,60.000000\du);
\pgfsetlinewidth{0.100000\du}
\pgfsetdash{}{0pt}
\pgfsetmiterjoin
\pgfsetbuttcap
\definecolor{dialinecolor}{rgb}{0.000000, 0.000000, 0.000000}
\pgfsetfillcolor{dialinecolor}
\pgfpathmoveto{\pgfpoint{47.250000\du}{60.000000\du}}
\pgfpathcurveto{\pgfpoint{47.250000\du}{60.125000\du}}{\pgfpoint{47.125000\du}{60.250000\du}}{\pgfpoint{47.000000\du}{60.250000\du}}
\pgfpathcurveto{\pgfpoint{46.875000\du}{60.250000\du}}{\pgfpoint{46.750000\du}{60.125000\du}}{\pgfpoint{46.750000\du}{60.000000\du}}
\pgfpathcurveto{\pgfpoint{46.750000\du}{59.875000\du}}{\pgfpoint{46.875000\du}{59.750000\du}}{\pgfpoint{47.000000\du}{59.750000\du}}
\pgfpathcurveto{\pgfpoint{47.125000\du}{59.750000\du}}{\pgfpoint{47.250000\du}{59.875000\du}}{\pgfpoint{47.250000\du}{60.000000\du}}
\pgfusepath{fill}
\definecolor{dialinecolor}{rgb}{0.000000, 0.000000, 0.000000}
\pgfsetstrokecolor{dialinecolor}
\pgfpathmoveto{\pgfpoint{47.250000\du}{60.000000\du}}
\pgfpathcurveto{\pgfpoint{47.250000\du}{60.125000\du}}{\pgfpoint{47.125000\du}{60.250000\du}}{\pgfpoint{47.000000\du}{60.250000\du}}
\pgfpathcurveto{\pgfpoint{46.875000\du}{60.250000\du}}{\pgfpoint{46.750000\du}{60.125000\du}}{\pgfpoint{46.750000\du}{60.000000\du}}
\pgfpathcurveto{\pgfpoint{46.750000\du}{59.875000\du}}{\pgfpoint{46.875000\du}{59.750000\du}}{\pgfpoint{47.000000\du}{59.750000\du}}
\pgfpathcurveto{\pgfpoint{47.125000\du}{59.750000\du}}{\pgfpoint{47.250000\du}{59.875000\du}}{\pgfpoint{47.250000\du}{60.000000\du}}
\pgfusepath{stroke}
\pgfsetlinewidth{0.100000\du}
\pgfsetdash{}{0pt}
\pgfsetdash{}{0pt}
\pgfsetbuttcap
{
\definecolor{dialinecolor}{rgb}{0.000000, 0.000000, 0.000000}
\pgfsetfillcolor{dialinecolor}
% was here!!!
\pgfsetarrowsend{latex}
\definecolor{dialinecolor}{rgb}{0.000000, 0.000000, 0.000000}
\pgfsetstrokecolor{dialinecolor}
\draw (47.000000\du,64.000000\du)--(47.000000\du,61.750000\du);
}
\pgfsetlinewidth{0.100000\du}
\pgfsetdash{}{0pt}
\pgfsetdash{}{0pt}
\pgfsetbuttcap
{
\definecolor{dialinecolor}{rgb}{0.000000, 0.000000, 0.000000}
\pgfsetfillcolor{dialinecolor}
% was here!!!
\definecolor{dialinecolor}{rgb}{0.000000, 0.000000, 0.000000}
\pgfsetstrokecolor{dialinecolor}
\draw (47.000000\du,60.000000\du)--(47.000000\du,62.000000\du);
}
\pgfsetlinewidth{0.100000\du}
\pgfsetdash{}{0pt}
\pgfsetdash{}{0pt}
\pgfsetbuttcap
{
\definecolor{dialinecolor}{rgb}{0.000000, 0.000000, 0.000000}
\pgfsetfillcolor{dialinecolor}
% was here!!!
\pgfsetarrowsend{latex}
\definecolor{dialinecolor}{rgb}{0.000000, 0.000000, 0.000000}
\pgfsetstrokecolor{dialinecolor}
\draw (47.250000\du,64.000000\du)--(44.750000\du,64.000000\du);
}
\definecolor{dialinecolor}{rgb}{0.000000, 0.000000, 0.000000}
\pgfsetstrokecolor{dialinecolor}
\draw (47.250000\du,64.000000\du)--(44.750000\du,64.000000\du);
\pgfsetlinewidth{0.100000\du}
\pgfsetdash{}{0pt}
\pgfsetmiterjoin
\pgfsetbuttcap
\definecolor{dialinecolor}{rgb}{0.000000, 0.000000, 0.000000}
\pgfsetfillcolor{dialinecolor}
\pgfpathmoveto{\pgfpoint{47.250000\du}{64.000000\du}}
\pgfpathcurveto{\pgfpoint{47.250000\du}{64.125000\du}}{\pgfpoint{47.125000\du}{64.250000\du}}{\pgfpoint{47.000000\du}{64.250000\du}}
\pgfpathcurveto{\pgfpoint{46.875000\du}{64.250000\du}}{\pgfpoint{46.750000\du}{64.125000\du}}{\pgfpoint{46.750000\du}{64.000000\du}}
\pgfpathcurveto{\pgfpoint{46.750000\du}{63.875000\du}}{\pgfpoint{46.875000\du}{63.750000\du}}{\pgfpoint{47.000000\du}{63.750000\du}}
\pgfpathcurveto{\pgfpoint{47.125000\du}{63.750000\du}}{\pgfpoint{47.250000\du}{63.875000\du}}{\pgfpoint{47.250000\du}{64.000000\du}}
\pgfusepath{fill}
\definecolor{dialinecolor}{rgb}{0.000000, 0.000000, 0.000000}
\pgfsetstrokecolor{dialinecolor}
\pgfpathmoveto{\pgfpoint{47.250000\du}{64.000000\du}}
\pgfpathcurveto{\pgfpoint{47.250000\du}{64.125000\du}}{\pgfpoint{47.125000\du}{64.250000\du}}{\pgfpoint{47.000000\du}{64.250000\du}}
\pgfpathcurveto{\pgfpoint{46.875000\du}{64.250000\du}}{\pgfpoint{46.750000\du}{64.125000\du}}{\pgfpoint{46.750000\du}{64.000000\du}}
\pgfpathcurveto{\pgfpoint{46.750000\du}{63.875000\du}}{\pgfpoint{46.875000\du}{63.750000\du}}{\pgfpoint{47.000000\du}{63.750000\du}}
\pgfpathcurveto{\pgfpoint{47.125000\du}{63.750000\du}}{\pgfpoint{47.250000\du}{63.875000\du}}{\pgfpoint{47.250000\du}{64.000000\du}}
\pgfusepath{stroke}
\pgfsetlinewidth{0.100000\du}
\pgfsetdash{}{0pt}
\pgfsetdash{}{0pt}
\pgfsetbuttcap
{
\definecolor{dialinecolor}{rgb}{0.000000, 0.000000, 0.000000}
\pgfsetfillcolor{dialinecolor}
% was here!!!
\definecolor{dialinecolor}{rgb}{0.000000, 0.000000, 0.000000}
\pgfsetstrokecolor{dialinecolor}
\draw (43.000000\du,64.000000\du)--(45.000000\du,64.000000\du);
}
% setfont left to latex
\definecolor{dialinecolor}{rgb}{0.000000, 0.000000, 0.000000}
\pgfsetstrokecolor{dialinecolor}
\node at (43.000000\du,59.222500\du){Port 2};
% setfont left to latex
\definecolor{dialinecolor}{rgb}{0.000000, 0.000000, 0.000000}
\pgfsetstrokecolor{dialinecolor}
\node at (45.000000\du,60.949900\du){$e_{32}$};
% setfont left to latex
\definecolor{dialinecolor}{rgb}{0.000000, 0.000000, 0.000000}
\pgfsetstrokecolor{dialinecolor}
\node at (45.000000\du,63.495100\du){$e_{23}$};
% setfont left to latex
\definecolor{dialinecolor}{rgb}{0.000000, 0.000000, 0.000000}
\pgfsetstrokecolor{dialinecolor}
\node at (43.809200\du,62.222500\du){$e_{22}$};
% setfont left to latex
\definecolor{dialinecolor}{rgb}{0.000000, 0.000000, 0.000000}
\pgfsetstrokecolor{dialinecolor}
\node at (46.190800\du,62.222500\du){$e_{33}$};
% setfont left to latex
\definecolor{dialinecolor}{rgb}{0.000000, 0.000000, 0.000000}
\pgfsetstrokecolor{dialinecolor}
\node at (37.000000\du,63.222500\du){$S_{12}$};
% setfont left to latex
\definecolor{dialinecolor}{rgb}{0.000000, 0.000000, 0.000000}
\pgfsetstrokecolor{dialinecolor}
\node at (38.000000\du,62.222500\du){$S_{22}$};
\pgfsetlinewidth{0.100000\du}
\pgfsetdash{}{0pt}
\pgfsetdash{}{0pt}
\pgfsetbuttcap
{
\definecolor{dialinecolor}{rgb}{0.000000, 0.000000, 0.000000}
\pgfsetfillcolor{dialinecolor}
% was here!!!
\pgfsetarrowsend{latex}
\definecolor{dialinecolor}{rgb}{0.000000, 0.000000, 0.000000}
\pgfsetstrokecolor{dialinecolor}
\draw (51.250000\du,64.000000\du)--(48.750000\du,64.000000\du);
}
\definecolor{dialinecolor}{rgb}{0.000000, 0.000000, 0.000000}
\pgfsetstrokecolor{dialinecolor}
\draw (50.700000\du,64.000000\du)--(48.750000\du,64.000000\du);
\pgfsetlinewidth{0.100000\du}
\pgfsetdash{}{0pt}
\pgfsetmiterjoin
\pgfsetbuttcap
\definecolor{dialinecolor}{rgb}{1.000000, 1.000000, 1.000000}
\pgfsetfillcolor{dialinecolor}
\pgfpathmoveto{\pgfpoint{51.200000\du}{64.000000\du}}
\pgfpathcurveto{\pgfpoint{51.200000\du}{64.125000\du}}{\pgfpoint{51.075000\du}{64.250000\du}}{\pgfpoint{50.950000\du}{64.250000\du}}
\pgfpathcurveto{\pgfpoint{50.825000\du}{64.250000\du}}{\pgfpoint{50.700000\du}{64.125000\du}}{\pgfpoint{50.700000\du}{64.000000\du}}
\pgfpathcurveto{\pgfpoint{50.700000\du}{63.875000\du}}{\pgfpoint{50.825000\du}{63.750000\du}}{\pgfpoint{50.950000\du}{63.750000\du}}
\pgfpathcurveto{\pgfpoint{51.075000\du}{63.750000\du}}{\pgfpoint{51.200000\du}{63.875000\du}}{\pgfpoint{51.200000\du}{64.000000\du}}
\pgfusepath{fill}
\definecolor{dialinecolor}{rgb}{0.000000, 0.000000, 0.000000}
\pgfsetstrokecolor{dialinecolor}
\pgfpathmoveto{\pgfpoint{51.200000\du}{64.000000\du}}
\pgfpathcurveto{\pgfpoint{51.200000\du}{64.125000\du}}{\pgfpoint{51.075000\du}{64.250000\du}}{\pgfpoint{50.950000\du}{64.250000\du}}
\pgfpathcurveto{\pgfpoint{50.825000\du}{64.250000\du}}{\pgfpoint{50.700000\du}{64.125000\du}}{\pgfpoint{50.700000\du}{64.000000\du}}
\pgfpathcurveto{\pgfpoint{50.700000\du}{63.875000\du}}{\pgfpoint{50.825000\du}{63.750000\du}}{\pgfpoint{50.950000\du}{63.750000\du}}
\pgfpathcurveto{\pgfpoint{51.075000\du}{63.750000\du}}{\pgfpoint{51.200000\du}{63.875000\du}}{\pgfpoint{51.200000\du}{64.000000\du}}
\pgfusepath{stroke}
\pgfsetlinewidth{0.100000\du}
\pgfsetdash{}{0pt}
\pgfsetdash{}{0pt}
\pgfsetbuttcap
{
\definecolor{dialinecolor}{rgb}{0.000000, 0.000000, 0.000000}
\pgfsetfillcolor{dialinecolor}
% was here!!!
\definecolor{dialinecolor}{rgb}{0.000000, 0.000000, 0.000000}
\pgfsetstrokecolor{dialinecolor}
\draw (47.000000\du,64.000000\du)--(49.000000\du,64.000000\du);
}
\pgfsetlinewidth{0.100000\du}
\pgfsetdash{}{0pt}
\pgfsetdash{}{0pt}
\pgfsetbuttcap
{
\definecolor{dialinecolor}{rgb}{0.000000, 0.000000, 0.000000}
\pgfsetfillcolor{dialinecolor}
% was here!!!
\definecolor{dialinecolor}{rgb}{0.000000, 0.000000, 0.000000}
\pgfsetstrokecolor{dialinecolor}
\draw (51.250000\du,60.000000\du)--(49.000000\du,60.000000\du);
}
\definecolor{dialinecolor}{rgb}{0.000000, 0.000000, 0.000000}
\pgfsetstrokecolor{dialinecolor}
\draw (50.700000\du,60.000000\du)--(49.000000\du,60.000000\du);
\pgfsetlinewidth{0.100000\du}
\pgfsetdash{}{0pt}
\pgfsetmiterjoin
\pgfsetbuttcap
\definecolor{dialinecolor}{rgb}{1.000000, 1.000000, 1.000000}
\pgfsetfillcolor{dialinecolor}
\pgfpathmoveto{\pgfpoint{51.200000\du}{60.000000\du}}
\pgfpathcurveto{\pgfpoint{51.200000\du}{60.125000\du}}{\pgfpoint{51.075000\du}{60.250000\du}}{\pgfpoint{50.950000\du}{60.250000\du}}
\pgfpathcurveto{\pgfpoint{50.825000\du}{60.250000\du}}{\pgfpoint{50.700000\du}{60.125000\du}}{\pgfpoint{50.700000\du}{60.000000\du}}
\pgfpathcurveto{\pgfpoint{50.700000\du}{59.875000\du}}{\pgfpoint{50.825000\du}{59.750000\du}}{\pgfpoint{50.950000\du}{59.750000\du}}
\pgfpathcurveto{\pgfpoint{51.075000\du}{59.750000\du}}{\pgfpoint{51.200000\du}{59.875000\du}}{\pgfpoint{51.200000\du}{60.000000\du}}
\pgfusepath{fill}
\definecolor{dialinecolor}{rgb}{0.000000, 0.000000, 0.000000}
\pgfsetstrokecolor{dialinecolor}
\pgfpathmoveto{\pgfpoint{51.200000\du}{60.000000\du}}
\pgfpathcurveto{\pgfpoint{51.200000\du}{60.125000\du}}{\pgfpoint{51.075000\du}{60.250000\du}}{\pgfpoint{50.950000\du}{60.250000\du}}
\pgfpathcurveto{\pgfpoint{50.825000\du}{60.250000\du}}{\pgfpoint{50.700000\du}{60.125000\du}}{\pgfpoint{50.700000\du}{60.000000\du}}
\pgfpathcurveto{\pgfpoint{50.700000\du}{59.875000\du}}{\pgfpoint{50.825000\du}{59.750000\du}}{\pgfpoint{50.950000\du}{59.750000\du}}
\pgfpathcurveto{\pgfpoint{51.075000\du}{59.750000\du}}{\pgfpoint{51.200000\du}{59.875000\du}}{\pgfpoint{51.200000\du}{60.000000\du}}
\pgfusepath{stroke}
\pgfsetlinewidth{0.100000\du}
\pgfsetdash{}{0pt}
\pgfsetdash{}{0pt}
\pgfsetbuttcap
{
\definecolor{dialinecolor}{rgb}{0.000000, 0.000000, 0.000000}
\pgfsetfillcolor{dialinecolor}
% was here!!!
\pgfsetarrowsend{latex}
\definecolor{dialinecolor}{rgb}{0.000000, 0.000000, 0.000000}
\pgfsetstrokecolor{dialinecolor}
\draw (47.000000\du,60.000000\du)--(49.250000\du,60.000000\du);
}
% setfont left to latex
\definecolor{dialinecolor}{rgb}{0.000000, 0.000000, 0.000000}
\pgfsetstrokecolor{dialinecolor}
\node at (52.000000\du,60.213188\du){$b_{3}$};
% setfont left to latex
\definecolor{dialinecolor}{rgb}{0.000000, 0.000000, 0.000000}
\pgfsetstrokecolor{dialinecolor}
\node at (52.000000\du,64.213188\du){$a_{3}$};
\pgfsetlinewidth{0.100000\du}
\pgfsetdash{{\pgflinewidth}{0.200000\du}}{0cm}
\pgfsetdash{{\pgflinewidth}{0.200000\du}}{0cm}
\pgfsetbuttcap
{
\definecolor{dialinecolor}{rgb}{0.000000, 0.000000, 0.000000}
\pgfsetfillcolor{dialinecolor}
% was here!!!
\definecolor{dialinecolor}{rgb}{0.000000, 0.000000, 0.000000}
\pgfsetstrokecolor{dialinecolor}
\draw (27.000000\du,60.000000\du)--(27.000000\du,56.000000\du);
}
\pgfsetlinewidth{0.100000\du}
\pgfsetdash{{\pgflinewidth}{0.200000\du}}{0cm}
\pgfsetdash{{\pgflinewidth}{0.200000\du}}{0cm}
\pgfsetbuttcap
{
\definecolor{dialinecolor}{rgb}{0.000000, 0.000000, 0.000000}
\pgfsetfillcolor{dialinecolor}
% was here!!!
\pgfsetarrowsend{latex}
\definecolor{dialinecolor}{rgb}{0.000000, 0.000000, 0.000000}
\pgfsetstrokecolor{dialinecolor}
\draw (27.000000\du,56.000000\du)--(37.250000\du,56.000000\du);
}
\pgfsetlinewidth{0.100000\du}
\pgfsetdash{{\pgflinewidth}{0.200000\du}}{0cm}
\pgfsetdash{{\pgflinewidth}{0.200000\du}}{0cm}
\pgfsetbuttcap
{
\definecolor{dialinecolor}{rgb}{0.000000, 0.000000, 0.000000}
\pgfsetfillcolor{dialinecolor}
% was here!!!
\definecolor{dialinecolor}{rgb}{0.000000, 0.000000, 0.000000}
\pgfsetstrokecolor{dialinecolor}
\draw (37.000000\du,56.000000\du)--(47.000000\du,56.000000\du);
}
\pgfsetlinewidth{0.100000\du}
\pgfsetdash{{\pgflinewidth}{0.200000\du}}{0cm}
\pgfsetdash{{\pgflinewidth}{0.200000\du}}{0cm}
\pgfsetbuttcap
{
\definecolor{dialinecolor}{rgb}{0.000000, 0.000000, 0.000000}
\pgfsetfillcolor{dialinecolor}
% was here!!!
\definecolor{dialinecolor}{rgb}{0.000000, 0.000000, 0.000000}
\pgfsetstrokecolor{dialinecolor}
\draw (47.000000\du,56.000000\du)--(47.000000\du,60.000000\du);
}
\pgfsetlinewidth{0.100000\du}
\pgfsetdash{{\pgflinewidth}{0.200000\du}}{0cm}
\pgfsetdash{{\pgflinewidth}{0.200000\du}}{0cm}
\pgfsetbuttcap
{
\definecolor{dialinecolor}{rgb}{0.000000, 0.000000, 0.000000}
\pgfsetfillcolor{dialinecolor}
% was here!!!
\definecolor{dialinecolor}{rgb}{0.000000, 0.000000, 0.000000}
\pgfsetstrokecolor{dialinecolor}
\draw (47.000000\du,64.000000\du)--(47.000000\du,68.000000\du);
}
\pgfsetlinewidth{0.100000\du}
\pgfsetdash{{\pgflinewidth}{0.200000\du}}{0cm}
\pgfsetdash{{\pgflinewidth}{0.200000\du}}{0cm}
\pgfsetbuttcap
{
\definecolor{dialinecolor}{rgb}{0.000000, 0.000000, 0.000000}
\pgfsetfillcolor{dialinecolor}
% was here!!!
\pgfsetarrowsend{latex}
\definecolor{dialinecolor}{rgb}{0.000000, 0.000000, 0.000000}
\pgfsetstrokecolor{dialinecolor}
\draw (47.000000\du,68.000000\du)--(36.750000\du,68.000000\du);
}
\pgfsetlinewidth{0.100000\du}
\pgfsetdash{{\pgflinewidth}{0.200000\du}}{0cm}
\pgfsetdash{{\pgflinewidth}{0.200000\du}}{0cm}
\pgfsetbuttcap
{
\definecolor{dialinecolor}{rgb}{0.000000, 0.000000, 0.000000}
\pgfsetfillcolor{dialinecolor}
% was here!!!
\definecolor{dialinecolor}{rgb}{0.000000, 0.000000, 0.000000}
\pgfsetstrokecolor{dialinecolor}
\draw (37.000000\du,68.000000\du)--(27.000000\du,68.000000\du);
}
\pgfsetlinewidth{0.100000\du}
\pgfsetdash{{\pgflinewidth}{0.200000\du}}{0cm}
\pgfsetdash{{\pgflinewidth}{0.200000\du}}{0cm}
\pgfsetbuttcap
{
\definecolor{dialinecolor}{rgb}{0.000000, 0.000000, 0.000000}
\pgfsetfillcolor{dialinecolor}
% was here!!!
\definecolor{dialinecolor}{rgb}{0.000000, 0.000000, 0.000000}
\pgfsetstrokecolor{dialinecolor}
\draw (27.000000\du,68.000000\du)--(27.000000\du,64.000000\du);
}
% setfont left to latex
\definecolor{dialinecolor}{rgb}{0.000000, 0.000000, 0.000000}
\pgfsetstrokecolor{dialinecolor}
\node at (37.000000\du,55.222500\du){$e_{30}$};
% setfont left to latex
\definecolor{dialinecolor}{rgb}{0.000000, 0.000000, 0.000000}
\pgfsetstrokecolor{dialinecolor}
\node at (37.000000\du,69.222500\du){$e_{03}$};
\end{tikzpicture}

	\caption{Two port model}
	\label{fig:twoportmodel}
\end{figure}

Where the SOL calibration is using 3 error coefficients, the full two port calibration needs 12 for SOLTI calibration (Short - Open - Load - Through - Isolation). In order to obtain the 12 error terms, we need Short Open Load measurements for every port as performed in section \ref{sec:solcal}. Apart from the SOL measurements we also need a Through measurement (two cables connected) and an isolation measurement. An additional standard is also introduced, a non-perfect through standard (see section \ref{sec:calstds}). The 12 error coefficients needed for SOLTI calibration are obtained in section \ref{sec:obtainingsolti}.

\begin{equation}
D = \left[1+\frac{S_{11M}-e_{00}}{e_{10}e_{01}}e_{11}\right]
\left[1+\frac{S_{22M}-e_{33}}{e_{23}e_{32}}e_{22}\right]-
\frac{S_{21M}-e_{30}}{e_{10}e_{32}}
\frac{S_{12M}-e_{03}}{e_{23}e_{01}}
e'_{22}e'_{11}
\end{equation}
\begin{equation}
S_{11}= \frac{\frac{S_{11M}-e_{00}}{e_{10}e_{01}}\left[ 1+\frac{S_{22M}-e_{33}}{e_{23}e_{32}}e_{22}\right]-e'_{22}\frac{S_{21M}-e_{30}}{e_{10}e_{32}}\frac{S_{12M}-e_{03}}{e_{23}e_{01}}}{D}
\end{equation}
\begin{equation}
S_{21} = \frac{\frac{S_{21M}-e_{30}}{e_{10}e_{32}}\left[1+\frac{S_{22M}-e_{33}}{e_{23}e_{32}}(e_{22}-e'_{22})\right]}{D}
\end{equation}
\begin{equation}
S_{12} = \frac{\frac{S_{12M}-e_{03}}{e_{23}e_{01}}\left[1+\frac{S_{11M}-e_{00}}{e_{10}e_{01}}(e_{11}-e'_{11})\right]}{D}
\end{equation}
\begin{equation}
S_{22}= \frac{\frac{S_{22M}-e_{33}}{e_{23}e_{32}}\left[ 1+\frac{S_{11M}-e_{00}}{e_{10}e_{01}}e_{11}\right]-e'_{11}\frac{S_{12M}-e_{03}}{e_{23}e_{01}}\frac{S_{21M}-e_{30}}{e_{10}e_{32}}}{D}
\end{equation}
\newpage
\begin{itemize}
	\item $e_{00}$: Port 1 directivity
	\item $e_{11}$: Port 1 match
	\item $e_{10}e_{01}$: Port 1 reflection tracking
	\item $e_{10}e_{32}$: Forward transmission tracking
	\item $e'_{22}$: Port 2 match seen from port 1
	\item $e_{30}$: Forward transmission leakage
	\item $e_{33}$: Port 2 directivity
	\item $e_{22}$: Port 2 match
	\item $e_{23}e_{32}$: Port 2 reflection tracking
	\item $e_{23}e_{01}$: Reverse transmission tracking
	\item $e'_{11}$: Port 1 match seen from port 2
	\item $e_{03}$: Reverse transmission leakage
\end{itemize}



\subsubsection{obtaining the error coefficients for SOLTI Calibration}
\label{sec:obtainingsolti}
The Through - Isolation calibration depends on the SOL calibration, this calibration has to be done first. 
$e_{00}$, $e_{11}$ and $\Delta_{eP1}$ are obtained from the measurements in \ref{sec:obtainingerrorcoefsSOL}, $e_{33}$, $e_{22}$ and $\Delta_{eP2}$ use the same equations but with Short / Open / Load measurements taken from port 2.
\begin{itemize}
	\item {$e_{00}$ and $e_{33}$ are obtained from equation \ref{eqn:e00}}
	\item {$e_{11}$ and $e_{22}$ are obtained from equation \ref{eqn:e11}}	
	\item {$\Delta_{eP1}$ and $\Delta_{eP2}$ are obtained from equation \ref{eqn:deltae}}
\end{itemize}

\begin{equation}
(e_{10}e_{01}) = -\Delta_{eP1}+e_{11}e_{00}
\end{equation}
\begin{equation}
(e_{23}e_{32}) = -\Delta_{eP2}+e_{22}e_{33}
\end{equation}

For the next set of equations, we determine 5 sets of 2-port s-parameter matrices.
\begin{itemize}
	\item {$S_i$: The isolation measurement executed with the two measurement cables connected to $50\Omega$ loads. From this matrix only $S_{21i}$ and $S_{12i}$ are used.}
	\item {$S_t$: The through measurement, this s-parameter contains data of the measured through standard with the uncalibrated network analyzer.}
	\item {$S_s$: The through standard (see paragraph \ref{sec:calstds})}
	\item {$S_M$: The measured S-parameter matrix of the 2-port DUT}
	\item {$S$: This is the resulting calibrated S-parameter matrix containing $S_{11}$, $S_{21}$, $S_{12}$ and $S_{22}$}
	
\end{itemize}
The two port isolation measurement file is created with the measurement cables connected to a 50$\Omega$ load. From this file the following error terms are obtained:
\begin{itemize}
	\item $e_{30} = S_{21i}$, this is the forward isolation
	\item $e_{03} = S_{12i}$, this is the reverse isolation
\end{itemize}



$e'_{22}$ is in fact the same error term as $e_{22}$, but it is seen from port 1, for that reason we use a different calculation. Instead we use the SOL calibration method to de-embed the reflection measured at the through measurement $S_{11t}$, normalized to the standard $S_{21s}$. The same is true for $e'_{11}$. 

\begin{equation}
\label{eqn:e'22}
e'_{22} = \frac{\frac{S_{11t}}{S_{21s}}-e_{00}}{\frac{S_{11t}}{S_{21s}}e_{11}-\Delta_{eP1}}
\end{equation}
\begin{equation}
\label{eqn:e'11}
e'_{11} = \frac{\frac{S_{22t}}{S_{12s}}-e_{33}}{\frac{S_{22t}}{S_{12s}}e_{22}-\Delta_{eP2}}
\end{equation}

\begin{equation}
\label{eqn:e10e32}
e_{10}e_{32} = \left(\frac{S_{21t}}{S_{21s}} - e_{30}\right)\left(1-e_{11}e'_{22}\right)
\end{equation}
\begin{equation}
\label{eqn:e23e01}
e_{23}e_{01} = \left(\frac{S_{12t}}{S_{12s}} - e_{03}\right)\left(1-e_{22}e'_{11}\right)
\end{equation}

The 12-term calibration method from Agilent \cite{agilent_calibration} does not mention any through standard in its equations, Equations \ref{eqn:e'11} to \ref{eqn:e23e01} are slightly modified with respect to \cite{agilent_calibration}, the Through and Reflect measurements $S_{11t}$, $S_{21t}$, $S_{12t}$ and $S_{22t}$ are compensated with (divided by) the known through standard \ref{sec:throughstd}

\subsection{3 Port SOLTI Calibration}
\label{sec:3psoltical}

\begin{figure}[H]
	\centering
	% Graphic for TeX using PGF
% Title: /home/franss/DeEmbed/deembed/doc/figures/ThreePortModel.dia
% Creator: Dia v0.97.3
% CreationDate: Thu Feb 16 09:42:39 2017
% For: franss
% \usepackage{tikz}
% The following commands are not supported in PSTricks at present
% We define them conditionally, so when they are implemented,
% this pgf file will use them.
\ifx\du\undefined
  \newlength{\du}
\fi
\setlength{\du}{15\unitlength}
\begin{tikzpicture}
\pgftransformxscale{1.000000}
\pgftransformyscale{-1.000000}
\definecolor{dialinecolor}{rgb}{0.000000, 0.000000, 0.000000}
\pgfsetstrokecolor{dialinecolor}
\definecolor{dialinecolor}{rgb}{1.000000, 1.000000, 1.000000}
\pgfsetfillcolor{dialinecolor}
\pgfsetlinewidth{0.100000\du}
\pgfsetdash{}{0pt}
\pgfsetdash{}{0pt}
\pgfsetmiterjoin
\pgfsetbuttcap
\definecolor{dialinecolor}{rgb}{1.000000, 0.996078, 0.870588}
\pgfsetfillcolor{dialinecolor}
\fill (33.750000\du,66.684000\du)--(39.750000\du,66.684000\du)--(42.750000\du,71.880100\du)--(39.750000\du,77.076300\du)--(33.750000\du,77.076300\du)--(30.750000\du,71.880100\du)--cycle;
\definecolor{dialinecolor}{rgb}{0.000000, 0.000000, 0.000000}
\pgfsetstrokecolor{dialinecolor}
\draw (33.750000\du,66.684000\du)--(39.750000\du,66.684000\du)--(42.750000\du,71.880100\du)--(39.750000\du,77.076300\du)--(33.750000\du,77.076300\du)--(30.750000\du,71.880100\du)--cycle;
\pgfsetlinewidth{0.100000\du}
\pgfsetdash{}{0pt}
\pgfsetdash{}{0pt}
\pgfsetbuttcap
{
\definecolor{dialinecolor}{rgb}{0.000000, 0.000000, 0.000000}
\pgfsetfillcolor{dialinecolor}
% was here!!!
\pgfsetarrowsend{latex}
\definecolor{dialinecolor}{rgb}{0.000000, 0.000000, 0.000000}
\pgfsetstrokecolor{dialinecolor}
\draw (20.783492\du,60.063803\du)--(23.466508\du,61.612797\du);
}
\definecolor{dialinecolor}{rgb}{0.000000, 0.000000, 0.000000}
\pgfsetstrokecolor{dialinecolor}
\draw (21.259810\du,60.338797\du)--(23.466508\du,61.612797\du);
\pgfsetlinewidth{0.100000\du}
\pgfsetdash{}{0pt}
\pgfsetmiterjoin
\pgfsetbuttcap
\definecolor{dialinecolor}{rgb}{1.000000, 1.000000, 1.000000}
\pgfsetfillcolor{dialinecolor}
\pgfpathmoveto{\pgfpoint{20.826794\du}{60.088802\du}}
\pgfpathcurveto{\pgfpoint{20.889292\du}{59.980548\du}}{\pgfpoint{21.060045\du}{59.934793\du}}{\pgfpoint{21.168299\du}{59.997292\du}}
\pgfpathcurveto{\pgfpoint{21.276553\du}{60.059790\du}}{\pgfpoint{21.322308\du}{60.230543\du}}{\pgfpoint{21.259810\du}{60.338797\du}}
\pgfpathcurveto{\pgfpoint{21.197311\du}{60.447051\du}}{\pgfpoint{21.026558\du}{60.492806\du}}{\pgfpoint{20.918304\du}{60.430307\du}}
\pgfpathcurveto{\pgfpoint{20.810050\du}{60.367809\du}}{\pgfpoint{20.764295\du}{60.197056\du}}{\pgfpoint{20.826794\du}{60.088802\du}}
\pgfusepath{fill}
\definecolor{dialinecolor}{rgb}{0.000000, 0.000000, 0.000000}
\pgfsetstrokecolor{dialinecolor}
\pgfpathmoveto{\pgfpoint{20.826794\du}{60.088802\du}}
\pgfpathcurveto{\pgfpoint{20.889292\du}{59.980548\du}}{\pgfpoint{21.060045\du}{59.934793\du}}{\pgfpoint{21.168299\du}{59.997292\du}}
\pgfpathcurveto{\pgfpoint{21.276553\du}{60.059790\du}}{\pgfpoint{21.322308\du}{60.230543\du}}{\pgfpoint{21.259810\du}{60.338797\du}}
\pgfpathcurveto{\pgfpoint{21.197311\du}{60.447051\du}}{\pgfpoint{21.026558\du}{60.492806\du}}{\pgfpoint{20.918304\du}{60.430307\du}}
\pgfpathcurveto{\pgfpoint{20.810050\du}{60.367809\du}}{\pgfpoint{20.764295\du}{60.197056\du}}{\pgfpoint{20.826794\du}{60.088802\du}}
\pgfusepath{stroke}
\pgfsetlinewidth{0.100000\du}
\pgfsetdash{}{0pt}
\pgfsetdash{}{0pt}
\pgfsetbuttcap
{
\definecolor{dialinecolor}{rgb}{0.000000, 0.000000, 0.000000}
\pgfsetfillcolor{dialinecolor}
% was here!!!
\definecolor{dialinecolor}{rgb}{0.000000, 0.000000, 0.000000}
\pgfsetstrokecolor{dialinecolor}
\draw (23.250000\du,61.487800\du)--(25.716508\du,62.911797\du);
}
\definecolor{dialinecolor}{rgb}{0.000000, 0.000000, 0.000000}
\pgfsetstrokecolor{dialinecolor}
\draw (23.250000\du,61.487800\du)--(25.716508\du,62.911797\du);
\pgfsetlinewidth{0.100000\du}
\pgfsetdash{}{0pt}
\pgfsetmiterjoin
\pgfsetbuttcap
\definecolor{dialinecolor}{rgb}{0.000000, 0.000000, 0.000000}
\pgfsetfillcolor{dialinecolor}
\pgfpathmoveto{\pgfpoint{25.716508\du}{62.911797\du}}
\pgfpathcurveto{\pgfpoint{25.654009\du}{63.020051\du}}{\pgfpoint{25.483257\du}{63.065807\du}}{\pgfpoint{25.375003\du}{63.003308\du}}
\pgfpathcurveto{\pgfpoint{25.266749\du}{62.940809\du}}{\pgfpoint{25.220993\du}{62.770057\du}}{\pgfpoint{25.283492\du}{62.661803\du}}
\pgfpathcurveto{\pgfpoint{25.345991\du}{62.553549\du}}{\pgfpoint{25.516743\du}{62.507793\du}}{\pgfpoint{25.624997\du}{62.570292\du}}
\pgfpathcurveto{\pgfpoint{25.733251\du}{62.632791\du}}{\pgfpoint{25.779007\du}{62.803543\du}}{\pgfpoint{25.716508\du}{62.911797\du}}
\pgfusepath{fill}
\definecolor{dialinecolor}{rgb}{0.000000, 0.000000, 0.000000}
\pgfsetstrokecolor{dialinecolor}
\pgfpathmoveto{\pgfpoint{25.716508\du}{62.911797\du}}
\pgfpathcurveto{\pgfpoint{25.654009\du}{63.020051\du}}{\pgfpoint{25.483257\du}{63.065807\du}}{\pgfpoint{25.375003\du}{63.003308\du}}
\pgfpathcurveto{\pgfpoint{25.266749\du}{62.940809\du}}{\pgfpoint{25.220993\du}{62.770057\du}}{\pgfpoint{25.283492\du}{62.661803\du}}
\pgfpathcurveto{\pgfpoint{25.345991\du}{62.553549\du}}{\pgfpoint{25.516743\du}{62.507793\du}}{\pgfpoint{25.624997\du}{62.570292\du}}
\pgfpathcurveto{\pgfpoint{25.733251\du}{62.632791\du}}{\pgfpoint{25.779007\du}{62.803543\du}}{\pgfpoint{25.716508\du}{62.911797\du}}
\pgfusepath{stroke}
\pgfsetlinewidth{0.100000\du}
\pgfsetdash{}{0pt}
\pgfsetdash{}{0pt}
\pgfsetbuttcap
{
\definecolor{dialinecolor}{rgb}{0.000000, 0.000000, 0.000000}
\pgfsetfillcolor{dialinecolor}
% was here!!!
\definecolor{dialinecolor}{rgb}{0.000000, 0.000000, 0.000000}
\pgfsetstrokecolor{dialinecolor}
\draw (18.533492\du,63.960903\du)--(21.000000\du,65.384900\du);
}
\definecolor{dialinecolor}{rgb}{0.000000, 0.000000, 0.000000}
\pgfsetstrokecolor{dialinecolor}
\draw (19.009810\du,64.235897\du)--(21.000000\du,65.384900\du);
\pgfsetlinewidth{0.100000\du}
\pgfsetdash{}{0pt}
\pgfsetmiterjoin
\pgfsetbuttcap
\definecolor{dialinecolor}{rgb}{1.000000, 1.000000, 1.000000}
\pgfsetfillcolor{dialinecolor}
\pgfpathmoveto{\pgfpoint{18.576794\du}{63.985902\du}}
\pgfpathcurveto{\pgfpoint{18.639292\du}{63.877648\du}}{\pgfpoint{18.810045\du}{63.831893\du}}{\pgfpoint{18.918299\du}{63.894392\du}}
\pgfpathcurveto{\pgfpoint{19.026553\du}{63.956890\du}}{\pgfpoint{19.072308\du}{64.127643\du}}{\pgfpoint{19.009810\du}{64.235897\du}}
\pgfpathcurveto{\pgfpoint{18.947311\du}{64.344151\du}}{\pgfpoint{18.776558\du}{64.389906\du}}{\pgfpoint{18.668304\du}{64.327407\du}}
\pgfpathcurveto{\pgfpoint{18.560050\du}{64.264909\du}}{\pgfpoint{18.514295\du}{64.094156\du}}{\pgfpoint{18.576794\du}{63.985902\du}}
\pgfusepath{fill}
\definecolor{dialinecolor}{rgb}{0.000000, 0.000000, 0.000000}
\pgfsetstrokecolor{dialinecolor}
\pgfpathmoveto{\pgfpoint{18.576794\du}{63.985902\du}}
\pgfpathcurveto{\pgfpoint{18.639292\du}{63.877648\du}}{\pgfpoint{18.810045\du}{63.831893\du}}{\pgfpoint{18.918299\du}{63.894392\du}}
\pgfpathcurveto{\pgfpoint{19.026553\du}{63.956890\du}}{\pgfpoint{19.072308\du}{64.127643\du}}{\pgfpoint{19.009810\du}{64.235897\du}}
\pgfpathcurveto{\pgfpoint{18.947311\du}{64.344151\du}}{\pgfpoint{18.776558\du}{64.389906\du}}{\pgfpoint{18.668304\du}{64.327407\du}}
\pgfpathcurveto{\pgfpoint{18.560050\du}{64.264909\du}}{\pgfpoint{18.514295\du}{64.094156\du}}{\pgfpoint{18.576794\du}{63.985902\du}}
\pgfusepath{stroke}
\pgfsetlinewidth{0.100000\du}
\pgfsetdash{}{0pt}
\pgfsetdash{}{0pt}
\pgfsetbuttcap
{
\definecolor{dialinecolor}{rgb}{0.000000, 0.000000, 0.000000}
\pgfsetfillcolor{dialinecolor}
% was here!!!
\pgfsetarrowsstart{latex}
\definecolor{dialinecolor}{rgb}{0.000000, 0.000000, 0.000000}
\pgfsetstrokecolor{dialinecolor}
\draw (20.783496\du,65.259896\du)--(23.466504\du,66.809004\du);
}
\definecolor{dialinecolor}{rgb}{0.000000, 0.000000, 0.000000}
\pgfsetstrokecolor{dialinecolor}
\draw (20.783496\du,65.259896\du)--(23.466504\du,66.809004\du);
\pgfsetlinewidth{0.100000\du}
\pgfsetdash{}{0pt}
\pgfsetmiterjoin
\pgfsetbuttcap
\definecolor{dialinecolor}{rgb}{0.000000, 0.000000, 0.000000}
\pgfsetfillcolor{dialinecolor}
\pgfpathmoveto{\pgfpoint{23.466504\du}{66.809004\du}}
\pgfpathcurveto{\pgfpoint{23.404002\du}{66.917256\du}}{\pgfpoint{23.233247\du}{66.963006\du}}{\pgfpoint{23.124996\du}{66.900504\du}}
\pgfpathcurveto{\pgfpoint{23.016744\du}{66.838002\du}}{\pgfpoint{22.970994\du}{66.667247\du}}{\pgfpoint{23.033496\du}{66.558996\du}}
\pgfpathcurveto{\pgfpoint{23.095998\du}{66.450744\du}}{\pgfpoint{23.266753\du}{66.404994\du}}{\pgfpoint{23.375004\du}{66.467496\du}}
\pgfpathcurveto{\pgfpoint{23.483256\du}{66.529998\du}}{\pgfpoint{23.529006\du}{66.700753\du}}{\pgfpoint{23.466504\du}{66.809004\du}}
\pgfusepath{fill}
\definecolor{dialinecolor}{rgb}{0.000000, 0.000000, 0.000000}
\pgfsetstrokecolor{dialinecolor}
\pgfpathmoveto{\pgfpoint{23.466504\du}{66.809004\du}}
\pgfpathcurveto{\pgfpoint{23.404002\du}{66.917256\du}}{\pgfpoint{23.233247\du}{66.963006\du}}{\pgfpoint{23.124996\du}{66.900504\du}}
\pgfpathcurveto{\pgfpoint{23.016744\du}{66.838002\du}}{\pgfpoint{22.970994\du}{66.667247\du}}{\pgfpoint{23.033496\du}{66.558996\du}}
\pgfpathcurveto{\pgfpoint{23.095998\du}{66.450744\du}}{\pgfpoint{23.266753\du}{66.404994\du}}{\pgfpoint{23.375004\du}{66.467496\du}}
\pgfpathcurveto{\pgfpoint{23.483256\du}{66.529998\du}}{\pgfpoint{23.529006\du}{66.700753\du}}{\pgfpoint{23.466504\du}{66.809004\du}}
\pgfusepath{stroke}
\pgfsetlinewidth{0.100000\du}
\pgfsetdash{}{0pt}
\pgfsetdash{}{0pt}
\pgfsetbuttcap
{
\definecolor{dialinecolor}{rgb}{0.000000, 0.000000, 0.000000}
\pgfsetfillcolor{dialinecolor}
% was here!!!
\pgfsetarrowsend{latex}
\definecolor{dialinecolor}{rgb}{0.000000, 0.000000, 0.000000}
\pgfsetstrokecolor{dialinecolor}
\draw (25.500000\du,62.786800\du)--(24.125002\du,65.168407\du);
}
\pgfsetlinewidth{0.100000\du}
\pgfsetdash{}{0pt}
\pgfsetdash{}{0pt}
\pgfsetbuttcap
{
\definecolor{dialinecolor}{rgb}{0.000000, 0.000000, 0.000000}
\pgfsetfillcolor{dialinecolor}
% was here!!!
\definecolor{dialinecolor}{rgb}{0.000000, 0.000000, 0.000000}
\pgfsetstrokecolor{dialinecolor}
\draw (24.250000\du,64.951900\du)--(23.250000\du,66.684000\du);
}
\pgfsetlinewidth{0.100000\du}
\pgfsetdash{}{0pt}
\pgfsetdash{}{0pt}
\pgfsetbuttcap
{
\definecolor{dialinecolor}{rgb}{0.000000, 0.000000, 0.000000}
\pgfsetfillcolor{dialinecolor}
% was here!!!
\pgfsetarrowsend{latex}
\definecolor{dialinecolor}{rgb}{0.000000, 0.000000, 0.000000}
\pgfsetstrokecolor{dialinecolor}
\draw (25.500000\du,62.786800\du)--(27.966504\du,64.210904\du);
}
\pgfsetlinewidth{0.100000\du}
\pgfsetdash{}{0pt}
\pgfsetdash{}{0pt}
\pgfsetbuttcap
{
\definecolor{dialinecolor}{rgb}{0.000000, 0.000000, 0.000000}
\pgfsetfillcolor{dialinecolor}
% was here!!!
\definecolor{dialinecolor}{rgb}{0.000000, 0.000000, 0.000000}
\pgfsetstrokecolor{dialinecolor}
\draw (27.750000\du,64.085900\du)--(30.216508\du,65.509897\du);
}
\definecolor{dialinecolor}{rgb}{0.000000, 0.000000, 0.000000}
\pgfsetstrokecolor{dialinecolor}
\draw (27.750000\du,64.085900\du)--(30.216508\du,65.509897\du);
\pgfsetlinewidth{0.100000\du}
\pgfsetdash{}{0pt}
\pgfsetmiterjoin
\pgfsetbuttcap
\definecolor{dialinecolor}{rgb}{0.000000, 0.000000, 0.000000}
\pgfsetfillcolor{dialinecolor}
\pgfpathmoveto{\pgfpoint{30.216508\du}{65.509897\du}}
\pgfpathcurveto{\pgfpoint{30.154009\du}{65.618151\du}}{\pgfpoint{29.983257\du}{65.663907\du}}{\pgfpoint{29.875003\du}{65.601408\du}}
\pgfpathcurveto{\pgfpoint{29.766749\du}{65.538909\du}}{\pgfpoint{29.720993\du}{65.368157\du}}{\pgfpoint{29.783492\du}{65.259903\du}}
\pgfpathcurveto{\pgfpoint{29.845991\du}{65.151649\du}}{\pgfpoint{30.016743\du}{65.105893\du}}{\pgfpoint{30.124997\du}{65.168392\du}}
\pgfpathcurveto{\pgfpoint{30.233251\du}{65.230891\du}}{\pgfpoint{30.279007\du}{65.401643\du}}{\pgfpoint{30.216508\du}{65.509897\du}}
\pgfusepath{fill}
\definecolor{dialinecolor}{rgb}{0.000000, 0.000000, 0.000000}
\pgfsetstrokecolor{dialinecolor}
\pgfpathmoveto{\pgfpoint{30.216508\du}{65.509897\du}}
\pgfpathcurveto{\pgfpoint{30.154009\du}{65.618151\du}}{\pgfpoint{29.983257\du}{65.663907\du}}{\pgfpoint{29.875003\du}{65.601408\du}}
\pgfpathcurveto{\pgfpoint{29.766749\du}{65.538909\du}}{\pgfpoint{29.720993\du}{65.368157\du}}{\pgfpoint{29.783492\du}{65.259903\du}}
\pgfpathcurveto{\pgfpoint{29.845991\du}{65.151649\du}}{\pgfpoint{30.016743\du}{65.105893\du}}{\pgfpoint{30.124997\du}{65.168392\du}}
\pgfpathcurveto{\pgfpoint{30.233251\du}{65.230891\du}}{\pgfpoint{30.279007\du}{65.401643\du}}{\pgfpoint{30.216508\du}{65.509897\du}}
\pgfusepath{stroke}
\pgfsetlinewidth{0.100000\du}
\pgfsetdash{}{0pt}
\pgfsetdash{}{0pt}
\pgfsetbuttcap
{
\definecolor{dialinecolor}{rgb}{0.000000, 0.000000, 0.000000}
\pgfsetfillcolor{dialinecolor}
% was here!!!
\pgfsetarrowsend{latex}
\definecolor{dialinecolor}{rgb}{0.000000, 0.000000, 0.000000}
\pgfsetstrokecolor{dialinecolor}
\draw (30.000000\du,65.384900\du)--(32.466504\du,66.809004\du);
}
\pgfsetlinewidth{0.100000\du}
\pgfsetdash{}{0pt}
\pgfsetdash{}{0pt}
\pgfsetbuttcap
{
\definecolor{dialinecolor}{rgb}{0.000000, 0.000000, 0.000000}
\pgfsetfillcolor{dialinecolor}
% was here!!!
\definecolor{dialinecolor}{rgb}{0.000000, 0.000000, 0.000000}
\pgfsetstrokecolor{dialinecolor}
\draw (32.250000\du,66.684000\du)--(34.716508\du,68.107997\du);
}
\definecolor{dialinecolor}{rgb}{0.000000, 0.000000, 0.000000}
\pgfsetstrokecolor{dialinecolor}
\draw (32.250000\du,66.684000\du)--(34.716508\du,68.107997\du);
\pgfsetlinewidth{0.100000\du}
\pgfsetdash{}{0pt}
\pgfsetmiterjoin
\pgfsetbuttcap
\definecolor{dialinecolor}{rgb}{0.000000, 0.000000, 0.000000}
\pgfsetfillcolor{dialinecolor}
\pgfpathmoveto{\pgfpoint{34.716508\du}{68.107997\du}}
\pgfpathcurveto{\pgfpoint{34.654009\du}{68.216251\du}}{\pgfpoint{34.483257\du}{68.262007\du}}{\pgfpoint{34.375003\du}{68.199508\du}}
\pgfpathcurveto{\pgfpoint{34.266749\du}{68.137009\du}}{\pgfpoint{34.220993\du}{67.966257\du}}{\pgfpoint{34.283492\du}{67.858003\du}}
\pgfpathcurveto{\pgfpoint{34.345991\du}{67.749749\du}}{\pgfpoint{34.516743\du}{67.703993\du}}{\pgfpoint{34.624997\du}{67.766492\du}}
\pgfpathcurveto{\pgfpoint{34.733251\du}{67.828991\du}}{\pgfpoint{34.779007\du}{67.999743\du}}{\pgfpoint{34.716508\du}{68.107997\du}}
\pgfusepath{fill}
\definecolor{dialinecolor}{rgb}{0.000000, 0.000000, 0.000000}
\pgfsetstrokecolor{dialinecolor}
\pgfpathmoveto{\pgfpoint{34.716508\du}{68.107997\du}}
\pgfpathcurveto{\pgfpoint{34.654009\du}{68.216251\du}}{\pgfpoint{34.483257\du}{68.262007\du}}{\pgfpoint{34.375003\du}{68.199508\du}}
\pgfpathcurveto{\pgfpoint{34.266749\du}{68.137009\du}}{\pgfpoint{34.220993\du}{67.966257\du}}{\pgfpoint{34.283492\du}{67.858003\du}}
\pgfpathcurveto{\pgfpoint{34.345991\du}{67.749749\du}}{\pgfpoint{34.516743\du}{67.703993\du}}{\pgfpoint{34.624997\du}{67.766492\du}}
\pgfpathcurveto{\pgfpoint{34.733251\du}{67.828991\du}}{\pgfpoint{34.779007\du}{67.999743\du}}{\pgfpoint{34.716508\du}{68.107997\du}}
\pgfusepath{stroke}
\pgfsetlinewidth{0.100000\du}
\pgfsetdash{}{0pt}
\pgfsetdash{}{0pt}
\pgfsetbuttcap
{
\definecolor{dialinecolor}{rgb}{0.000000, 0.000000, 0.000000}
\pgfsetfillcolor{dialinecolor}
% was here!!!
\pgfsetarrowsend{latex}
\definecolor{dialinecolor}{rgb}{0.000000, 0.000000, 0.000000}
\pgfsetstrokecolor{dialinecolor}
\draw (27.750000\du,69.282000\du)--(28.875003\du,67.333495\du);
}
\pgfsetlinewidth{0.100000\du}
\pgfsetdash{}{0pt}
\pgfsetdash{}{0pt}
\pgfsetbuttcap
{
\definecolor{dialinecolor}{rgb}{0.000000, 0.000000, 0.000000}
\pgfsetfillcolor{dialinecolor}
% was here!!!
\definecolor{dialinecolor}{rgb}{0.000000, 0.000000, 0.000000}
\pgfsetstrokecolor{dialinecolor}
\draw (30.000000\du,65.384900\du)--(28.750000\du,67.550000\du);
}
\pgfsetlinewidth{0.100000\du}
\pgfsetdash{}{0pt}
\pgfsetdash{}{0pt}
\pgfsetbuttcap
{
\definecolor{dialinecolor}{rgb}{0.000000, 0.000000, 0.000000}
\pgfsetfillcolor{dialinecolor}
% was here!!!
\pgfsetarrowsend{latex}
\definecolor{dialinecolor}{rgb}{0.000000, 0.000000, 0.000000}
\pgfsetstrokecolor{dialinecolor}
\draw (27.966508\du,69.406997\du)--(25.283492\du,67.858003\du);
}
\definecolor{dialinecolor}{rgb}{0.000000, 0.000000, 0.000000}
\pgfsetstrokecolor{dialinecolor}
\draw (27.966508\du,69.406997\du)--(25.283492\du,67.858003\du);
\pgfsetlinewidth{0.100000\du}
\pgfsetdash{}{0pt}
\pgfsetmiterjoin
\pgfsetbuttcap
\definecolor{dialinecolor}{rgb}{0.000000, 0.000000, 0.000000}
\pgfsetfillcolor{dialinecolor}
\pgfpathmoveto{\pgfpoint{27.966508\du}{69.406997\du}}
\pgfpathcurveto{\pgfpoint{27.904009\du}{69.515251\du}}{\pgfpoint{27.733257\du}{69.561007\du}}{\pgfpoint{27.625003\du}{69.498508\du}}
\pgfpathcurveto{\pgfpoint{27.516749\du}{69.436009\du}}{\pgfpoint{27.470993\du}{69.265257\du}}{\pgfpoint{27.533492\du}{69.157003\du}}
\pgfpathcurveto{\pgfpoint{27.595991\du}{69.048749\du}}{\pgfpoint{27.766743\du}{69.002993\du}}{\pgfpoint{27.874997\du}{69.065492\du}}
\pgfpathcurveto{\pgfpoint{27.983251\du}{69.127991\du}}{\pgfpoint{28.029007\du}{69.298743\du}}{\pgfpoint{27.966508\du}{69.406997\du}}
\pgfusepath{fill}
\definecolor{dialinecolor}{rgb}{0.000000, 0.000000, 0.000000}
\pgfsetstrokecolor{dialinecolor}
\pgfpathmoveto{\pgfpoint{27.966508\du}{69.406997\du}}
\pgfpathcurveto{\pgfpoint{27.904009\du}{69.515251\du}}{\pgfpoint{27.733257\du}{69.561007\du}}{\pgfpoint{27.625003\du}{69.498508\du}}
\pgfpathcurveto{\pgfpoint{27.516749\du}{69.436009\du}}{\pgfpoint{27.470993\du}{69.265257\du}}{\pgfpoint{27.533492\du}{69.157003\du}}
\pgfpathcurveto{\pgfpoint{27.595991\du}{69.048749\du}}{\pgfpoint{27.766743\du}{69.002993\du}}{\pgfpoint{27.874997\du}{69.065492\du}}
\pgfpathcurveto{\pgfpoint{27.983251\du}{69.127991\du}}{\pgfpoint{28.029007\du}{69.298743\du}}{\pgfpoint{27.966508\du}{69.406997\du}}
\pgfusepath{stroke}
\pgfsetlinewidth{0.100000\du}
\pgfsetdash{}{0pt}
\pgfsetdash{}{0pt}
\pgfsetbuttcap
{
\definecolor{dialinecolor}{rgb}{0.000000, 0.000000, 0.000000}
\pgfsetfillcolor{dialinecolor}
% was here!!!
\definecolor{dialinecolor}{rgb}{0.000000, 0.000000, 0.000000}
\pgfsetstrokecolor{dialinecolor}
\draw (23.250000\du,66.684000\du)--(25.500000\du,67.983000\du);
}
\pgfsetlinewidth{0.100000\du}
\pgfsetdash{}{0pt}
\pgfsetdash{}{0pt}
\pgfsetbuttcap
{
\definecolor{dialinecolor}{rgb}{0.000000, 0.000000, 0.000000}
\pgfsetfillcolor{dialinecolor}
% was here!!!
\pgfsetarrowsend{latex}
\definecolor{dialinecolor}{rgb}{0.000000, 0.000000, 0.000000}
\pgfsetstrokecolor{dialinecolor}
\draw (32.466508\du,72.005097\du)--(29.783492\du,70.456103\du);
}
\definecolor{dialinecolor}{rgb}{0.000000, 0.000000, 0.000000}
\pgfsetstrokecolor{dialinecolor}
\draw (32.466508\du,72.005097\du)--(29.783492\du,70.456103\du);
\pgfsetlinewidth{0.100000\du}
\pgfsetdash{}{0pt}
\pgfsetmiterjoin
\pgfsetbuttcap
\definecolor{dialinecolor}{rgb}{0.000000, 0.000000, 0.000000}
\pgfsetfillcolor{dialinecolor}
\pgfpathmoveto{\pgfpoint{32.466508\du}{72.005097\du}}
\pgfpathcurveto{\pgfpoint{32.404009\du}{72.113351\du}}{\pgfpoint{32.233257\du}{72.159107\du}}{\pgfpoint{32.125003\du}{72.096608\du}}
\pgfpathcurveto{\pgfpoint{32.016749\du}{72.034109\du}}{\pgfpoint{31.970993\du}{71.863357\du}}{\pgfpoint{32.033492\du}{71.755103\du}}
\pgfpathcurveto{\pgfpoint{32.095991\du}{71.646849\du}}{\pgfpoint{32.266743\du}{71.601093\du}}{\pgfpoint{32.374997\du}{71.663592\du}}
\pgfpathcurveto{\pgfpoint{32.483251\du}{71.726091\du}}{\pgfpoint{32.529007\du}{71.896843\du}}{\pgfpoint{32.466508\du}{72.005097\du}}
\pgfusepath{fill}
\definecolor{dialinecolor}{rgb}{0.000000, 0.000000, 0.000000}
\pgfsetstrokecolor{dialinecolor}
\pgfpathmoveto{\pgfpoint{32.466508\du}{72.005097\du}}
\pgfpathcurveto{\pgfpoint{32.404009\du}{72.113351\du}}{\pgfpoint{32.233257\du}{72.159107\du}}{\pgfpoint{32.125003\du}{72.096608\du}}
\pgfpathcurveto{\pgfpoint{32.016749\du}{72.034109\du}}{\pgfpoint{31.970993\du}{71.863357\du}}{\pgfpoint{32.033492\du}{71.755103\du}}
\pgfpathcurveto{\pgfpoint{32.095991\du}{71.646849\du}}{\pgfpoint{32.266743\du}{71.601093\du}}{\pgfpoint{32.374997\du}{71.663592\du}}
\pgfpathcurveto{\pgfpoint{32.483251\du}{71.726091\du}}{\pgfpoint{32.529007\du}{71.896843\du}}{\pgfpoint{32.466508\du}{72.005097\du}}
\pgfusepath{stroke}
\pgfsetlinewidth{0.100000\du}
\pgfsetdash{}{0pt}
\pgfsetdash{}{0pt}
\pgfsetbuttcap
{
\definecolor{dialinecolor}{rgb}{0.000000, 0.000000, 0.000000}
\pgfsetfillcolor{dialinecolor}
% was here!!!
\definecolor{dialinecolor}{rgb}{0.000000, 0.000000, 0.000000}
\pgfsetstrokecolor{dialinecolor}
\draw (27.750000\du,69.282000\du)--(30.000000\du,70.581100\du);
}
\pgfsetlinewidth{0.100000\du}
\pgfsetdash{}{0pt}
\pgfsetdash{}{0pt}
\pgfsetbuttcap
{
\definecolor{dialinecolor}{rgb}{0.000000, 0.000000, 0.000000}
\pgfsetfillcolor{dialinecolor}
% was here!!!
\pgfsetarrowsend{latex}
\definecolor{dialinecolor}{rgb}{0.000000, 0.000000, 0.000000}
\pgfsetstrokecolor{dialinecolor}
\draw (34.500000\du,67.983000\du)--(33.125002\du,70.364607\du);
}
\pgfsetlinewidth{0.100000\du}
\pgfsetdash{}{0pt}
\pgfsetdash{}{0pt}
\pgfsetbuttcap
{
\definecolor{dialinecolor}{rgb}{0.000000, 0.000000, 0.000000}
\pgfsetfillcolor{dialinecolor}
% was here!!!
\definecolor{dialinecolor}{rgb}{0.000000, 0.000000, 0.000000}
\pgfsetstrokecolor{dialinecolor}
\draw (33.250000\du,70.148100\du)--(32.250000\du,71.880100\du);
}
% setfont left to latex
\definecolor{dialinecolor}{rgb}{0.000000, 0.000000, 0.000000}
\pgfsetstrokecolor{dialinecolor}
\node[anchor=west] at (23.000000\du,59.000000\du){};
% setfont left to latex
\definecolor{dialinecolor}{rgb}{0.000000, 0.000000, 0.000000}
\pgfsetstrokecolor{dialinecolor}
\node at (20.500000\du,59.545200\du){$a\_4$};
% setfont left to latex
\definecolor{dialinecolor}{rgb}{0.000000, 0.000000, 0.000000}
\pgfsetstrokecolor{dialinecolor}
\node at (18.250000\du,63.442400\du){$b\_4$};
% setfont left to latex
\definecolor{dialinecolor}{rgb}{0.000000, 0.000000, 0.000000}
\pgfsetstrokecolor{dialinecolor}
\node at (27.750000\du,63.442400\du){$e\_14$};
% setfont left to latex
\definecolor{dialinecolor}{rgb}{0.000000, 0.000000, 0.000000}
\pgfsetstrokecolor{dialinecolor}
\node at (25.500000\du,69.071500\du){$e\_41$};
% setfont left to latex
\definecolor{dialinecolor}{rgb}{0.000000, 0.000000, 0.000000}
\pgfsetstrokecolor{dialinecolor}
\node at (23.250000\du,64.308400\du){$e\_44$};
% setfont left to latex
\definecolor{dialinecolor}{rgb}{0.000000, 0.000000, 0.000000}
\pgfsetstrokecolor{dialinecolor}
\node at (29.869900\du,68.274700\du){$e\_11$};
% setfont left to latex
\definecolor{dialinecolor}{rgb}{0.000000, 0.000000, 0.000000}
\pgfsetstrokecolor{dialinecolor}
\node at (34.679100\du,67.422700\du){$a\_1$};
% setfont left to latex
\definecolor{dialinecolor}{rgb}{0.000000, 0.000000, 0.000000}
\pgfsetstrokecolor{dialinecolor}
\node at (31.579600\du,72.418400\du){$b\_1$};
% setfont left to latex
\definecolor{dialinecolor}{rgb}{0.000000, 0.000000, 0.000000}
\pgfsetstrokecolor{dialinecolor}
\node at (32.715200\du,69.856000\du){$S\_11$};
% setfont left to latex
\definecolor{dialinecolor}{rgb}{0.000000, 0.000000, 0.000000}
\pgfsetstrokecolor{dialinecolor}
\node at (36.770800\du,65.974200\du){DUT};
\pgfsetlinewidth{0.100000\du}
\pgfsetdash{}{0pt}
\pgfsetdash{}{0pt}
\pgfsetbuttcap
{
\definecolor{dialinecolor}{rgb}{0.000000, 0.000000, 0.000000}
\pgfsetfillcolor{dialinecolor}
% was here!!!
\pgfsetarrowsend{latex}
\definecolor{dialinecolor}{rgb}{0.000000, 0.000000, 0.000000}
\pgfsetstrokecolor{dialinecolor}
\draw (34.500000\du,67.983000\du)--(37.250000\du,67.983000\du);
}
\pgfsetlinewidth{0.100000\du}
\pgfsetdash{}{0pt}
\pgfsetdash{}{0pt}
\pgfsetbuttcap
{
\definecolor{dialinecolor}{rgb}{0.000000, 0.000000, 0.000000}
\pgfsetfillcolor{dialinecolor}
% was here!!!
\definecolor{dialinecolor}{rgb}{0.000000, 0.000000, 0.000000}
\pgfsetstrokecolor{dialinecolor}
\draw (37.000000\du,67.983000\du)--(39.250000\du,67.983000\du);
}
\definecolor{dialinecolor}{rgb}{0.000000, 0.000000, 0.000000}
\pgfsetstrokecolor{dialinecolor}
\draw (37.000000\du,67.983000\du)--(39.250000\du,67.983000\du);
\pgfsetlinewidth{0.100000\du}
\pgfsetdash{}{0pt}
\pgfsetmiterjoin
\pgfsetbuttcap
\definecolor{dialinecolor}{rgb}{0.000000, 0.000000, 0.000000}
\pgfsetfillcolor{dialinecolor}
\pgfpathmoveto{\pgfpoint{39.250000\du}{67.983000\du}}
\pgfpathcurveto{\pgfpoint{39.250000\du}{68.108000\du}}{\pgfpoint{39.125000\du}{68.233000\du}}{\pgfpoint{39.000000\du}{68.233000\du}}
\pgfpathcurveto{\pgfpoint{38.875000\du}{68.233000\du}}{\pgfpoint{38.750000\du}{68.108000\du}}{\pgfpoint{38.750000\du}{67.983000\du}}
\pgfpathcurveto{\pgfpoint{38.750000\du}{67.858000\du}}{\pgfpoint{38.875000\du}{67.733000\du}}{\pgfpoint{39.000000\du}{67.733000\du}}
\pgfpathcurveto{\pgfpoint{39.125000\du}{67.733000\du}}{\pgfpoint{39.250000\du}{67.858000\du}}{\pgfpoint{39.250000\du}{67.983000\du}}
\pgfusepath{fill}
\definecolor{dialinecolor}{rgb}{0.000000, 0.000000, 0.000000}
\pgfsetstrokecolor{dialinecolor}
\pgfpathmoveto{\pgfpoint{39.250000\du}{67.983000\du}}
\pgfpathcurveto{\pgfpoint{39.250000\du}{68.108000\du}}{\pgfpoint{39.125000\du}{68.233000\du}}{\pgfpoint{39.000000\du}{68.233000\du}}
\pgfpathcurveto{\pgfpoint{38.875000\du}{68.233000\du}}{\pgfpoint{38.750000\du}{68.108000\du}}{\pgfpoint{38.750000\du}{67.983000\du}}
\pgfpathcurveto{\pgfpoint{38.750000\du}{67.858000\du}}{\pgfpoint{38.875000\du}{67.733000\du}}{\pgfpoint{39.000000\du}{67.733000\du}}
\pgfpathcurveto{\pgfpoint{39.125000\du}{67.733000\du}}{\pgfpoint{39.250000\du}{67.858000\du}}{\pgfpoint{39.250000\du}{67.983000\du}}
\pgfusepath{stroke}
\pgfsetlinewidth{0.100000\du}
\pgfsetdash{}{0pt}
\pgfsetdash{}{0pt}
\pgfsetbuttcap
{
\definecolor{dialinecolor}{rgb}{0.000000, 0.000000, 0.000000}
\pgfsetfillcolor{dialinecolor}
% was here!!!
\pgfsetarrowsend{latex}
\definecolor{dialinecolor}{rgb}{0.000000, 0.000000, 0.000000}
\pgfsetstrokecolor{dialinecolor}
\draw (41.250000\du,71.880100\du)--(39.875002\du,69.498493\du);
}
\pgfsetlinewidth{0.100000\du}
\pgfsetdash{}{0pt}
\pgfsetdash{}{0pt}
\pgfsetbuttcap
{
\definecolor{dialinecolor}{rgb}{0.000000, 0.000000, 0.000000}
\pgfsetfillcolor{dialinecolor}
% was here!!!
\definecolor{dialinecolor}{rgb}{0.000000, 0.000000, 0.000000}
\pgfsetstrokecolor{dialinecolor}
\draw (39.000000\du,67.983000\du)--(40.000000\du,69.715000\du);
}
\pgfsetlinewidth{0.100000\du}
\pgfsetdash{}{0pt}
\pgfsetdash{}{0pt}
\pgfsetbuttcap
{
\definecolor{dialinecolor}{rgb}{0.000000, 0.000000, 0.000000}
\pgfsetfillcolor{dialinecolor}
% was here!!!
\pgfsetarrowsend{latex}
\definecolor{dialinecolor}{rgb}{0.000000, 0.000000, 0.000000}
\pgfsetstrokecolor{dialinecolor}
\draw (41.374998\du,71.663593\du)--(39.875002\du,74.261707\du);
}
\definecolor{dialinecolor}{rgb}{0.000000, 0.000000, 0.000000}
\pgfsetstrokecolor{dialinecolor}
\draw (41.374998\du,71.663593\du)--(39.875002\du,74.261707\du);
\pgfsetlinewidth{0.100000\du}
\pgfsetdash{}{0pt}
\pgfsetmiterjoin
\pgfsetbuttcap
\definecolor{dialinecolor}{rgb}{0.000000, 0.000000, 0.000000}
\pgfsetfillcolor{dialinecolor}
\pgfpathmoveto{\pgfpoint{41.374998\du}{71.663593\du}}
\pgfpathcurveto{\pgfpoint{41.483252\du}{71.726092\du}}{\pgfpoint{41.529006\du}{71.896845\du}}{\pgfpoint{41.466507\du}{72.005098\du}}
\pgfpathcurveto{\pgfpoint{41.404008\du}{72.113352\du}}{\pgfpoint{41.233255\du}{72.159106\du}}{\pgfpoint{41.125002\du}{72.096607\du}}
\pgfpathcurveto{\pgfpoint{41.016748\du}{72.034108\du}}{\pgfpoint{40.970994\du}{71.863355\du}}{\pgfpoint{41.033493\du}{71.755102\du}}
\pgfpathcurveto{\pgfpoint{41.095992\du}{71.646848\du}}{\pgfpoint{41.266745\du}{71.601094\du}}{\pgfpoint{41.374998\du}{71.663593\du}}
\pgfusepath{fill}
\definecolor{dialinecolor}{rgb}{0.000000, 0.000000, 0.000000}
\pgfsetstrokecolor{dialinecolor}
\pgfpathmoveto{\pgfpoint{41.374998\du}{71.663593\du}}
\pgfpathcurveto{\pgfpoint{41.483252\du}{71.726092\du}}{\pgfpoint{41.529006\du}{71.896845\du}}{\pgfpoint{41.466507\du}{72.005098\du}}
\pgfpathcurveto{\pgfpoint{41.404008\du}{72.113352\du}}{\pgfpoint{41.233255\du}{72.159106\du}}{\pgfpoint{41.125002\du}{72.096607\du}}
\pgfpathcurveto{\pgfpoint{41.016748\du}{72.034108\du}}{\pgfpoint{40.970994\du}{71.863355\du}}{\pgfpoint{41.033493\du}{71.755102\du}}
\pgfpathcurveto{\pgfpoint{41.095992\du}{71.646848\du}}{\pgfpoint{41.266745\du}{71.601094\du}}{\pgfpoint{41.374998\du}{71.663593\du}}
\pgfusepath{stroke}
\pgfsetlinewidth{0.100000\du}
\pgfsetdash{}{0pt}
\pgfsetdash{}{0pt}
\pgfsetbuttcap
{
\definecolor{dialinecolor}{rgb}{0.000000, 0.000000, 0.000000}
\pgfsetfillcolor{dialinecolor}
% was here!!!
\definecolor{dialinecolor}{rgb}{0.000000, 0.000000, 0.000000}
\pgfsetstrokecolor{dialinecolor}
\draw (39.000000\du,75.777200\du)--(40.000000\du,74.045200\du);
}
% setfont left to latex
\definecolor{dialinecolor}{rgb}{0.000000, 0.000000, 0.000000}
\pgfsetstrokecolor{dialinecolor}
\node at (42.026100\du,72.463000\du){$a\_2$};
% setfont left to latex
\definecolor{dialinecolor}{rgb}{0.000000, 0.000000, 0.000000}
\pgfsetstrokecolor{dialinecolor}
\node at (39.250000\du,67.415900\du){$b\_2$};
% setfont left to latex
\definecolor{dialinecolor}{rgb}{0.000000, 0.000000, 0.000000}
\pgfsetstrokecolor{dialinecolor}
\node at (37.000000\du,67.339500\du){$S\_21$};
\pgfsetlinewidth{0.100000\du}
\pgfsetdash{}{0pt}
\pgfsetdash{}{0pt}
\pgfsetbuttcap
{
\definecolor{dialinecolor}{rgb}{0.000000, 0.000000, 0.000000}
\pgfsetfillcolor{dialinecolor}
% was here!!!
\pgfsetarrowsend{latex}
\definecolor{dialinecolor}{rgb}{0.000000, 0.000000, 0.000000}
\pgfsetstrokecolor{dialinecolor}
\draw (39.000000\du,67.983000\du)--(41.466508\du,66.559003\du);
}
\pgfsetlinewidth{0.100000\du}
\pgfsetdash{}{0pt}
\pgfsetdash{}{0pt}
\pgfsetbuttcap
{
\definecolor{dialinecolor}{rgb}{0.000000, 0.000000, 0.000000}
\pgfsetfillcolor{dialinecolor}
% was here!!!
\definecolor{dialinecolor}{rgb}{0.000000, 0.000000, 0.000000}
\pgfsetstrokecolor{dialinecolor}
\draw (41.250000\du,66.684000\du)--(43.716504\du,65.259896\du);
}
\definecolor{dialinecolor}{rgb}{0.000000, 0.000000, 0.000000}
\pgfsetstrokecolor{dialinecolor}
\draw (41.250000\du,66.684000\du)--(43.716504\du,65.259896\du);
\pgfsetlinewidth{0.100000\du}
\pgfsetdash{}{0pt}
\pgfsetmiterjoin
\pgfsetbuttcap
\definecolor{dialinecolor}{rgb}{0.000000, 0.000000, 0.000000}
\pgfsetfillcolor{dialinecolor}
\pgfpathmoveto{\pgfpoint{43.716504\du}{65.259896\du}}
\pgfpathcurveto{\pgfpoint{43.779006\du}{65.368147\du}}{\pgfpoint{43.733256\du}{65.538902\du}}{\pgfpoint{43.625004\du}{65.601404\du}}
\pgfpathcurveto{\pgfpoint{43.516753\du}{65.663906\du}}{\pgfpoint{43.345998\du}{65.618156\du}}{\pgfpoint{43.283496\du}{65.509904\du}}
\pgfpathcurveto{\pgfpoint{43.220994\du}{65.401653\du}}{\pgfpoint{43.266744\du}{65.230898\du}}{\pgfpoint{43.374996\du}{65.168396\du}}
\pgfpathcurveto{\pgfpoint{43.483247\du}{65.105894\du}}{\pgfpoint{43.654002\du}{65.151644\du}}{\pgfpoint{43.716504\du}{65.259896\du}}
\pgfusepath{fill}
\definecolor{dialinecolor}{rgb}{0.000000, 0.000000, 0.000000}
\pgfsetstrokecolor{dialinecolor}
\pgfpathmoveto{\pgfpoint{43.716504\du}{65.259896\du}}
\pgfpathcurveto{\pgfpoint{43.779006\du}{65.368147\du}}{\pgfpoint{43.733256\du}{65.538902\du}}{\pgfpoint{43.625004\du}{65.601404\du}}
\pgfpathcurveto{\pgfpoint{43.516753\du}{65.663906\du}}{\pgfpoint{43.345998\du}{65.618156\du}}{\pgfpoint{43.283496\du}{65.509904\du}}
\pgfpathcurveto{\pgfpoint{43.220994\du}{65.401653\du}}{\pgfpoint{43.266744\du}{65.230898\du}}{\pgfpoint{43.374996\du}{65.168396\du}}
\pgfpathcurveto{\pgfpoint{43.483247\du}{65.105894\du}}{\pgfpoint{43.654002\du}{65.151644\du}}{\pgfpoint{43.716504\du}{65.259896\du}}
\pgfusepath{stroke}
\pgfsetlinewidth{0.100000\du}
\pgfsetdash{}{0pt}
\pgfsetdash{}{0pt}
\pgfsetbuttcap
{
\definecolor{dialinecolor}{rgb}{0.000000, 0.000000, 0.000000}
\pgfsetfillcolor{dialinecolor}
% was here!!!
\pgfsetarrowsend{latex}
\definecolor{dialinecolor}{rgb}{0.000000, 0.000000, 0.000000}
\pgfsetstrokecolor{dialinecolor}
\draw (45.750000\du,69.282000\du)--(44.374997\du,66.900495\du);
}
\pgfsetlinewidth{0.100000\du}
\pgfsetdash{}{0pt}
\pgfsetdash{}{0pt}
\pgfsetbuttcap
{
\definecolor{dialinecolor}{rgb}{0.000000, 0.000000, 0.000000}
\pgfsetfillcolor{dialinecolor}
% was here!!!
\definecolor{dialinecolor}{rgb}{0.000000, 0.000000, 0.000000}
\pgfsetstrokecolor{dialinecolor}
\draw (43.500000\du,65.384900\du)--(44.500000\du,67.117000\du);
}
\pgfsetlinewidth{0.100000\du}
\pgfsetdash{}{0pt}
\pgfsetdash{}{0pt}
\pgfsetbuttcap
{
\definecolor{dialinecolor}{rgb}{0.000000, 0.000000, 0.000000}
\pgfsetfillcolor{dialinecolor}
% was here!!!
\pgfsetarrowsend{latex}
\definecolor{dialinecolor}{rgb}{0.000000, 0.000000, 0.000000}
\pgfsetstrokecolor{dialinecolor}
\draw (45.966504\du,69.156996\du)--(43.283496\du,70.706104\du);
}
\definecolor{dialinecolor}{rgb}{0.000000, 0.000000, 0.000000}
\pgfsetstrokecolor{dialinecolor}
\draw (45.966504\du,69.156996\du)--(43.283496\du,70.706104\du);
\pgfsetlinewidth{0.100000\du}
\pgfsetdash{}{0pt}
\pgfsetmiterjoin
\pgfsetbuttcap
\definecolor{dialinecolor}{rgb}{0.000000, 0.000000, 0.000000}
\pgfsetfillcolor{dialinecolor}
\pgfpathmoveto{\pgfpoint{45.966504\du}{69.156996\du}}
\pgfpathcurveto{\pgfpoint{46.029006\du}{69.265247\du}}{\pgfpoint{45.983256\du}{69.436002\du}}{\pgfpoint{45.875004\du}{69.498504\du}}
\pgfpathcurveto{\pgfpoint{45.766753\du}{69.561006\du}}{\pgfpoint{45.595998\du}{69.515256\du}}{\pgfpoint{45.533496\du}{69.407004\du}}
\pgfpathcurveto{\pgfpoint{45.470994\du}{69.298753\du}}{\pgfpoint{45.516744\du}{69.127998\du}}{\pgfpoint{45.624996\du}{69.065496\du}}
\pgfpathcurveto{\pgfpoint{45.733247\du}{69.002994\du}}{\pgfpoint{45.904002\du}{69.048744\du}}{\pgfpoint{45.966504\du}{69.156996\du}}
\pgfusepath{fill}
\definecolor{dialinecolor}{rgb}{0.000000, 0.000000, 0.000000}
\pgfsetstrokecolor{dialinecolor}
\pgfpathmoveto{\pgfpoint{45.966504\du}{69.156996\du}}
\pgfpathcurveto{\pgfpoint{46.029006\du}{69.265247\du}}{\pgfpoint{45.983256\du}{69.436002\du}}{\pgfpoint{45.875004\du}{69.498504\du}}
\pgfpathcurveto{\pgfpoint{45.766753\du}{69.561006\du}}{\pgfpoint{45.595998\du}{69.515256\du}}{\pgfpoint{45.533496\du}{69.407004\du}}
\pgfpathcurveto{\pgfpoint{45.470994\du}{69.298753\du}}{\pgfpoint{45.516744\du}{69.127998\du}}{\pgfpoint{45.624996\du}{69.065496\du}}
\pgfpathcurveto{\pgfpoint{45.733247\du}{69.002994\du}}{\pgfpoint{45.904002\du}{69.048744\du}}{\pgfpoint{45.966504\du}{69.156996\du}}
\pgfusepath{stroke}
\pgfsetlinewidth{0.100000\du}
\pgfsetdash{}{0pt}
\pgfsetdash{}{0pt}
\pgfsetbuttcap
{
\definecolor{dialinecolor}{rgb}{0.000000, 0.000000, 0.000000}
\pgfsetfillcolor{dialinecolor}
% was here!!!
\definecolor{dialinecolor}{rgb}{0.000000, 0.000000, 0.000000}
\pgfsetstrokecolor{dialinecolor}
\draw (41.250000\du,71.880100\du)--(43.500000\du,70.581100\du);
}
\pgfsetlinewidth{0.100000\du}
\pgfsetdash{}{0pt}
\pgfsetdash{}{0pt}
\pgfsetbuttcap
{
\definecolor{dialinecolor}{rgb}{0.000000, 0.000000, 0.000000}
\pgfsetfillcolor{dialinecolor}
% was here!!!
\pgfsetarrowsend{latex}
\definecolor{dialinecolor}{rgb}{0.000000, 0.000000, 0.000000}
\pgfsetstrokecolor{dialinecolor}
\draw (43.500000\du,65.384900\du)--(45.966508\du,63.960903\du);
}
\pgfsetlinewidth{0.100000\du}
\pgfsetdash{}{0pt}
\pgfsetdash{}{0pt}
\pgfsetbuttcap
{
\definecolor{dialinecolor}{rgb}{0.000000, 0.000000, 0.000000}
\pgfsetfillcolor{dialinecolor}
% was here!!!
\definecolor{dialinecolor}{rgb}{0.000000, 0.000000, 0.000000}
\pgfsetstrokecolor{dialinecolor}
\draw (45.750000\du,64.085900\du)--(48.216504\du,62.661796\du);
}
\definecolor{dialinecolor}{rgb}{0.000000, 0.000000, 0.000000}
\pgfsetstrokecolor{dialinecolor}
\draw (45.750000\du,64.085900\du)--(48.216504\du,62.661796\du);
\pgfsetlinewidth{0.100000\du}
\pgfsetdash{}{0pt}
\pgfsetmiterjoin
\pgfsetbuttcap
\definecolor{dialinecolor}{rgb}{0.000000, 0.000000, 0.000000}
\pgfsetfillcolor{dialinecolor}
\pgfpathmoveto{\pgfpoint{48.216504\du}{62.661796\du}}
\pgfpathcurveto{\pgfpoint{48.279006\du}{62.770047\du}}{\pgfpoint{48.233256\du}{62.940802\du}}{\pgfpoint{48.125004\du}{63.003304\du}}
\pgfpathcurveto{\pgfpoint{48.016753\du}{63.065806\du}}{\pgfpoint{47.845998\du}{63.020056\du}}{\pgfpoint{47.783496\du}{62.911804\du}}
\pgfpathcurveto{\pgfpoint{47.720994\du}{62.803553\du}}{\pgfpoint{47.766744\du}{62.632798\du}}{\pgfpoint{47.874996\du}{62.570296\du}}
\pgfpathcurveto{\pgfpoint{47.983247\du}{62.507794\du}}{\pgfpoint{48.154002\du}{62.553544\du}}{\pgfpoint{48.216504\du}{62.661796\du}}
\pgfusepath{fill}
\definecolor{dialinecolor}{rgb}{0.000000, 0.000000, 0.000000}
\pgfsetstrokecolor{dialinecolor}
\pgfpathmoveto{\pgfpoint{48.216504\du}{62.661796\du}}
\pgfpathcurveto{\pgfpoint{48.279006\du}{62.770047\du}}{\pgfpoint{48.233256\du}{62.940802\du}}{\pgfpoint{48.125004\du}{63.003304\du}}
\pgfpathcurveto{\pgfpoint{48.016753\du}{63.065806\du}}{\pgfpoint{47.845998\du}{63.020056\du}}{\pgfpoint{47.783496\du}{62.911804\du}}
\pgfpathcurveto{\pgfpoint{47.720994\du}{62.803553\du}}{\pgfpoint{47.766744\du}{62.632798\du}}{\pgfpoint{47.874996\du}{62.570296\du}}
\pgfpathcurveto{\pgfpoint{47.983247\du}{62.507794\du}}{\pgfpoint{48.154002\du}{62.553544\du}}{\pgfpoint{48.216504\du}{62.661796\du}}
\pgfusepath{stroke}
\pgfsetlinewidth{0.100000\du}
\pgfsetdash{}{0pt}
\pgfsetdash{}{0pt}
\pgfsetbuttcap
{
\definecolor{dialinecolor}{rgb}{0.000000, 0.000000, 0.000000}
\pgfsetfillcolor{dialinecolor}
% was here!!!
\pgfsetarrowsend{latex}
\definecolor{dialinecolor}{rgb}{0.000000, 0.000000, 0.000000}
\pgfsetstrokecolor{dialinecolor}
\draw (50.250000\du,66.684000\du)--(48.875002\du,64.302393\du);
}
\pgfsetlinewidth{0.100000\du}
\pgfsetdash{}{0pt}
\pgfsetdash{}{0pt}
\pgfsetbuttcap
{
\definecolor{dialinecolor}{rgb}{0.000000, 0.000000, 0.000000}
\pgfsetfillcolor{dialinecolor}
% was here!!!
\definecolor{dialinecolor}{rgb}{0.000000, 0.000000, 0.000000}
\pgfsetstrokecolor{dialinecolor}
\draw (48.000000\du,62.786800\du)--(49.000000\du,64.518900\du);
}
\pgfsetlinewidth{0.100000\du}
\pgfsetdash{}{0pt}
\pgfsetdash{}{0pt}
\pgfsetbuttcap
{
\definecolor{dialinecolor}{rgb}{0.000000, 0.000000, 0.000000}
\pgfsetfillcolor{dialinecolor}
% was here!!!
\pgfsetarrowsend{latex}
\definecolor{dialinecolor}{rgb}{0.000000, 0.000000, 0.000000}
\pgfsetstrokecolor{dialinecolor}
\draw (50.466508\du,66.559003\du)--(47.783492\du,68.107997\du);
}
\definecolor{dialinecolor}{rgb}{0.000000, 0.000000, 0.000000}
\pgfsetstrokecolor{dialinecolor}
\draw (50.466508\du,66.559003\du)--(47.783492\du,68.107997\du);
\pgfsetlinewidth{0.100000\du}
\pgfsetdash{}{0pt}
\pgfsetmiterjoin
\pgfsetbuttcap
\definecolor{dialinecolor}{rgb}{0.000000, 0.000000, 0.000000}
\pgfsetfillcolor{dialinecolor}
\pgfpathmoveto{\pgfpoint{50.466508\du}{66.559003\du}}
\pgfpathcurveto{\pgfpoint{50.529007\du}{66.667257\du}}{\pgfpoint{50.483251\du}{66.838009\du}}{\pgfpoint{50.374997\du}{66.900508\du}}
\pgfpathcurveto{\pgfpoint{50.266743\du}{66.963007\du}}{\pgfpoint{50.095991\du}{66.917251\du}}{\pgfpoint{50.033492\du}{66.808997\du}}
\pgfpathcurveto{\pgfpoint{49.970993\du}{66.700743\du}}{\pgfpoint{50.016749\du}{66.529991\du}}{\pgfpoint{50.125003\du}{66.467492\du}}
\pgfpathcurveto{\pgfpoint{50.233257\du}{66.404993\du}}{\pgfpoint{50.404009\du}{66.450749\du}}{\pgfpoint{50.466508\du}{66.559003\du}}
\pgfusepath{fill}
\definecolor{dialinecolor}{rgb}{0.000000, 0.000000, 0.000000}
\pgfsetstrokecolor{dialinecolor}
\pgfpathmoveto{\pgfpoint{50.466508\du}{66.559003\du}}
\pgfpathcurveto{\pgfpoint{50.529007\du}{66.667257\du}}{\pgfpoint{50.483251\du}{66.838009\du}}{\pgfpoint{50.374997\du}{66.900508\du}}
\pgfpathcurveto{\pgfpoint{50.266743\du}{66.963007\du}}{\pgfpoint{50.095991\du}{66.917251\du}}{\pgfpoint{50.033492\du}{66.808997\du}}
\pgfpathcurveto{\pgfpoint{49.970993\du}{66.700743\du}}{\pgfpoint{50.016749\du}{66.529991\du}}{\pgfpoint{50.125003\du}{66.467492\du}}
\pgfpathcurveto{\pgfpoint{50.233257\du}{66.404993\du}}{\pgfpoint{50.404009\du}{66.450749\du}}{\pgfpoint{50.466508\du}{66.559003\du}}
\pgfusepath{stroke}
\pgfsetlinewidth{0.100000\du}
\pgfsetdash{}{0pt}
\pgfsetdash{}{0pt}
\pgfsetbuttcap
{
\definecolor{dialinecolor}{rgb}{0.000000, 0.000000, 0.000000}
\pgfsetfillcolor{dialinecolor}
% was here!!!
\definecolor{dialinecolor}{rgb}{0.000000, 0.000000, 0.000000}
\pgfsetstrokecolor{dialinecolor}
\draw (45.750000\du,69.282000\du)--(48.000000\du,67.983000\du);
}
% setfont left to latex
\definecolor{dialinecolor}{rgb}{0.000000, 0.000000, 0.000000}
\pgfsetstrokecolor{dialinecolor}
\node at (45.250000\du,63.442400\du){$e\_52$};
% setfont left to latex
\definecolor{dialinecolor}{rgb}{0.000000, 0.000000, 0.000000}
\pgfsetstrokecolor{dialinecolor}
\node at (49.000000\du,69.071500\du){$e\_25$};
% setfont left to latex
\definecolor{dialinecolor}{rgb}{0.000000, 0.000000, 0.000000}
\pgfsetstrokecolor{dialinecolor}
\node at (43.500000\du,68.205500\du){$e\_22$};
% setfont left to latex
\definecolor{dialinecolor}{rgb}{0.000000, 0.000000, 0.000000}
\pgfsetstrokecolor{dialinecolor}
\node at (50.250000\du,64.308400\du){$e\_55$};
% setfont left to latex
\definecolor{dialinecolor}{rgb}{0.000000, 0.000000, 0.000000}
\pgfsetstrokecolor{dialinecolor}
\node at (39.000000\du,71.669600\du){$S\_12$};
% setfont left to latex
\definecolor{dialinecolor}{rgb}{0.000000, 0.000000, 0.000000}
\pgfsetstrokecolor{dialinecolor}
\node at (40.750000\du,69.504500\du){$S\_22$};
\pgfsetlinewidth{0.100000\du}
\pgfsetdash{}{0pt}
\pgfsetdash{}{0pt}
\pgfsetbuttcap
{
\definecolor{dialinecolor}{rgb}{0.000000, 0.000000, 0.000000}
\pgfsetfillcolor{dialinecolor}
% was here!!!
\pgfsetarrowsend{latex}
\definecolor{dialinecolor}{rgb}{0.000000, 0.000000, 0.000000}
\pgfsetstrokecolor{dialinecolor}
\draw (54.966508\du,63.960903\du)--(52.283492\du,65.509897\du);
}
\definecolor{dialinecolor}{rgb}{0.000000, 0.000000, 0.000000}
\pgfsetstrokecolor{dialinecolor}
\draw (54.490190\du,64.235897\du)--(52.283492\du,65.509897\du);
\pgfsetlinewidth{0.100000\du}
\pgfsetdash{}{0pt}
\pgfsetmiterjoin
\pgfsetbuttcap
\definecolor{dialinecolor}{rgb}{1.000000, 1.000000, 1.000000}
\pgfsetfillcolor{dialinecolor}
\pgfpathmoveto{\pgfpoint{54.923206\du}{63.985902\du}}
\pgfpathcurveto{\pgfpoint{54.985705\du}{64.094156\du}}{\pgfpoint{54.939950\du}{64.264909\du}}{\pgfpoint{54.831696\du}{64.327407\du}}
\pgfpathcurveto{\pgfpoint{54.723442\du}{64.389906\du}}{\pgfpoint{54.552689\du}{64.344151\du}}{\pgfpoint{54.490190\du}{64.235897\du}}
\pgfpathcurveto{\pgfpoint{54.427692\du}{64.127643\du}}{\pgfpoint{54.473447\du}{63.956890\du}}{\pgfpoint{54.581701\du}{63.894392\du}}
\pgfpathcurveto{\pgfpoint{54.689955\du}{63.831893\du}}{\pgfpoint{54.860708\du}{63.877648\du}}{\pgfpoint{54.923206\du}{63.985902\du}}
\pgfusepath{fill}
\definecolor{dialinecolor}{rgb}{0.000000, 0.000000, 0.000000}
\pgfsetstrokecolor{dialinecolor}
\pgfpathmoveto{\pgfpoint{54.923206\du}{63.985902\du}}
\pgfpathcurveto{\pgfpoint{54.985705\du}{64.094156\du}}{\pgfpoint{54.939950\du}{64.264909\du}}{\pgfpoint{54.831696\du}{64.327407\du}}
\pgfpathcurveto{\pgfpoint{54.723442\du}{64.389906\du}}{\pgfpoint{54.552689\du}{64.344151\du}}{\pgfpoint{54.490190\du}{64.235897\du}}
\pgfpathcurveto{\pgfpoint{54.427692\du}{64.127643\du}}{\pgfpoint{54.473447\du}{63.956890\du}}{\pgfpoint{54.581701\du}{63.894392\du}}
\pgfpathcurveto{\pgfpoint{54.689955\du}{63.831893\du}}{\pgfpoint{54.860708\du}{63.877648\du}}{\pgfpoint{54.923206\du}{63.985902\du}}
\pgfusepath{stroke}
\pgfsetlinewidth{0.100000\du}
\pgfsetdash{}{0pt}
\pgfsetdash{}{0pt}
\pgfsetbuttcap
{
\definecolor{dialinecolor}{rgb}{0.000000, 0.000000, 0.000000}
\pgfsetfillcolor{dialinecolor}
% was here!!!
\definecolor{dialinecolor}{rgb}{0.000000, 0.000000, 0.000000}
\pgfsetstrokecolor{dialinecolor}
\draw (50.250000\du,66.684000\du)--(52.500000\du,65.384900\du);
}
\pgfsetlinewidth{0.100000\du}
\pgfsetdash{}{0pt}
\pgfsetdash{}{0pt}
\pgfsetbuttcap
{
\definecolor{dialinecolor}{rgb}{0.000000, 0.000000, 0.000000}
\pgfsetfillcolor{dialinecolor}
% was here!!!
\definecolor{dialinecolor}{rgb}{0.000000, 0.000000, 0.000000}
\pgfsetstrokecolor{dialinecolor}
\draw (52.716508\du,60.063803\du)--(50.250000\du,61.487800\du);
}
\definecolor{dialinecolor}{rgb}{0.000000, 0.000000, 0.000000}
\pgfsetstrokecolor{dialinecolor}
\draw (52.240190\du,60.338797\du)--(50.250000\du,61.487800\du);
\pgfsetlinewidth{0.100000\du}
\pgfsetdash{}{0pt}
\pgfsetmiterjoin
\pgfsetbuttcap
\definecolor{dialinecolor}{rgb}{1.000000, 1.000000, 1.000000}
\pgfsetfillcolor{dialinecolor}
\pgfpathmoveto{\pgfpoint{52.673206\du}{60.088802\du}}
\pgfpathcurveto{\pgfpoint{52.735705\du}{60.197056\du}}{\pgfpoint{52.689950\du}{60.367809\du}}{\pgfpoint{52.581696\du}{60.430307\du}}
\pgfpathcurveto{\pgfpoint{52.473442\du}{60.492806\du}}{\pgfpoint{52.302689\du}{60.447051\du}}{\pgfpoint{52.240190\du}{60.338797\du}}
\pgfpathcurveto{\pgfpoint{52.177692\du}{60.230543\du}}{\pgfpoint{52.223447\du}{60.059790\du}}{\pgfpoint{52.331701\du}{59.997292\du}}
\pgfpathcurveto{\pgfpoint{52.439955\du}{59.934793\du}}{\pgfpoint{52.610708\du}{59.980548\du}}{\pgfpoint{52.673206\du}{60.088802\du}}
\pgfusepath{fill}
\definecolor{dialinecolor}{rgb}{0.000000, 0.000000, 0.000000}
\pgfsetstrokecolor{dialinecolor}
\pgfpathmoveto{\pgfpoint{52.673206\du}{60.088802\du}}
\pgfpathcurveto{\pgfpoint{52.735705\du}{60.197056\du}}{\pgfpoint{52.689950\du}{60.367809\du}}{\pgfpoint{52.581696\du}{60.430307\du}}
\pgfpathcurveto{\pgfpoint{52.473442\du}{60.492806\du}}{\pgfpoint{52.302689\du}{60.447051\du}}{\pgfpoint{52.240190\du}{60.338797\du}}
\pgfpathcurveto{\pgfpoint{52.177692\du}{60.230543\du}}{\pgfpoint{52.223447\du}{60.059790\du}}{\pgfpoint{52.331701\du}{59.997292\du}}
\pgfpathcurveto{\pgfpoint{52.439955\du}{59.934793\du}}{\pgfpoint{52.610708\du}{59.980548\du}}{\pgfpoint{52.673206\du}{60.088802\du}}
\pgfusepath{stroke}
\pgfsetlinewidth{0.100000\du}
\pgfsetdash{}{0pt}
\pgfsetdash{}{0pt}
\pgfsetbuttcap
{
\definecolor{dialinecolor}{rgb}{0.000000, 0.000000, 0.000000}
\pgfsetfillcolor{dialinecolor}
% was here!!!
\pgfsetarrowsend{latex}
\definecolor{dialinecolor}{rgb}{0.000000, 0.000000, 0.000000}
\pgfsetstrokecolor{dialinecolor}
\draw (48.000000\du,62.786800\du)--(50.466508\du,61.362803\du);
}
% setfont left to latex
\definecolor{dialinecolor}{rgb}{0.000000, 0.000000, 0.000000}
\pgfsetstrokecolor{dialinecolor}
\node at (39.674700\du,76.539700\du){$b\_3$};
% setfont left to latex
\definecolor{dialinecolor}{rgb}{0.000000, 0.000000, 0.000000}
\pgfsetstrokecolor{dialinecolor}
\node at (33.891900\du,76.432700\du){$a\_3$};
\pgfsetlinewidth{0.100000\du}
\pgfsetdash{{\pgflinewidth}{0.200000\du}}{0cm}
\pgfsetdash{{\pgflinewidth}{0.200000\du}}{0cm}
\pgfsetbuttcap
{
\definecolor{dialinecolor}{rgb}{0.000000, 0.000000, 0.000000}
\pgfsetfillcolor{dialinecolor}
% was here!!!
\pgfsetarrowsend{latex}
\definecolor{dialinecolor}{rgb}{0.000000, 0.000000, 0.000000}
\pgfsetstrokecolor{dialinecolor}
\draw (25.500000\du,62.786800\du)--(37.250000\du,62.786800\du);
}
\pgfsetlinewidth{0.100000\du}
\pgfsetdash{{\pgflinewidth}{0.200000\du}}{0cm}
\pgfsetdash{{\pgflinewidth}{0.200000\du}}{0cm}
\pgfsetbuttcap
{
\definecolor{dialinecolor}{rgb}{0.000000, 0.000000, 0.000000}
\pgfsetfillcolor{dialinecolor}
% was here!!!
\definecolor{dialinecolor}{rgb}{0.000000, 0.000000, 0.000000}
\pgfsetstrokecolor{dialinecolor}
\draw (37.000000\du,62.786800\du)--(48.000000\du,62.786800\du);
}
\pgfsetlinewidth{0.100000\du}
\pgfsetdash{{\pgflinewidth}{0.200000\du}}{0cm}
\pgfsetdash{{\pgflinewidth}{0.200000\du}}{0cm}
\pgfsetbuttcap
{
\definecolor{dialinecolor}{rgb}{0.000000, 0.000000, 0.000000}
\pgfsetfillcolor{dialinecolor}
% was here!!!
\pgfsetarrowsend{latex}
\definecolor{dialinecolor}{rgb}{0.000000, 0.000000, 0.000000}
\pgfsetstrokecolor{dialinecolor}
\draw (46.500000\du,64.518900\du)--(36.250000\du,64.518900\du);
}
\pgfsetlinewidth{0.100000\du}
\pgfsetdash{{\pgflinewidth}{0.200000\du}}{0cm}
\pgfsetdash{{\pgflinewidth}{0.200000\du}}{0cm}
\pgfsetbuttcap
{
\definecolor{dialinecolor}{rgb}{0.000000, 0.000000, 0.000000}
\pgfsetfillcolor{dialinecolor}
% was here!!!
\definecolor{dialinecolor}{rgb}{0.000000, 0.000000, 0.000000}
\pgfsetstrokecolor{dialinecolor}
\draw (37.000000\du,64.518900\du)--(27.000000\du,64.518900\du);
}
% setfont left to latex
\definecolor{dialinecolor}{rgb}{0.000000, 0.000000, 0.000000}
\pgfsetstrokecolor{dialinecolor}
\node at (37.000000\du,62.143300\du){$e\_54$};
% setfont left to latex
\definecolor{dialinecolor}{rgb}{0.000000, 0.000000, 0.000000}
\pgfsetstrokecolor{dialinecolor}
\node at (36.954200\du,63.875400\du){$e\_45$};
\pgfsetlinewidth{0.100000\du}
\pgfsetdash{}{0pt}
\pgfsetdash{}{0pt}
\pgfsetbuttcap
{
\definecolor{dialinecolor}{rgb}{0.000000, 0.000000, 0.000000}
\pgfsetfillcolor{dialinecolor}
% was here!!!
\pgfsetarrowsend{latex}
\definecolor{dialinecolor}{rgb}{0.000000, 0.000000, 0.000000}
\pgfsetstrokecolor{dialinecolor}
\draw (39.000000\du,75.777200\du)--(39.000000\du,78.625300\du);
}
\pgfsetlinewidth{0.100000\du}
\pgfsetdash{}{0pt}
\pgfsetdash{}{0pt}
\pgfsetbuttcap
{
\definecolor{dialinecolor}{rgb}{0.000000, 0.000000, 0.000000}
\pgfsetfillcolor{dialinecolor}
% was here!!!
\definecolor{dialinecolor}{rgb}{0.000000, 0.000000, 0.000000}
\pgfsetstrokecolor{dialinecolor}
\draw (39.000000\du,78.375300\du)--(39.000000\du,81.223400\du);
}
\definecolor{dialinecolor}{rgb}{0.000000, 0.000000, 0.000000}
\pgfsetstrokecolor{dialinecolor}
\draw (39.000000\du,78.375300\du)--(39.000000\du,81.223400\du);
\pgfsetlinewidth{0.100000\du}
\pgfsetdash{}{0pt}
\pgfsetmiterjoin
\pgfsetbuttcap
\definecolor{dialinecolor}{rgb}{0.000000, 0.000000, 0.000000}
\pgfsetfillcolor{dialinecolor}
\pgfpathmoveto{\pgfpoint{39.000000\du}{81.223400\du}}
\pgfpathcurveto{\pgfpoint{38.875000\du}{81.223400\du}}{\pgfpoint{38.750000\du}{81.098400\du}}{\pgfpoint{38.750000\du}{80.973400\du}}
\pgfpathcurveto{\pgfpoint{38.750000\du}{80.848400\du}}{\pgfpoint{38.875000\du}{80.723400\du}}{\pgfpoint{39.000000\du}{80.723400\du}}
\pgfpathcurveto{\pgfpoint{39.125000\du}{80.723400\du}}{\pgfpoint{39.250000\du}{80.848400\du}}{\pgfpoint{39.250000\du}{80.973400\du}}
\pgfpathcurveto{\pgfpoint{39.250000\du}{81.098400\du}}{\pgfpoint{39.125000\du}{81.223400\du}}{\pgfpoint{39.000000\du}{81.223400\du}}
\pgfusepath{fill}
\definecolor{dialinecolor}{rgb}{0.000000, 0.000000, 0.000000}
\pgfsetstrokecolor{dialinecolor}
\pgfpathmoveto{\pgfpoint{39.000000\du}{81.223400\du}}
\pgfpathcurveto{\pgfpoint{38.875000\du}{81.223400\du}}{\pgfpoint{38.750000\du}{81.098400\du}}{\pgfpoint{38.750000\du}{80.973400\du}}
\pgfpathcurveto{\pgfpoint{38.750000\du}{80.848400\du}}{\pgfpoint{38.875000\du}{80.723400\du}}{\pgfpoint{39.000000\du}{80.723400\du}}
\pgfpathcurveto{\pgfpoint{39.125000\du}{80.723400\du}}{\pgfpoint{39.250000\du}{80.848400\du}}{\pgfpoint{39.250000\du}{80.973400\du}}
\pgfpathcurveto{\pgfpoint{39.250000\du}{81.098400\du}}{\pgfpoint{39.125000\du}{81.223400\du}}{\pgfpoint{39.000000\du}{81.223400\du}}
\pgfusepath{stroke}
\pgfsetlinewidth{0.100000\du}
\pgfsetdash{}{0pt}
\pgfsetdash{}{0pt}
\pgfsetbuttcap
{
\definecolor{dialinecolor}{rgb}{0.000000, 0.000000, 0.000000}
\pgfsetfillcolor{dialinecolor}
% was here!!!
\pgfsetarrowsend{latex}
\definecolor{dialinecolor}{rgb}{0.000000, 0.000000, 0.000000}
\pgfsetstrokecolor{dialinecolor}
\draw (34.500000\du,75.777200\du)--(37.250000\du,75.777200\du);
}
\pgfsetlinewidth{0.100000\du}
\pgfsetdash{}{0pt}
\pgfsetdash{}{0pt}
\pgfsetbuttcap
{
\definecolor{dialinecolor}{rgb}{0.000000, 0.000000, 0.000000}
\pgfsetfillcolor{dialinecolor}
% was here!!!
\definecolor{dialinecolor}{rgb}{0.000000, 0.000000, 0.000000}
\pgfsetstrokecolor{dialinecolor}
\draw (37.000000\du,75.777200\du)--(39.250000\du,75.777200\du);
}
\definecolor{dialinecolor}{rgb}{0.000000, 0.000000, 0.000000}
\pgfsetstrokecolor{dialinecolor}
\draw (37.000000\du,75.777200\du)--(39.250000\du,75.777200\du);
\pgfsetlinewidth{0.100000\du}
\pgfsetdash{}{0pt}
\pgfsetmiterjoin
\pgfsetbuttcap
\definecolor{dialinecolor}{rgb}{0.000000, 0.000000, 0.000000}
\pgfsetfillcolor{dialinecolor}
\pgfpathmoveto{\pgfpoint{39.250000\du}{75.777200\du}}
\pgfpathcurveto{\pgfpoint{39.250000\du}{75.902200\du}}{\pgfpoint{39.125000\du}{76.027200\du}}{\pgfpoint{39.000000\du}{76.027200\du}}
\pgfpathcurveto{\pgfpoint{38.875000\du}{76.027200\du}}{\pgfpoint{38.750000\du}{75.902200\du}}{\pgfpoint{38.750000\du}{75.777200\du}}
\pgfpathcurveto{\pgfpoint{38.750000\du}{75.652200\du}}{\pgfpoint{38.875000\du}{75.527200\du}}{\pgfpoint{39.000000\du}{75.527200\du}}
\pgfpathcurveto{\pgfpoint{39.125000\du}{75.527200\du}}{\pgfpoint{39.250000\du}{75.652200\du}}{\pgfpoint{39.250000\du}{75.777200\du}}
\pgfusepath{fill}
\definecolor{dialinecolor}{rgb}{0.000000, 0.000000, 0.000000}
\pgfsetstrokecolor{dialinecolor}
\pgfpathmoveto{\pgfpoint{39.250000\du}{75.777200\du}}
\pgfpathcurveto{\pgfpoint{39.250000\du}{75.902200\du}}{\pgfpoint{39.125000\du}{76.027200\du}}{\pgfpoint{39.000000\du}{76.027200\du}}
\pgfpathcurveto{\pgfpoint{38.875000\du}{76.027200\du}}{\pgfpoint{38.750000\du}{75.902200\du}}{\pgfpoint{38.750000\du}{75.777200\du}}
\pgfpathcurveto{\pgfpoint{38.750000\du}{75.652200\du}}{\pgfpoint{38.875000\du}{75.527200\du}}{\pgfpoint{39.000000\du}{75.527200\du}}
\pgfpathcurveto{\pgfpoint{39.125000\du}{75.527200\du}}{\pgfpoint{39.250000\du}{75.652200\du}}{\pgfpoint{39.250000\du}{75.777200\du}}
\pgfusepath{stroke}
\pgfsetlinewidth{0.100000\du}
\pgfsetdash{}{0pt}
\pgfsetdash{}{0pt}
\pgfsetbuttcap
{
\definecolor{dialinecolor}{rgb}{0.000000, 0.000000, 0.000000}
\pgfsetfillcolor{dialinecolor}
% was here!!!
\pgfsetarrowsend{latex}
\definecolor{dialinecolor}{rgb}{0.000000, 0.000000, 0.000000}
\pgfsetstrokecolor{dialinecolor}
\draw (34.625003\du,75.993705\du)--(33.124997\du,73.395695\du);
}
\definecolor{dialinecolor}{rgb}{0.000000, 0.000000, 0.000000}
\pgfsetstrokecolor{dialinecolor}
\draw (34.625003\du,75.993705\du)--(33.124997\du,73.395695\du);
\pgfsetlinewidth{0.100000\du}
\pgfsetdash{}{0pt}
\pgfsetmiterjoin
\pgfsetbuttcap
\definecolor{dialinecolor}{rgb}{0.000000, 0.000000, 0.000000}
\pgfsetfillcolor{dialinecolor}
\pgfpathmoveto{\pgfpoint{34.625003\du}{75.993705\du}}
\pgfpathcurveto{\pgfpoint{34.516750\du}{76.056206\du}}{\pgfpoint{34.345997\du}{76.010455\du}}{\pgfpoint{34.283495\du}{75.902203\du}}
\pgfpathcurveto{\pgfpoint{34.220994\du}{75.793950\du}}{\pgfpoint{34.266745\du}{75.623197\du}}{\pgfpoint{34.374997\du}{75.560695\du}}
\pgfpathcurveto{\pgfpoint{34.483250\du}{75.498194\du}}{\pgfpoint{34.654003\du}{75.543945\du}}{\pgfpoint{34.716505\du}{75.652197\du}}
\pgfpathcurveto{\pgfpoint{34.779006\du}{75.760450\du}}{\pgfpoint{34.733255\du}{75.931203\du}}{\pgfpoint{34.625003\du}{75.993705\du}}
\pgfusepath{fill}
\definecolor{dialinecolor}{rgb}{0.000000, 0.000000, 0.000000}
\pgfsetstrokecolor{dialinecolor}
\pgfpathmoveto{\pgfpoint{34.625003\du}{75.993705\du}}
\pgfpathcurveto{\pgfpoint{34.516750\du}{76.056206\du}}{\pgfpoint{34.345997\du}{76.010455\du}}{\pgfpoint{34.283495\du}{75.902203\du}}
\pgfpathcurveto{\pgfpoint{34.220994\du}{75.793950\du}}{\pgfpoint{34.266745\du}{75.623197\du}}{\pgfpoint{34.374997\du}{75.560695\du}}
\pgfpathcurveto{\pgfpoint{34.483250\du}{75.498194\du}}{\pgfpoint{34.654003\du}{75.543945\du}}{\pgfpoint{34.716505\du}{75.652197\du}}
\pgfpathcurveto{\pgfpoint{34.779006\du}{75.760450\du}}{\pgfpoint{34.733255\du}{75.931203\du}}{\pgfpoint{34.625003\du}{75.993705\du}}
\pgfusepath{stroke}
\pgfsetlinewidth{0.100000\du}
\pgfsetdash{}{0pt}
\pgfsetdash{}{0pt}
\pgfsetbuttcap
{
\definecolor{dialinecolor}{rgb}{0.000000, 0.000000, 0.000000}
\pgfsetfillcolor{dialinecolor}
% was here!!!
\definecolor{dialinecolor}{rgb}{0.000000, 0.000000, 0.000000}
\pgfsetstrokecolor{dialinecolor}
\draw (32.250000\du,71.880100\du)--(33.250000\du,73.612200\du);
}
\pgfsetlinewidth{0.100000\du}
\pgfsetdash{}{0pt}
\pgfsetdash{}{0pt}
\pgfsetbuttcap
{
\definecolor{dialinecolor}{rgb}{0.000000, 0.000000, 0.000000}
\pgfsetfillcolor{dialinecolor}
% was here!!!
\pgfsetarrowsend{latex}
\definecolor{dialinecolor}{rgb}{0.000000, 0.000000, 0.000000}
\pgfsetstrokecolor{dialinecolor}
\draw (34.500000\du,81.223400\du)--(34.500000\du,78.125300\du);
}
\definecolor{dialinecolor}{rgb}{0.000000, 0.000000, 0.000000}
\pgfsetstrokecolor{dialinecolor}
\draw (34.500000\du,81.223400\du)--(34.500000\du,78.125300\du);
\pgfsetlinewidth{0.100000\du}
\pgfsetdash{}{0pt}
\pgfsetmiterjoin
\pgfsetbuttcap
\definecolor{dialinecolor}{rgb}{0.000000, 0.000000, 0.000000}
\pgfsetfillcolor{dialinecolor}
\pgfpathmoveto{\pgfpoint{34.500000\du}{81.223400\du}}
\pgfpathcurveto{\pgfpoint{34.375000\du}{81.223400\du}}{\pgfpoint{34.250000\du}{81.098400\du}}{\pgfpoint{34.250000\du}{80.973400\du}}
\pgfpathcurveto{\pgfpoint{34.250000\du}{80.848400\du}}{\pgfpoint{34.375000\du}{80.723400\du}}{\pgfpoint{34.500000\du}{80.723400\du}}
\pgfpathcurveto{\pgfpoint{34.625000\du}{80.723400\du}}{\pgfpoint{34.750000\du}{80.848400\du}}{\pgfpoint{34.750000\du}{80.973400\du}}
\pgfpathcurveto{\pgfpoint{34.750000\du}{81.098400\du}}{\pgfpoint{34.625000\du}{81.223400\du}}{\pgfpoint{34.500000\du}{81.223400\du}}
\pgfusepath{fill}
\definecolor{dialinecolor}{rgb}{0.000000, 0.000000, 0.000000}
\pgfsetstrokecolor{dialinecolor}
\pgfpathmoveto{\pgfpoint{34.500000\du}{81.223400\du}}
\pgfpathcurveto{\pgfpoint{34.375000\du}{81.223400\du}}{\pgfpoint{34.250000\du}{81.098400\du}}{\pgfpoint{34.250000\du}{80.973400\du}}
\pgfpathcurveto{\pgfpoint{34.250000\du}{80.848400\du}}{\pgfpoint{34.375000\du}{80.723400\du}}{\pgfpoint{34.500000\du}{80.723400\du}}
\pgfpathcurveto{\pgfpoint{34.625000\du}{80.723400\du}}{\pgfpoint{34.750000\du}{80.848400\du}}{\pgfpoint{34.750000\du}{80.973400\du}}
\pgfpathcurveto{\pgfpoint{34.750000\du}{81.098400\du}}{\pgfpoint{34.625000\du}{81.223400\du}}{\pgfpoint{34.500000\du}{81.223400\du}}
\pgfusepath{stroke}
\pgfsetlinewidth{0.100000\du}
\pgfsetdash{}{0pt}
\pgfsetdash{}{0pt}
\pgfsetbuttcap
{
\definecolor{dialinecolor}{rgb}{0.000000, 0.000000, 0.000000}
\pgfsetfillcolor{dialinecolor}
% was here!!!
\definecolor{dialinecolor}{rgb}{0.000000, 0.000000, 0.000000}
\pgfsetstrokecolor{dialinecolor}
\draw (34.500000\du,75.777200\du)--(34.500000\du,78.375300\du);
}
\pgfsetlinewidth{0.100000\du}
\pgfsetdash{}{0pt}
\pgfsetdash{}{0pt}
\pgfsetbuttcap
{
\definecolor{dialinecolor}{rgb}{0.000000, 0.000000, 0.000000}
\pgfsetfillcolor{dialinecolor}
% was here!!!
\pgfsetarrowsend{latex}
\definecolor{dialinecolor}{rgb}{0.000000, 0.000000, 0.000000}
\pgfsetstrokecolor{dialinecolor}
\draw (34.500000\du,80.973400\du)--(37.250000\du,80.973400\du);
}
\pgfsetlinewidth{0.100000\du}
\pgfsetdash{}{0pt}
\pgfsetdash{}{0pt}
\pgfsetbuttcap
{
\definecolor{dialinecolor}{rgb}{0.000000, 0.000000, 0.000000}
\pgfsetfillcolor{dialinecolor}
% was here!!!
\definecolor{dialinecolor}{rgb}{0.000000, 0.000000, 0.000000}
\pgfsetstrokecolor{dialinecolor}
\draw (39.000000\du,80.973400\du)--(37.000000\du,80.973400\du);
}
\pgfsetlinewidth{0.100000\du}
\pgfsetdash{}{0pt}
\pgfsetdash{}{0pt}
\pgfsetbuttcap
{
\definecolor{dialinecolor}{rgb}{0.000000, 0.000000, 0.000000}
\pgfsetfillcolor{dialinecolor}
% was here!!!
\pgfsetarrowsend{latex}
\definecolor{dialinecolor}{rgb}{0.000000, 0.000000, 0.000000}
\pgfsetstrokecolor{dialinecolor}
\draw (39.000000\du,80.973400\du)--(39.000000\du,83.821500\du);
}
\pgfsetlinewidth{0.100000\du}
\pgfsetdash{}{0pt}
\pgfsetdash{}{0pt}
\pgfsetbuttcap
{
\definecolor{dialinecolor}{rgb}{0.000000, 0.000000, 0.000000}
\pgfsetfillcolor{dialinecolor}
% was here!!!
\definecolor{dialinecolor}{rgb}{0.000000, 0.000000, 0.000000}
\pgfsetstrokecolor{dialinecolor}
\draw (39.000000\du,83.571500\du)--(39.000000\du,86.419500\du);
}
\definecolor{dialinecolor}{rgb}{0.000000, 0.000000, 0.000000}
\pgfsetstrokecolor{dialinecolor}
\draw (39.000000\du,83.571500\du)--(39.000000\du,86.419500\du);
\pgfsetlinewidth{0.100000\du}
\pgfsetdash{}{0pt}
\pgfsetmiterjoin
\pgfsetbuttcap
\definecolor{dialinecolor}{rgb}{0.000000, 0.000000, 0.000000}
\pgfsetfillcolor{dialinecolor}
\pgfpathmoveto{\pgfpoint{39.000000\du}{86.419500\du}}
\pgfpathcurveto{\pgfpoint{38.875000\du}{86.419500\du}}{\pgfpoint{38.750000\du}{86.294500\du}}{\pgfpoint{38.750000\du}{86.169500\du}}
\pgfpathcurveto{\pgfpoint{38.750000\du}{86.044500\du}}{\pgfpoint{38.875000\du}{85.919500\du}}{\pgfpoint{39.000000\du}{85.919500\du}}
\pgfpathcurveto{\pgfpoint{39.125000\du}{85.919500\du}}{\pgfpoint{39.250000\du}{86.044500\du}}{\pgfpoint{39.250000\du}{86.169500\du}}
\pgfpathcurveto{\pgfpoint{39.250000\du}{86.294500\du}}{\pgfpoint{39.125000\du}{86.419500\du}}{\pgfpoint{39.000000\du}{86.419500\du}}
\pgfusepath{fill}
\definecolor{dialinecolor}{rgb}{0.000000, 0.000000, 0.000000}
\pgfsetstrokecolor{dialinecolor}
\pgfpathmoveto{\pgfpoint{39.000000\du}{86.419500\du}}
\pgfpathcurveto{\pgfpoint{38.875000\du}{86.419500\du}}{\pgfpoint{38.750000\du}{86.294500\du}}{\pgfpoint{38.750000\du}{86.169500\du}}
\pgfpathcurveto{\pgfpoint{38.750000\du}{86.044500\du}}{\pgfpoint{38.875000\du}{85.919500\du}}{\pgfpoint{39.000000\du}{85.919500\du}}
\pgfpathcurveto{\pgfpoint{39.125000\du}{85.919500\du}}{\pgfpoint{39.250000\du}{86.044500\du}}{\pgfpoint{39.250000\du}{86.169500\du}}
\pgfpathcurveto{\pgfpoint{39.250000\du}{86.294500\du}}{\pgfpoint{39.125000\du}{86.419500\du}}{\pgfpoint{39.000000\du}{86.419500\du}}
\pgfusepath{stroke}
\pgfsetlinewidth{0.100000\du}
\pgfsetdash{}{0pt}
\pgfsetdash{}{0pt}
\pgfsetbuttcap
{
\definecolor{dialinecolor}{rgb}{0.000000, 0.000000, 0.000000}
\pgfsetfillcolor{dialinecolor}
% was here!!!
\pgfsetarrowsend{latex}
\definecolor{dialinecolor}{rgb}{0.000000, 0.000000, 0.000000}
\pgfsetstrokecolor{dialinecolor}
\draw (34.500000\du,86.169500\du)--(37.250000\du,86.169500\du);
}
\pgfsetlinewidth{0.100000\du}
\pgfsetdash{}{0pt}
\pgfsetdash{}{0pt}
\pgfsetbuttcap
{
\definecolor{dialinecolor}{rgb}{0.000000, 0.000000, 0.000000}
\pgfsetfillcolor{dialinecolor}
% was here!!!
\definecolor{dialinecolor}{rgb}{0.000000, 0.000000, 0.000000}
\pgfsetstrokecolor{dialinecolor}
\draw (39.000000\du,86.169500\du)--(37.000000\du,86.169500\du);
}
\pgfsetlinewidth{0.100000\du}
\pgfsetdash{}{0pt}
\pgfsetdash{}{0pt}
\pgfsetbuttcap
{
\definecolor{dialinecolor}{rgb}{0.000000, 0.000000, 0.000000}
\pgfsetfillcolor{dialinecolor}
% was here!!!
\pgfsetarrowsend{latex}
\definecolor{dialinecolor}{rgb}{0.000000, 0.000000, 0.000000}
\pgfsetstrokecolor{dialinecolor}
\draw (34.500000\du,86.419500\du)--(34.500000\du,83.321500\du);
}
\definecolor{dialinecolor}{rgb}{0.000000, 0.000000, 0.000000}
\pgfsetstrokecolor{dialinecolor}
\draw (34.500000\du,86.419500\du)--(34.500000\du,83.321500\du);
\pgfsetlinewidth{0.100000\du}
\pgfsetdash{}{0pt}
\pgfsetmiterjoin
\pgfsetbuttcap
\definecolor{dialinecolor}{rgb}{0.000000, 0.000000, 0.000000}
\pgfsetfillcolor{dialinecolor}
\pgfpathmoveto{\pgfpoint{34.500000\du}{86.419500\du}}
\pgfpathcurveto{\pgfpoint{34.375000\du}{86.419500\du}}{\pgfpoint{34.250000\du}{86.294500\du}}{\pgfpoint{34.250000\du}{86.169500\du}}
\pgfpathcurveto{\pgfpoint{34.250000\du}{86.044500\du}}{\pgfpoint{34.375000\du}{85.919500\du}}{\pgfpoint{34.500000\du}{85.919500\du}}
\pgfpathcurveto{\pgfpoint{34.625000\du}{85.919500\du}}{\pgfpoint{34.750000\du}{86.044500\du}}{\pgfpoint{34.750000\du}{86.169500\du}}
\pgfpathcurveto{\pgfpoint{34.750000\du}{86.294500\du}}{\pgfpoint{34.625000\du}{86.419500\du}}{\pgfpoint{34.500000\du}{86.419500\du}}
\pgfusepath{fill}
\definecolor{dialinecolor}{rgb}{0.000000, 0.000000, 0.000000}
\pgfsetstrokecolor{dialinecolor}
\pgfpathmoveto{\pgfpoint{34.500000\du}{86.419500\du}}
\pgfpathcurveto{\pgfpoint{34.375000\du}{86.419500\du}}{\pgfpoint{34.250000\du}{86.294500\du}}{\pgfpoint{34.250000\du}{86.169500\du}}
\pgfpathcurveto{\pgfpoint{34.250000\du}{86.044500\du}}{\pgfpoint{34.375000\du}{85.919500\du}}{\pgfpoint{34.500000\du}{85.919500\du}}
\pgfpathcurveto{\pgfpoint{34.625000\du}{85.919500\du}}{\pgfpoint{34.750000\du}{86.044500\du}}{\pgfpoint{34.750000\du}{86.169500\du}}
\pgfpathcurveto{\pgfpoint{34.750000\du}{86.294500\du}}{\pgfpoint{34.625000\du}{86.419500\du}}{\pgfpoint{34.500000\du}{86.419500\du}}
\pgfusepath{stroke}
\pgfsetlinewidth{0.100000\du}
\pgfsetdash{}{0pt}
\pgfsetdash{}{0pt}
\pgfsetbuttcap
{
\definecolor{dialinecolor}{rgb}{0.000000, 0.000000, 0.000000}
\pgfsetfillcolor{dialinecolor}
% was here!!!
\definecolor{dialinecolor}{rgb}{0.000000, 0.000000, 0.000000}
\pgfsetstrokecolor{dialinecolor}
\draw (34.500000\du,80.973400\du)--(34.500000\du,83.571500\du);
}
\pgfsetlinewidth{0.100000\du}
\pgfsetdash{}{0pt}
\pgfsetdash{}{0pt}
\pgfsetbuttcap
{
\definecolor{dialinecolor}{rgb}{0.000000, 0.000000, 0.000000}
\pgfsetfillcolor{dialinecolor}
% was here!!!
\pgfsetarrowsend{latex}
\definecolor{dialinecolor}{rgb}{0.000000, 0.000000, 0.000000}
\pgfsetstrokecolor{dialinecolor}
\draw (34.500000\du,91.615700\du)--(34.500000\du,88.517600\du);
}
\definecolor{dialinecolor}{rgb}{0.000000, 0.000000, 0.000000}
\pgfsetstrokecolor{dialinecolor}
\draw (34.500000\du,91.065700\du)--(34.500000\du,88.517600\du);
\pgfsetlinewidth{0.100000\du}
\pgfsetdash{}{0pt}
\pgfsetmiterjoin
\pgfsetbuttcap
\definecolor{dialinecolor}{rgb}{1.000000, 1.000000, 1.000000}
\pgfsetfillcolor{dialinecolor}
\pgfpathmoveto{\pgfpoint{34.500000\du}{91.565700\du}}
\pgfpathcurveto{\pgfpoint{34.375000\du}{91.565700\du}}{\pgfpoint{34.250000\du}{91.440700\du}}{\pgfpoint{34.250000\du}{91.315700\du}}
\pgfpathcurveto{\pgfpoint{34.250000\du}{91.190700\du}}{\pgfpoint{34.375000\du}{91.065700\du}}{\pgfpoint{34.500000\du}{91.065700\du}}
\pgfpathcurveto{\pgfpoint{34.625000\du}{91.065700\du}}{\pgfpoint{34.750000\du}{91.190700\du}}{\pgfpoint{34.750000\du}{91.315700\du}}
\pgfpathcurveto{\pgfpoint{34.750000\du}{91.440700\du}}{\pgfpoint{34.625000\du}{91.565700\du}}{\pgfpoint{34.500000\du}{91.565700\du}}
\pgfusepath{fill}
\definecolor{dialinecolor}{rgb}{0.000000, 0.000000, 0.000000}
\pgfsetstrokecolor{dialinecolor}
\pgfpathmoveto{\pgfpoint{34.500000\du}{91.565700\du}}
\pgfpathcurveto{\pgfpoint{34.375000\du}{91.565700\du}}{\pgfpoint{34.250000\du}{91.440700\du}}{\pgfpoint{34.250000\du}{91.315700\du}}
\pgfpathcurveto{\pgfpoint{34.250000\du}{91.190700\du}}{\pgfpoint{34.375000\du}{91.065700\du}}{\pgfpoint{34.500000\du}{91.065700\du}}
\pgfpathcurveto{\pgfpoint{34.625000\du}{91.065700\du}}{\pgfpoint{34.750000\du}{91.190700\du}}{\pgfpoint{34.750000\du}{91.315700\du}}
\pgfpathcurveto{\pgfpoint{34.750000\du}{91.440700\du}}{\pgfpoint{34.625000\du}{91.565700\du}}{\pgfpoint{34.500000\du}{91.565700\du}}
\pgfusepath{stroke}
\pgfsetlinewidth{0.100000\du}
\pgfsetdash{}{0pt}
\pgfsetdash{}{0pt}
\pgfsetbuttcap
{
\definecolor{dialinecolor}{rgb}{0.000000, 0.000000, 0.000000}
\pgfsetfillcolor{dialinecolor}
% was here!!!
\definecolor{dialinecolor}{rgb}{0.000000, 0.000000, 0.000000}
\pgfsetstrokecolor{dialinecolor}
\draw (34.500000\du,86.169500\du)--(34.500000\du,88.767600\du);
}
\pgfsetlinewidth{0.100000\du}
\pgfsetdash{}{0pt}
\pgfsetdash{}{0pt}
\pgfsetbuttcap
{
\definecolor{dialinecolor}{rgb}{0.000000, 0.000000, 0.000000}
\pgfsetfillcolor{dialinecolor}
% was here!!!
\definecolor{dialinecolor}{rgb}{0.000000, 0.000000, 0.000000}
\pgfsetstrokecolor{dialinecolor}
\draw (39.000000\du,91.615700\du)--(39.000000\du,88.767600\du);
}
\definecolor{dialinecolor}{rgb}{0.000000, 0.000000, 0.000000}
\pgfsetstrokecolor{dialinecolor}
\draw (39.000000\du,91.065700\du)--(39.000000\du,88.767600\du);
\pgfsetlinewidth{0.100000\du}
\pgfsetdash{}{0pt}
\pgfsetmiterjoin
\pgfsetbuttcap
\definecolor{dialinecolor}{rgb}{1.000000, 1.000000, 1.000000}
\pgfsetfillcolor{dialinecolor}
\pgfpathmoveto{\pgfpoint{39.000000\du}{91.565700\du}}
\pgfpathcurveto{\pgfpoint{38.875000\du}{91.565700\du}}{\pgfpoint{38.750000\du}{91.440700\du}}{\pgfpoint{38.750000\du}{91.315700\du}}
\pgfpathcurveto{\pgfpoint{38.750000\du}{91.190700\du}}{\pgfpoint{38.875000\du}{91.065700\du}}{\pgfpoint{39.000000\du}{91.065700\du}}
\pgfpathcurveto{\pgfpoint{39.125000\du}{91.065700\du}}{\pgfpoint{39.250000\du}{91.190700\du}}{\pgfpoint{39.250000\du}{91.315700\du}}
\pgfpathcurveto{\pgfpoint{39.250000\du}{91.440700\du}}{\pgfpoint{39.125000\du}{91.565700\du}}{\pgfpoint{39.000000\du}{91.565700\du}}
\pgfusepath{fill}
\definecolor{dialinecolor}{rgb}{0.000000, 0.000000, 0.000000}
\pgfsetstrokecolor{dialinecolor}
\pgfpathmoveto{\pgfpoint{39.000000\du}{91.565700\du}}
\pgfpathcurveto{\pgfpoint{38.875000\du}{91.565700\du}}{\pgfpoint{38.750000\du}{91.440700\du}}{\pgfpoint{38.750000\du}{91.315700\du}}
\pgfpathcurveto{\pgfpoint{38.750000\du}{91.190700\du}}{\pgfpoint{38.875000\du}{91.065700\du}}{\pgfpoint{39.000000\du}{91.065700\du}}
\pgfpathcurveto{\pgfpoint{39.125000\du}{91.065700\du}}{\pgfpoint{39.250000\du}{91.190700\du}}{\pgfpoint{39.250000\du}{91.315700\du}}
\pgfpathcurveto{\pgfpoint{39.250000\du}{91.440700\du}}{\pgfpoint{39.125000\du}{91.565700\du}}{\pgfpoint{39.000000\du}{91.565700\du}}
\pgfusepath{stroke}
\pgfsetlinewidth{0.100000\du}
\pgfsetdash{}{0pt}
\pgfsetdash{}{0pt}
\pgfsetbuttcap
{
\definecolor{dialinecolor}{rgb}{0.000000, 0.000000, 0.000000}
\pgfsetfillcolor{dialinecolor}
% was here!!!
\pgfsetarrowsend{latex}
\definecolor{dialinecolor}{rgb}{0.000000, 0.000000, 0.000000}
\pgfsetstrokecolor{dialinecolor}
\draw (39.000000\du,86.169500\du)--(39.000000\du,89.017600\du);
}
\pgfsetlinewidth{0.100000\du}
\pgfsetdash{}{0pt}
\pgfsetdash{}{0pt}
\pgfsetbuttcap
{
\definecolor{dialinecolor}{rgb}{0.000000, 0.000000, 0.000000}
\pgfsetfillcolor{dialinecolor}
% was here!!!
\pgfsetarrowsend{latex}
\definecolor{dialinecolor}{rgb}{0.000000, 0.000000, 0.000000}
\pgfsetstrokecolor{dialinecolor}
\draw (34.500000\du,67.983000\du)--(37.625001\du,73.395606\du);
}
\pgfsetlinewidth{0.100000\du}
\pgfsetdash{}{0pt}
\pgfsetdash{}{0pt}
\pgfsetbuttcap
{
\definecolor{dialinecolor}{rgb}{0.000000, 0.000000, 0.000000}
\pgfsetfillcolor{dialinecolor}
% was here!!!
\definecolor{dialinecolor}{rgb}{0.000000, 0.000000, 0.000000}
\pgfsetstrokecolor{dialinecolor}
\draw (37.500000\du,73.179100\du)--(39.124999\du,75.993707\du);
}
\definecolor{dialinecolor}{rgb}{0.000000, 0.000000, 0.000000}
\pgfsetstrokecolor{dialinecolor}
\draw (37.500000\du,73.179100\du)--(39.124999\du,75.993707\du);
\pgfsetlinewidth{0.100000\du}
\pgfsetdash{}{0pt}
\pgfsetmiterjoin
\pgfsetbuttcap
\definecolor{dialinecolor}{rgb}{0.000000, 0.000000, 0.000000}
\pgfsetfillcolor{dialinecolor}
\pgfpathmoveto{\pgfpoint{39.124999\du}{75.993707\du}}
\pgfpathcurveto{\pgfpoint{39.016746\du}{76.056206\du}}{\pgfpoint{38.845993\du}{76.010453\du}}{\pgfpoint{38.783493\du}{75.902199\du}}
\pgfpathcurveto{\pgfpoint{38.720994\du}{75.793946\du}}{\pgfpoint{38.766747\du}{75.623193\du}}{\pgfpoint{38.875001\du}{75.560693\du}}
\pgfpathcurveto{\pgfpoint{38.983254\du}{75.498194\du}}{\pgfpoint{39.154007\du}{75.543947\du}}{\pgfpoint{39.216507\du}{75.652201\du}}
\pgfpathcurveto{\pgfpoint{39.279006\du}{75.760454\du}}{\pgfpoint{39.233253\du}{75.931207\du}}{\pgfpoint{39.124999\du}{75.993707\du}}
\pgfusepath{fill}
\definecolor{dialinecolor}{rgb}{0.000000, 0.000000, 0.000000}
\pgfsetstrokecolor{dialinecolor}
\pgfpathmoveto{\pgfpoint{39.124999\du}{75.993707\du}}
\pgfpathcurveto{\pgfpoint{39.016746\du}{76.056206\du}}{\pgfpoint{38.845993\du}{76.010453\du}}{\pgfpoint{38.783493\du}{75.902199\du}}
\pgfpathcurveto{\pgfpoint{38.720994\du}{75.793946\du}}{\pgfpoint{38.766747\du}{75.623193\du}}{\pgfpoint{38.875001\du}{75.560693\du}}
\pgfpathcurveto{\pgfpoint{38.983254\du}{75.498194\du}}{\pgfpoint{39.154007\du}{75.543947\du}}{\pgfpoint{39.216507\du}{75.652201\du}}
\pgfpathcurveto{\pgfpoint{39.279006\du}{75.760454\du}}{\pgfpoint{39.233253\du}{75.931207\du}}{\pgfpoint{39.124999\du}{75.993707\du}}
\pgfusepath{stroke}
\pgfsetlinewidth{0.100000\du}
\pgfsetdash{}{0pt}
\pgfsetdash{}{0pt}
\pgfsetbuttcap
{
\definecolor{dialinecolor}{rgb}{0.000000, 0.000000, 0.000000}
\pgfsetfillcolor{dialinecolor}
% was here!!!
\pgfsetarrowsend{latex}
\definecolor{dialinecolor}{rgb}{0.000000, 0.000000, 0.000000}
\pgfsetstrokecolor{dialinecolor}
\draw (41.250000\du,71.880100\du)--(35.000000\du,71.880100\du);
}
\pgfsetlinewidth{0.100000\du}
\pgfsetdash{}{0pt}
\pgfsetdash{}{0pt}
\pgfsetbuttcap
{
\definecolor{dialinecolor}{rgb}{0.000000, 0.000000, 0.000000}
\pgfsetfillcolor{dialinecolor}
% was here!!!
\definecolor{dialinecolor}{rgb}{0.000000, 0.000000, 0.000000}
\pgfsetstrokecolor{dialinecolor}
\draw (35.250000\du,71.880100\du)--(32.000000\du,71.880100\du);
}
\definecolor{dialinecolor}{rgb}{0.000000, 0.000000, 0.000000}
\pgfsetstrokecolor{dialinecolor}
\draw (35.250000\du,71.880100\du)--(32.000000\du,71.880100\du);
\pgfsetlinewidth{0.100000\du}
\pgfsetdash{}{0pt}
\pgfsetmiterjoin
\pgfsetbuttcap
\definecolor{dialinecolor}{rgb}{0.000000, 0.000000, 0.000000}
\pgfsetfillcolor{dialinecolor}
\pgfpathmoveto{\pgfpoint{32.000000\du}{71.880100\du}}
\pgfpathcurveto{\pgfpoint{32.000000\du}{71.755100\du}}{\pgfpoint{32.125000\du}{71.630100\du}}{\pgfpoint{32.250000\du}{71.630100\du}}
\pgfpathcurveto{\pgfpoint{32.375000\du}{71.630100\du}}{\pgfpoint{32.500000\du}{71.755100\du}}{\pgfpoint{32.500000\du}{71.880100\du}}
\pgfpathcurveto{\pgfpoint{32.500000\du}{72.005100\du}}{\pgfpoint{32.375000\du}{72.130100\du}}{\pgfpoint{32.250000\du}{72.130100\du}}
\pgfpathcurveto{\pgfpoint{32.125000\du}{72.130100\du}}{\pgfpoint{32.000000\du}{72.005100\du}}{\pgfpoint{32.000000\du}{71.880100\du}}
\pgfusepath{fill}
\definecolor{dialinecolor}{rgb}{0.000000, 0.000000, 0.000000}
\pgfsetstrokecolor{dialinecolor}
\pgfpathmoveto{\pgfpoint{32.000000\du}{71.880100\du}}
\pgfpathcurveto{\pgfpoint{32.000000\du}{71.755100\du}}{\pgfpoint{32.125000\du}{71.630100\du}}{\pgfpoint{32.250000\du}{71.630100\du}}
\pgfpathcurveto{\pgfpoint{32.375000\du}{71.630100\du}}{\pgfpoint{32.500000\du}{71.755100\du}}{\pgfpoint{32.500000\du}{71.880100\du}}
\pgfpathcurveto{\pgfpoint{32.500000\du}{72.005100\du}}{\pgfpoint{32.375000\du}{72.130100\du}}{\pgfpoint{32.250000\du}{72.130100\du}}
\pgfpathcurveto{\pgfpoint{32.125000\du}{72.130100\du}}{\pgfpoint{32.000000\du}{72.005100\du}}{\pgfpoint{32.000000\du}{71.880100\du}}
\pgfusepath{stroke}
\pgfsetlinewidth{0.100000\du}
\pgfsetdash{}{0pt}
\pgfsetdash{}{0pt}
\pgfsetbuttcap
{
\definecolor{dialinecolor}{rgb}{0.000000, 0.000000, 0.000000}
\pgfsetfillcolor{dialinecolor}
% was here!!!
\pgfsetarrowsend{latex}
\definecolor{dialinecolor}{rgb}{0.000000, 0.000000, 0.000000}
\pgfsetstrokecolor{dialinecolor}
\draw (34.500000\du,75.777200\du)--(37.625001\du,70.364594\du);
}
\pgfsetlinewidth{0.100000\du}
\pgfsetdash{}{0pt}
\pgfsetdash{}{0pt}
\pgfsetbuttcap
{
\definecolor{dialinecolor}{rgb}{0.000000, 0.000000, 0.000000}
\pgfsetfillcolor{dialinecolor}
% was here!!!
\definecolor{dialinecolor}{rgb}{0.000000, 0.000000, 0.000000}
\pgfsetstrokecolor{dialinecolor}
\draw (37.500000\du,70.581100\du)--(39.124999\du,67.766493\du);
}
\definecolor{dialinecolor}{rgb}{0.000000, 0.000000, 0.000000}
\pgfsetstrokecolor{dialinecolor}
\draw (37.500000\du,70.581100\du)--(39.124999\du,67.766493\du);
\pgfsetlinewidth{0.100000\du}
\pgfsetdash{}{0pt}
\pgfsetmiterjoin
\pgfsetbuttcap
\definecolor{dialinecolor}{rgb}{0.000000, 0.000000, 0.000000}
\pgfsetfillcolor{dialinecolor}
\pgfpathmoveto{\pgfpoint{39.124999\du}{67.766493\du}}
\pgfpathcurveto{\pgfpoint{39.233253\du}{67.828993\du}}{\pgfpoint{39.279006\du}{67.999746\du}}{\pgfpoint{39.216507\du}{68.107999\du}}
\pgfpathcurveto{\pgfpoint{39.154007\du}{68.216253\du}}{\pgfpoint{38.983254\du}{68.262006\du}}{\pgfpoint{38.875001\du}{68.199507\du}}
\pgfpathcurveto{\pgfpoint{38.766747\du}{68.137007\du}}{\pgfpoint{38.720994\du}{67.966254\du}}{\pgfpoint{38.783493\du}{67.858001\du}}
\pgfpathcurveto{\pgfpoint{38.845993\du}{67.749747\du}}{\pgfpoint{39.016746\du}{67.703994\du}}{\pgfpoint{39.124999\du}{67.766493\du}}
\pgfusepath{fill}
\definecolor{dialinecolor}{rgb}{0.000000, 0.000000, 0.000000}
\pgfsetstrokecolor{dialinecolor}
\pgfpathmoveto{\pgfpoint{39.124999\du}{67.766493\du}}
\pgfpathcurveto{\pgfpoint{39.233253\du}{67.828993\du}}{\pgfpoint{39.279006\du}{67.999746\du}}{\pgfpoint{39.216507\du}{68.107999\du}}
\pgfpathcurveto{\pgfpoint{39.154007\du}{68.216253\du}}{\pgfpoint{38.983254\du}{68.262006\du}}{\pgfpoint{38.875001\du}{68.199507\du}}
\pgfpathcurveto{\pgfpoint{38.766747\du}{68.137007\du}}{\pgfpoint{38.720994\du}{67.966254\du}}{\pgfpoint{38.783493\du}{67.858001\du}}
\pgfpathcurveto{\pgfpoint{38.845993\du}{67.749747\du}}{\pgfpoint{39.016746\du}{67.703994\du}}{\pgfpoint{39.124999\du}{67.766493\du}}
\pgfusepath{stroke}
% setfont left to latex
\definecolor{dialinecolor}{rgb}{0.000000, 0.000000, 0.000000}
\pgfsetstrokecolor{dialinecolor}
\node at (36.862500\du,76.636500\du){$S\_33$};
% setfont left to latex
\definecolor{dialinecolor}{rgb}{0.000000, 0.000000, 0.000000}
\pgfsetstrokecolor{dialinecolor}
\node at (32.883200\du,74.700700\du){$S\_13$};
% setfont left to latex
\definecolor{dialinecolor}{rgb}{0.000000, 0.000000, 0.000000}
\pgfsetstrokecolor{dialinecolor}
\node at (36.000000\du,69.937500\du){$S\_31$};
% setfont left to latex
\definecolor{dialinecolor}{rgb}{0.000000, 0.000000, 0.000000}
\pgfsetstrokecolor{dialinecolor}
\node at (40.703200\du,74.629300\du){$S\_32$};
% setfont left to latex
\definecolor{dialinecolor}{rgb}{0.000000, 0.000000, 0.000000}
\pgfsetstrokecolor{dialinecolor}
\node at (35.250000\du,73.834700\du){$S\_23$};
\pgfsetlinewidth{0.100000\du}
\pgfsetdash{{\pgflinewidth}{0.200000\du}}{0cm}
\pgfsetdash{{\pgflinewidth}{0.200000\du}}{0cm}
\pgfsetbuttcap
{
\definecolor{dialinecolor}{rgb}{0.000000, 0.000000, 0.000000}
\pgfsetfillcolor{dialinecolor}
% was here!!!
\definecolor{dialinecolor}{rgb}{0.000000, 0.000000, 0.000000}
\pgfsetstrokecolor{dialinecolor}
\draw (27.000000\du,64.518900\du)--(23.250000\du,66.684000\du);
}
\pgfsetlinewidth{0.100000\du}
\pgfsetdash{{\pgflinewidth}{0.200000\du}}{0cm}
\pgfsetdash{{\pgflinewidth}{0.200000\du}}{0cm}
\pgfsetbuttcap
{
\definecolor{dialinecolor}{rgb}{0.000000, 0.000000, 0.000000}
\pgfsetfillcolor{dialinecolor}
% was here!!!
\definecolor{dialinecolor}{rgb}{0.000000, 0.000000, 0.000000}
\pgfsetstrokecolor{dialinecolor}
\draw (50.250000\du,66.684000\du)--(46.500000\du,64.518900\du);
}
% setfont left to latex
\definecolor{dialinecolor}{rgb}{0.000000, 0.000000, 0.000000}
\pgfsetstrokecolor{dialinecolor}
\node at (52.500000\du,59.545200\du){$b\_5$};
% setfont left to latex
\definecolor{dialinecolor}{rgb}{0.000000, 0.000000, 0.000000}
\pgfsetstrokecolor{dialinecolor}
\node at (54.750000\du,63.442400\du){$a\_5$};
% setfont left to latex
\definecolor{dialinecolor}{rgb}{0.000000, 0.000000, 0.000000}
\pgfsetstrokecolor{dialinecolor}
\node at (39.000000\du,92.454200\du){$b\_6$};
% setfont left to latex
\definecolor{dialinecolor}{rgb}{0.000000, 0.000000, 0.000000}
\pgfsetstrokecolor{dialinecolor}
\node at (34.500000\du,92.454200\du){$a\_6$};
\pgfsetlinewidth{0.100000\du}
\pgfsetdash{{\pgflinewidth}{0.200000\du}}{0cm}
\pgfsetdash{{\pgflinewidth}{0.200000\du}}{0cm}
\pgfsetbuttcap
{
\definecolor{dialinecolor}{rgb}{0.000000, 0.000000, 0.000000}
\pgfsetfillcolor{dialinecolor}
% was here!!!
\pgfsetarrowsend{latex}
\definecolor{dialinecolor}{rgb}{0.000000, 0.000000, 0.000000}
\pgfsetstrokecolor{dialinecolor}
\draw (50.250000\du,66.684000\du)--(44.374999\du,76.859706\du);
}
\pgfsetlinewidth{0.100000\du}
\pgfsetdash{{\pgflinewidth}{0.200000\du}}{0cm}
\pgfsetdash{{\pgflinewidth}{0.200000\du}}{0cm}
\pgfsetbuttcap
{
\definecolor{dialinecolor}{rgb}{0.000000, 0.000000, 0.000000}
\pgfsetfillcolor{dialinecolor}
% was here!!!
\definecolor{dialinecolor}{rgb}{0.000000, 0.000000, 0.000000}
\pgfsetstrokecolor{dialinecolor}
\draw (44.500000\du,76.643200\du)--(39.000000\du,86.169500\du);
}
\pgfsetlinewidth{0.100000\du}
\pgfsetdash{{\pgflinewidth}{0.200000\du}}{0cm}
\pgfsetdash{{\pgflinewidth}{0.200000\du}}{0cm}
\pgfsetbuttcap
{
\definecolor{dialinecolor}{rgb}{0.000000, 0.000000, 0.000000}
\pgfsetfillcolor{dialinecolor}
% was here!!!
\pgfsetarrowsend{latex}
\definecolor{dialinecolor}{rgb}{0.000000, 0.000000, 0.000000}
\pgfsetstrokecolor{dialinecolor}
\draw (38.250000\du,84.004500\du)--(43.124999\du,75.560693\du);
}
\pgfsetlinewidth{0.100000\du}
\pgfsetdash{{\pgflinewidth}{0.200000\du}}{0cm}
\pgfsetdash{{\pgflinewidth}{0.200000\du}}{0cm}
\pgfsetbuttcap
{
\definecolor{dialinecolor}{rgb}{0.000000, 0.000000, 0.000000}
\pgfsetfillcolor{dialinecolor}
% was here!!!
\definecolor{dialinecolor}{rgb}{0.000000, 0.000000, 0.000000}
\pgfsetstrokecolor{dialinecolor}
\draw (43.000000\du,75.777200\du)--(48.000000\du,67.117000\du);
}
\pgfsetlinewidth{0.100000\du}
\pgfsetdash{{\pgflinewidth}{0.200000\du}}{0cm}
\pgfsetdash{{\pgflinewidth}{0.200000\du}}{0cm}
\pgfsetbuttcap
{
\definecolor{dialinecolor}{rgb}{0.000000, 0.000000, 0.000000}
\pgfsetfillcolor{dialinecolor}
% was here!!!
\definecolor{dialinecolor}{rgb}{0.000000, 0.000000, 0.000000}
\pgfsetstrokecolor{dialinecolor}
\draw (38.250000\du,84.004500\du)--(34.500000\du,86.169500\du);
}
\pgfsetlinewidth{0.100000\du}
\pgfsetdash{{\pgflinewidth}{0.200000\du}}{0cm}
\pgfsetdash{{\pgflinewidth}{0.200000\du}}{0cm}
\pgfsetbuttcap
{
\definecolor{dialinecolor}{rgb}{0.000000, 0.000000, 0.000000}
\pgfsetfillcolor{dialinecolor}
% was here!!!
\definecolor{dialinecolor}{rgb}{0.000000, 0.000000, 0.000000}
\pgfsetstrokecolor{dialinecolor}
\draw (48.000000\du,67.117000\du)--(48.000000\du,62.786800\du);
}
\pgfsetlinewidth{0.100000\du}
\pgfsetdash{{\pgflinewidth}{0.200000\du}}{0cm}
\pgfsetdash{{\pgflinewidth}{0.200000\du}}{0cm}
\pgfsetbuttcap
{
\definecolor{dialinecolor}{rgb}{0.000000, 0.000000, 0.000000}
\pgfsetfillcolor{dialinecolor}
% was here!!!
\pgfsetarrowsend{latex}
\definecolor{dialinecolor}{rgb}{0.000000, 0.000000, 0.000000}
\pgfsetstrokecolor{dialinecolor}
\draw (34.500000\du,86.169500\du)--(28.625000\du,75.993694\du);
}
\pgfsetlinewidth{0.100000\du}
\pgfsetdash{{\pgflinewidth}{0.200000\du}}{0cm}
\pgfsetdash{{\pgflinewidth}{0.200000\du}}{0cm}
\pgfsetbuttcap
{
\definecolor{dialinecolor}{rgb}{0.000000, 0.000000, 0.000000}
\pgfsetfillcolor{dialinecolor}
% was here!!!
\definecolor{dialinecolor}{rgb}{0.000000, 0.000000, 0.000000}
\pgfsetstrokecolor{dialinecolor}
\draw (28.750000\du,76.210200\du)--(23.250000\du,66.684000\du);
}
\pgfsetlinewidth{0.100000\du}
\pgfsetdash{{\pgflinewidth}{0.200000\du}}{0cm}
\pgfsetdash{{\pgflinewidth}{0.200000\du}}{0cm}
\pgfsetbuttcap
{
\definecolor{dialinecolor}{rgb}{0.000000, 0.000000, 0.000000}
\pgfsetfillcolor{dialinecolor}
% was here!!!
\pgfsetarrowsend{latex}
\definecolor{dialinecolor}{rgb}{0.000000, 0.000000, 0.000000}
\pgfsetstrokecolor{dialinecolor}
\draw (25.500000\du,67.117000\du)--(30.375000\du,75.560706\du);
}
\pgfsetlinewidth{0.100000\du}
\pgfsetdash{{\pgflinewidth}{0.200000\du}}{0cm}
\pgfsetdash{{\pgflinewidth}{0.200000\du}}{0cm}
\pgfsetbuttcap
{
\definecolor{dialinecolor}{rgb}{0.000000, 0.000000, 0.000000}
\pgfsetfillcolor{dialinecolor}
% was here!!!
\definecolor{dialinecolor}{rgb}{0.000000, 0.000000, 0.000000}
\pgfsetstrokecolor{dialinecolor}
\draw (30.250000\du,75.344200\du)--(35.250000\du,84.004500\du);
}
\pgfsetlinewidth{0.100000\du}
\pgfsetdash{{\pgflinewidth}{0.200000\du}}{0cm}
\pgfsetdash{{\pgflinewidth}{0.200000\du}}{0cm}
\pgfsetbuttcap
{
\definecolor{dialinecolor}{rgb}{0.000000, 0.000000, 0.000000}
\pgfsetfillcolor{dialinecolor}
% was here!!!
\definecolor{dialinecolor}{rgb}{0.000000, 0.000000, 0.000000}
\pgfsetstrokecolor{dialinecolor}
\draw (39.000000\du,86.169500\du)--(35.250000\du,84.004500\du);
}
\pgfsetlinewidth{0.100000\du}
\pgfsetdash{{\pgflinewidth}{0.200000\du}}{0cm}
\pgfsetdash{{\pgflinewidth}{0.200000\du}}{0cm}
\pgfsetbuttcap
{
\definecolor{dialinecolor}{rgb}{0.000000, 0.000000, 0.000000}
\pgfsetfillcolor{dialinecolor}
% was here!!!
\definecolor{dialinecolor}{rgb}{0.000000, 0.000000, 0.000000}
\pgfsetstrokecolor{dialinecolor}
\draw (25.500000\du,62.786800\du)--(25.500000\du,67.117000\du);
}
% setfont left to latex
\definecolor{dialinecolor}{rgb}{0.000000, 0.000000, 0.000000}
\pgfsetstrokecolor{dialinecolor}
\node at (37.000000\du,80.329800\du){$e\_33$};
% setfont left to latex
\definecolor{dialinecolor}{rgb}{0.000000, 0.000000, 0.000000}
\pgfsetstrokecolor{dialinecolor}
\node at (37.000000\du,87.258100\du){$e\_66$};
% setfont left to latex
\definecolor{dialinecolor}{rgb}{0.000000, 0.000000, 0.000000}
\pgfsetstrokecolor{dialinecolor}
\node at (33.250000\du,84.227000\du){$e\_36$};
% setfont left to latex
\definecolor{dialinecolor}{rgb}{0.000000, 0.000000, 0.000000}
\pgfsetstrokecolor{dialinecolor}
\node at (40.500000\du,83.794000\du){$e\_63$};
% setfont left to latex
\definecolor{dialinecolor}{rgb}{0.000000, 0.000000, 0.000000}
\pgfsetstrokecolor{dialinecolor}
\node at (45.750000\du,78.164800\du){$e\_65$};
% setfont left to latex
\definecolor{dialinecolor}{rgb}{0.000000, 0.000000, 0.000000}
\pgfsetstrokecolor{dialinecolor}
\node at (43.750000\du,76.432700\du){$e\_56$};
% setfont left to latex
\definecolor{dialinecolor}{rgb}{0.000000, 0.000000, 0.000000}
\pgfsetstrokecolor{dialinecolor}
\node at (30.750000\du,76.432700\du){$e\_64$};
% setfont left to latex
\definecolor{dialinecolor}{rgb}{0.000000, 0.000000, 0.000000}
\pgfsetstrokecolor{dialinecolor}
\node at (28.750000\du,77.298800\du){$e\_46$};
\end{tikzpicture}

	\caption{Three port model}
	\label{fig:threeportmodel}
\end{figure}


\subsection{Calibration Standards}
\label{sec:calstds}

The calibrations used in \ref{sec:solcal} and \ref{sec:soltical} contain measurements of known standards, but apart from the measured trace we also need to know what the actual standard looks like. In order to use the standard in the calibration methods, we need to know the $\Gamma$ (S-Parameter) of the standard.
\subsubsection{Short}
\label{sec:shortstd}
The short standard like other standards is never $0\Omega$, instead it usually has a series inductance that can be frequency dependent. The connector of the standard can also have electrical loss and an electrical length. The parameters of the Short standard in DeEmbed are:
\begin{itemize}
	\item $L_0$: Constant series inductance.
	\item $L_1 \cdot f$: Series inductance with linear frequency dependency
	\item $L_2 \cdot f^2$: Series inductance with quadratic frequency dependency
	\item $L_3 \cdot f^3$: Series inductance with cubic frequency dependency
	\item Loss(dB): Connector or trace loss of the standard
	\item Loss(dB/Hz): Frequency dependent connector or trace loss of the standard
	\item Length (m): Electrical (not physical) length of the connector / trace towards the standard.
\end{itemize}
$\Gamma_{sp}$ is calculated in 3 steps, first the complex impedance of the inductive part is calculated:
\begin{equation}
Z_{sp} = i 2 \pi f (L_0 + L_1 f + L_2 f^2 + L_3 f^3)
\end{equation}
Convert $Z_{sp}$ into $\Gamma_{sp}$ with equation \ref{eqn:stoz}.
\begin{equation}
\label{eqn:stoz}
\Gamma=\frac{\frac{Z}{50\Omega}-1}{\frac{Z}{50\Omega}+1}
\end{equation}
At last we apply electrical length and loss:
\begin{equation}
\Gamma_{sp} = \Gamma_{sp} e^{2 \pi length\cdot f\cdot c^{-1}} 10^{\frac{-Loss_{dB}}{20}} 10^{\frac{-Loss_{dBHz} f }{20}}
\end{equation}
where $c$ is the speed of light (299792458)


\subsubsection{Open}
\label{sec:openstd}
The open standard like other standards is never $\infty\Omega$, instead it usually has a capacitance that can be frequency dependent. The connector of the standard can also have electrical loss and an electrical length. The parameters of the Open standard in DeEmbed are:
\begin{itemize}
	\item $C_0$: Constant series capacitance.
	\item $C_1 \cdot f$: Series capacitance with linear frequency dependency
	\item $C_2 \cdot f^2$: Series capacitance with quadratic frequency dependency
	\item $C_3 \cdot f^3$: Series capacitance with cubic frequency dependency
	\item Loss(dB): Connector or trace loss of the standard
	\item Loss(dB/Hz): Frequency dependent connector or trace loss of the standard
	\item Length (m): Electrical (not physical) length of the connector / trace towards the standard.
\end{itemize}
$\Gamma_{op}$ is calculated in 3 steps, first the complex impedance of the capacitive part is calculated:
\begin{equation}
Z_{op} = \frac{-i}{2 \pi f (C_0 + C_1 f + C_2 f^2 + C_3 f^3)}
\end{equation}
Convert $Z_{op}$ into $\Gamma_{op}$ with equation \ref{eqn:stoz}. 
At last we apply electrical length and loss:
\begin{equation}
\Gamma_{op} = \Gamma_{op} e^{2 \pi length\cdot f\cdot c^{-1}} 10^{\frac{-Loss_{dB}}{20}} 10^{\frac{-Loss_{dBHz} f }{20}}
\end{equation}
where $c$ is the speed of light (299792458)
\subsubsection{Load}
\label{sec:loadstd}
The load standard in DeEmbed can be defined as a perfect resistance ($R_l$) with a series inductance ($L_l$).
\begin{equation}
Z_{lp} = R_l + 2 \pi f \cdot L_l
\end{equation}
Convert $Z_{lp}$ into $\Gamma_{lp}$ with equation \ref{eqn:stoz}.
\subsubsection{Through}
\label{sec:throughstd}
The through standard is only a connector between the two cables, we define the standard with only length and loss:
\begin{itemize}
	\item Loss(dB): Connector or trace loss of the standard
	\item Loss(dB/Hz): Frequency dependent connector or trace loss of the standard
	\item Length (m): Electrical (not physical) length of the connector / trace towards the standard.
\end{itemize}
\begin{equation}
\Gamma_{th} = 1 \cdot e^{2 \pi length\cdot f\cdot c^{-1}} 10^{\frac{-Loss_{dB}}{20}} 10^{\frac{-Loss_{dBHz} f }{20}}
\end{equation}

\newpage

