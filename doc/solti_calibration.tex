\section{Calibration}
SOLTI Calibration is a commonly used calibration method in network analyzers. SOL stands for Short-Open-Load, the standards that are commonly used for a single port calibration. In order to calibrate the s-parameters between different ports, also Through-Isolation (TI) can be used to calibrate S21 and S12.




\subsection{SOL Calibration}
\label{sec:solcal}

\begin{figure}[H]
	\centering
	% Graphic for TeX using PGF
% Title: /home/franss/DeEmbed/deembed/doc/figures/OnePortModel.dia
% Creator: Dia v0.97.3
% CreationDate: Tue Jan 31 14:42:01 2017
% For: franss
% \usepackage{tikz}
% The following commands are not supported in PSTricks at present
% We define them conditionally, so when they are implemented,
% this pgf file will use them.
\ifx\du\undefined
  \newlength{\du}
\fi
\setlength{\du}{15\unitlength}
\begin{tikzpicture}
\pgftransformxscale{1.000000}
\pgftransformyscale{-1.000000}
\definecolor{dialinecolor}{rgb}{0.000000, 0.000000, 0.000000}
\pgfsetstrokecolor{dialinecolor}
\definecolor{dialinecolor}{rgb}{1.000000, 1.000000, 1.000000}
\pgfsetfillcolor{dialinecolor}
\definecolor{dialinecolor}{rgb}{1.000000, 0.996078, 0.870588}
\pgfsetfillcolor{dialinecolor}
\fill (34.000000\du,59.000000\du)--(34.000000\du,65.000000\du)--(37.000000\du,65.000000\du)--(37.000000\du,59.000000\du)--cycle;
\pgfsetlinewidth{0.100000\du}
\pgfsetdash{}{0pt}
\pgfsetdash{}{0pt}
\pgfsetmiterjoin
\definecolor{dialinecolor}{rgb}{0.000000, 0.000000, 0.000000}
\pgfsetstrokecolor{dialinecolor}
\draw (34.000000\du,59.000000\du)--(34.000000\du,65.000000\du)--(37.000000\du,65.000000\du)--(37.000000\du,59.000000\du)--cycle;
% setfont left to latex
\definecolor{dialinecolor}{rgb}{0.000000, 0.000000, 0.000000}
\pgfsetstrokecolor{dialinecolor}
\node at (35.500000\du,62.195000\du){};
\pgfsetlinewidth{0.100000\du}
\pgfsetdash{}{0pt}
\pgfsetdash{}{0pt}
\pgfsetbuttcap
{
\definecolor{dialinecolor}{rgb}{0.000000, 0.000000, 0.000000}
\pgfsetfillcolor{dialinecolor}
% was here!!!
\pgfsetarrowsend{latex}
\definecolor{dialinecolor}{rgb}{0.000000, 0.000000, 0.000000}
\pgfsetstrokecolor{dialinecolor}
\draw (22.750000\du,60.000000\du)--(25.250000\du,60.000000\du);
}
\definecolor{dialinecolor}{rgb}{0.000000, 0.000000, 0.000000}
\pgfsetstrokecolor{dialinecolor}
\draw (23.300000\du,60.000000\du)--(25.250000\du,60.000000\du);
\pgfsetlinewidth{0.100000\du}
\pgfsetdash{}{0pt}
\pgfsetmiterjoin
\pgfsetbuttcap
\definecolor{dialinecolor}{rgb}{1.000000, 1.000000, 1.000000}
\pgfsetfillcolor{dialinecolor}
\pgfpathmoveto{\pgfpoint{22.800000\du}{60.000000\du}}
\pgfpathcurveto{\pgfpoint{22.800000\du}{59.875000\du}}{\pgfpoint{22.925000\du}{59.750000\du}}{\pgfpoint{23.050000\du}{59.750000\du}}
\pgfpathcurveto{\pgfpoint{23.175000\du}{59.750000\du}}{\pgfpoint{23.300000\du}{59.875000\du}}{\pgfpoint{23.300000\du}{60.000000\du}}
\pgfpathcurveto{\pgfpoint{23.300000\du}{60.125000\du}}{\pgfpoint{23.175000\du}{60.250000\du}}{\pgfpoint{23.050000\du}{60.250000\du}}
\pgfpathcurveto{\pgfpoint{22.925000\du}{60.250000\du}}{\pgfpoint{22.800000\du}{60.125000\du}}{\pgfpoint{22.800000\du}{60.000000\du}}
\pgfusepath{fill}
\definecolor{dialinecolor}{rgb}{0.000000, 0.000000, 0.000000}
\pgfsetstrokecolor{dialinecolor}
\pgfpathmoveto{\pgfpoint{22.800000\du}{60.000000\du}}
\pgfpathcurveto{\pgfpoint{22.800000\du}{59.875000\du}}{\pgfpoint{22.925000\du}{59.750000\du}}{\pgfpoint{23.050000\du}{59.750000\du}}
\pgfpathcurveto{\pgfpoint{23.175000\du}{59.750000\du}}{\pgfpoint{23.300000\du}{59.875000\du}}{\pgfpoint{23.300000\du}{60.000000\du}}
\pgfpathcurveto{\pgfpoint{23.300000\du}{60.125000\du}}{\pgfpoint{23.175000\du}{60.250000\du}}{\pgfpoint{23.050000\du}{60.250000\du}}
\pgfpathcurveto{\pgfpoint{22.925000\du}{60.250000\du}}{\pgfpoint{22.800000\du}{60.125000\du}}{\pgfpoint{22.800000\du}{60.000000\du}}
\pgfusepath{stroke}
\pgfsetlinewidth{0.100000\du}
\pgfsetdash{}{0pt}
\pgfsetdash{}{0pt}
\pgfsetbuttcap
{
\definecolor{dialinecolor}{rgb}{0.000000, 0.000000, 0.000000}
\pgfsetfillcolor{dialinecolor}
% was here!!!
\definecolor{dialinecolor}{rgb}{0.000000, 0.000000, 0.000000}
\pgfsetstrokecolor{dialinecolor}
\draw (25.000000\du,60.000000\du)--(27.250000\du,60.000000\du);
}
\definecolor{dialinecolor}{rgb}{0.000000, 0.000000, 0.000000}
\pgfsetstrokecolor{dialinecolor}
\draw (25.000000\du,60.000000\du)--(27.250000\du,60.000000\du);
\pgfsetlinewidth{0.100000\du}
\pgfsetdash{}{0pt}
\pgfsetmiterjoin
\pgfsetbuttcap
\definecolor{dialinecolor}{rgb}{0.000000, 0.000000, 0.000000}
\pgfsetfillcolor{dialinecolor}
\pgfpathmoveto{\pgfpoint{27.250000\du}{60.000000\du}}
\pgfpathcurveto{\pgfpoint{27.250000\du}{60.125000\du}}{\pgfpoint{27.125000\du}{60.250000\du}}{\pgfpoint{27.000000\du}{60.250000\du}}
\pgfpathcurveto{\pgfpoint{26.875000\du}{60.250000\du}}{\pgfpoint{26.750000\du}{60.125000\du}}{\pgfpoint{26.750000\du}{60.000000\du}}
\pgfpathcurveto{\pgfpoint{26.750000\du}{59.875000\du}}{\pgfpoint{26.875000\du}{59.750000\du}}{\pgfpoint{27.000000\du}{59.750000\du}}
\pgfpathcurveto{\pgfpoint{27.125000\du}{59.750000\du}}{\pgfpoint{27.250000\du}{59.875000\du}}{\pgfpoint{27.250000\du}{60.000000\du}}
\pgfusepath{fill}
\definecolor{dialinecolor}{rgb}{0.000000, 0.000000, 0.000000}
\pgfsetstrokecolor{dialinecolor}
\pgfpathmoveto{\pgfpoint{27.250000\du}{60.000000\du}}
\pgfpathcurveto{\pgfpoint{27.250000\du}{60.125000\du}}{\pgfpoint{27.125000\du}{60.250000\du}}{\pgfpoint{27.000000\du}{60.250000\du}}
\pgfpathcurveto{\pgfpoint{26.875000\du}{60.250000\du}}{\pgfpoint{26.750000\du}{60.125000\du}}{\pgfpoint{26.750000\du}{60.000000\du}}
\pgfpathcurveto{\pgfpoint{26.750000\du}{59.875000\du}}{\pgfpoint{26.875000\du}{59.750000\du}}{\pgfpoint{27.000000\du}{59.750000\du}}
\pgfpathcurveto{\pgfpoint{27.125000\du}{59.750000\du}}{\pgfpoint{27.250000\du}{59.875000\du}}{\pgfpoint{27.250000\du}{60.000000\du}}
\pgfusepath{stroke}
\pgfsetlinewidth{0.100000\du}
\pgfsetdash{}{0pt}
\pgfsetdash{}{0pt}
\pgfsetbuttcap
{
\definecolor{dialinecolor}{rgb}{0.000000, 0.000000, 0.000000}
\pgfsetfillcolor{dialinecolor}
% was here!!!
\definecolor{dialinecolor}{rgb}{0.000000, 0.000000, 0.000000}
\pgfsetstrokecolor{dialinecolor}
\draw (22.750000\du,64.000000\du)--(25.000000\du,64.000000\du);
}
\definecolor{dialinecolor}{rgb}{0.000000, 0.000000, 0.000000}
\pgfsetstrokecolor{dialinecolor}
\draw (23.300000\du,64.000000\du)--(25.000000\du,64.000000\du);
\pgfsetlinewidth{0.100000\du}
\pgfsetdash{}{0pt}
\pgfsetmiterjoin
\pgfsetbuttcap
\definecolor{dialinecolor}{rgb}{1.000000, 1.000000, 1.000000}
\pgfsetfillcolor{dialinecolor}
\pgfpathmoveto{\pgfpoint{22.800000\du}{64.000000\du}}
\pgfpathcurveto{\pgfpoint{22.800000\du}{63.875000\du}}{\pgfpoint{22.925000\du}{63.750000\du}}{\pgfpoint{23.050000\du}{63.750000\du}}
\pgfpathcurveto{\pgfpoint{23.175000\du}{63.750000\du}}{\pgfpoint{23.300000\du}{63.875000\du}}{\pgfpoint{23.300000\du}{64.000000\du}}
\pgfpathcurveto{\pgfpoint{23.300000\du}{64.125000\du}}{\pgfpoint{23.175000\du}{64.250000\du}}{\pgfpoint{23.050000\du}{64.250000\du}}
\pgfpathcurveto{\pgfpoint{22.925000\du}{64.250000\du}}{\pgfpoint{22.800000\du}{64.125000\du}}{\pgfpoint{22.800000\du}{64.000000\du}}
\pgfusepath{fill}
\definecolor{dialinecolor}{rgb}{0.000000, 0.000000, 0.000000}
\pgfsetstrokecolor{dialinecolor}
\pgfpathmoveto{\pgfpoint{22.800000\du}{64.000000\du}}
\pgfpathcurveto{\pgfpoint{22.800000\du}{63.875000\du}}{\pgfpoint{22.925000\du}{63.750000\du}}{\pgfpoint{23.050000\du}{63.750000\du}}
\pgfpathcurveto{\pgfpoint{23.175000\du}{63.750000\du}}{\pgfpoint{23.300000\du}{63.875000\du}}{\pgfpoint{23.300000\du}{64.000000\du}}
\pgfpathcurveto{\pgfpoint{23.300000\du}{64.125000\du}}{\pgfpoint{23.175000\du}{64.250000\du}}{\pgfpoint{23.050000\du}{64.250000\du}}
\pgfpathcurveto{\pgfpoint{22.925000\du}{64.250000\du}}{\pgfpoint{22.800000\du}{64.125000\du}}{\pgfpoint{22.800000\du}{64.000000\du}}
\pgfusepath{stroke}
\pgfsetlinewidth{0.100000\du}
\pgfsetdash{}{0pt}
\pgfsetdash{}{0pt}
\pgfsetbuttcap
{
\definecolor{dialinecolor}{rgb}{0.000000, 0.000000, 0.000000}
\pgfsetfillcolor{dialinecolor}
% was here!!!
\pgfsetarrowsstart{latex}
\definecolor{dialinecolor}{rgb}{0.000000, 0.000000, 0.000000}
\pgfsetstrokecolor{dialinecolor}
\draw (24.750000\du,64.000000\du)--(27.250000\du,64.000000\du);
}
\definecolor{dialinecolor}{rgb}{0.000000, 0.000000, 0.000000}
\pgfsetstrokecolor{dialinecolor}
\draw (24.750000\du,64.000000\du)--(27.250000\du,64.000000\du);
\pgfsetlinewidth{0.100000\du}
\pgfsetdash{}{0pt}
\pgfsetmiterjoin
\pgfsetbuttcap
\definecolor{dialinecolor}{rgb}{0.000000, 0.000000, 0.000000}
\pgfsetfillcolor{dialinecolor}
\pgfpathmoveto{\pgfpoint{27.250000\du}{64.000000\du}}
\pgfpathcurveto{\pgfpoint{27.250000\du}{64.125000\du}}{\pgfpoint{27.125000\du}{64.250000\du}}{\pgfpoint{27.000000\du}{64.250000\du}}
\pgfpathcurveto{\pgfpoint{26.875000\du}{64.250000\du}}{\pgfpoint{26.750000\du}{64.125000\du}}{\pgfpoint{26.750000\du}{64.000000\du}}
\pgfpathcurveto{\pgfpoint{26.750000\du}{63.875000\du}}{\pgfpoint{26.875000\du}{63.750000\du}}{\pgfpoint{27.000000\du}{63.750000\du}}
\pgfpathcurveto{\pgfpoint{27.125000\du}{63.750000\du}}{\pgfpoint{27.250000\du}{63.875000\du}}{\pgfpoint{27.250000\du}{64.000000\du}}
\pgfusepath{fill}
\definecolor{dialinecolor}{rgb}{0.000000, 0.000000, 0.000000}
\pgfsetstrokecolor{dialinecolor}
\pgfpathmoveto{\pgfpoint{27.250000\du}{64.000000\du}}
\pgfpathcurveto{\pgfpoint{27.250000\du}{64.125000\du}}{\pgfpoint{27.125000\du}{64.250000\du}}{\pgfpoint{27.000000\du}{64.250000\du}}
\pgfpathcurveto{\pgfpoint{26.875000\du}{64.250000\du}}{\pgfpoint{26.750000\du}{64.125000\du}}{\pgfpoint{26.750000\du}{64.000000\du}}
\pgfpathcurveto{\pgfpoint{26.750000\du}{63.875000\du}}{\pgfpoint{26.875000\du}{63.750000\du}}{\pgfpoint{27.000000\du}{63.750000\du}}
\pgfpathcurveto{\pgfpoint{27.125000\du}{63.750000\du}}{\pgfpoint{27.250000\du}{63.875000\du}}{\pgfpoint{27.250000\du}{64.000000\du}}
\pgfusepath{stroke}
\pgfsetlinewidth{0.100000\du}
\pgfsetdash{}{0pt}
\pgfsetdash{}{0pt}
\pgfsetbuttcap
{
\definecolor{dialinecolor}{rgb}{0.000000, 0.000000, 0.000000}
\pgfsetfillcolor{dialinecolor}
% was here!!!
\pgfsetarrowsend{latex}
\definecolor{dialinecolor}{rgb}{0.000000, 0.000000, 0.000000}
\pgfsetstrokecolor{dialinecolor}
\draw (27.000000\du,60.000000\du)--(27.000000\du,62.250000\du);
}
\pgfsetlinewidth{0.100000\du}
\pgfsetdash{}{0pt}
\pgfsetdash{}{0pt}
\pgfsetbuttcap
{
\definecolor{dialinecolor}{rgb}{0.000000, 0.000000, 0.000000}
\pgfsetfillcolor{dialinecolor}
% was here!!!
\definecolor{dialinecolor}{rgb}{0.000000, 0.000000, 0.000000}
\pgfsetstrokecolor{dialinecolor}
\draw (27.000000\du,62.000000\du)--(27.000000\du,64.000000\du);
}
\pgfsetlinewidth{0.100000\du}
\pgfsetdash{}{0pt}
\pgfsetdash{}{0pt}
\pgfsetbuttcap
{
\definecolor{dialinecolor}{rgb}{0.000000, 0.000000, 0.000000}
\pgfsetfillcolor{dialinecolor}
% was here!!!
\pgfsetarrowsend{latex}
\definecolor{dialinecolor}{rgb}{0.000000, 0.000000, 0.000000}
\pgfsetstrokecolor{dialinecolor}
\draw (27.000000\du,60.000000\du)--(29.250000\du,60.000000\du);
}
\pgfsetlinewidth{0.100000\du}
\pgfsetdash{}{0pt}
\pgfsetdash{}{0pt}
\pgfsetbuttcap
{
\definecolor{dialinecolor}{rgb}{0.000000, 0.000000, 0.000000}
\pgfsetfillcolor{dialinecolor}
% was here!!!
\definecolor{dialinecolor}{rgb}{0.000000, 0.000000, 0.000000}
\pgfsetstrokecolor{dialinecolor}
\draw (29.000000\du,60.000000\du)--(31.250000\du,60.000000\du);
}
\definecolor{dialinecolor}{rgb}{0.000000, 0.000000, 0.000000}
\pgfsetstrokecolor{dialinecolor}
\draw (29.000000\du,60.000000\du)--(31.250000\du,60.000000\du);
\pgfsetlinewidth{0.100000\du}
\pgfsetdash{}{0pt}
\pgfsetmiterjoin
\pgfsetbuttcap
\definecolor{dialinecolor}{rgb}{0.000000, 0.000000, 0.000000}
\pgfsetfillcolor{dialinecolor}
\pgfpathmoveto{\pgfpoint{31.250000\du}{60.000000\du}}
\pgfpathcurveto{\pgfpoint{31.250000\du}{60.125000\du}}{\pgfpoint{31.125000\du}{60.250000\du}}{\pgfpoint{31.000000\du}{60.250000\du}}
\pgfpathcurveto{\pgfpoint{30.875000\du}{60.250000\du}}{\pgfpoint{30.750000\du}{60.125000\du}}{\pgfpoint{30.750000\du}{60.000000\du}}
\pgfpathcurveto{\pgfpoint{30.750000\du}{59.875000\du}}{\pgfpoint{30.875000\du}{59.750000\du}}{\pgfpoint{31.000000\du}{59.750000\du}}
\pgfpathcurveto{\pgfpoint{31.125000\du}{59.750000\du}}{\pgfpoint{31.250000\du}{59.875000\du}}{\pgfpoint{31.250000\du}{60.000000\du}}
\pgfusepath{fill}
\definecolor{dialinecolor}{rgb}{0.000000, 0.000000, 0.000000}
\pgfsetstrokecolor{dialinecolor}
\pgfpathmoveto{\pgfpoint{31.250000\du}{60.000000\du}}
\pgfpathcurveto{\pgfpoint{31.250000\du}{60.125000\du}}{\pgfpoint{31.125000\du}{60.250000\du}}{\pgfpoint{31.000000\du}{60.250000\du}}
\pgfpathcurveto{\pgfpoint{30.875000\du}{60.250000\du}}{\pgfpoint{30.750000\du}{60.125000\du}}{\pgfpoint{30.750000\du}{60.000000\du}}
\pgfpathcurveto{\pgfpoint{30.750000\du}{59.875000\du}}{\pgfpoint{30.875000\du}{59.750000\du}}{\pgfpoint{31.000000\du}{59.750000\du}}
\pgfpathcurveto{\pgfpoint{31.125000\du}{59.750000\du}}{\pgfpoint{31.250000\du}{59.875000\du}}{\pgfpoint{31.250000\du}{60.000000\du}}
\pgfusepath{stroke}
\pgfsetlinewidth{0.100000\du}
\pgfsetdash{}{0pt}
\pgfsetdash{}{0pt}
\pgfsetbuttcap
{
\definecolor{dialinecolor}{rgb}{0.000000, 0.000000, 0.000000}
\pgfsetfillcolor{dialinecolor}
% was here!!!
\pgfsetarrowsend{latex}
\definecolor{dialinecolor}{rgb}{0.000000, 0.000000, 0.000000}
\pgfsetstrokecolor{dialinecolor}
\draw (31.000000\du,60.000000\du)--(33.250000\du,60.000000\du);
}
\pgfsetlinewidth{0.100000\du}
\pgfsetdash{}{0pt}
\pgfsetdash{}{0pt}
\pgfsetbuttcap
{
\definecolor{dialinecolor}{rgb}{0.000000, 0.000000, 0.000000}
\pgfsetfillcolor{dialinecolor}
% was here!!!
\definecolor{dialinecolor}{rgb}{0.000000, 0.000000, 0.000000}
\pgfsetstrokecolor{dialinecolor}
\draw (33.000000\du,60.000000\du)--(35.250000\du,60.000000\du);
}
\definecolor{dialinecolor}{rgb}{0.000000, 0.000000, 0.000000}
\pgfsetstrokecolor{dialinecolor}
\draw (33.000000\du,60.000000\du)--(35.250000\du,60.000000\du);
\pgfsetlinewidth{0.100000\du}
\pgfsetdash{}{0pt}
\pgfsetmiterjoin
\pgfsetbuttcap
\definecolor{dialinecolor}{rgb}{0.000000, 0.000000, 0.000000}
\pgfsetfillcolor{dialinecolor}
\pgfpathmoveto{\pgfpoint{35.250000\du}{60.000000\du}}
\pgfpathcurveto{\pgfpoint{35.250000\du}{60.125000\du}}{\pgfpoint{35.125000\du}{60.250000\du}}{\pgfpoint{35.000000\du}{60.250000\du}}
\pgfpathcurveto{\pgfpoint{34.875000\du}{60.250000\du}}{\pgfpoint{34.750000\du}{60.125000\du}}{\pgfpoint{34.750000\du}{60.000000\du}}
\pgfpathcurveto{\pgfpoint{34.750000\du}{59.875000\du}}{\pgfpoint{34.875000\du}{59.750000\du}}{\pgfpoint{35.000000\du}{59.750000\du}}
\pgfpathcurveto{\pgfpoint{35.125000\du}{59.750000\du}}{\pgfpoint{35.250000\du}{59.875000\du}}{\pgfpoint{35.250000\du}{60.000000\du}}
\pgfusepath{fill}
\definecolor{dialinecolor}{rgb}{0.000000, 0.000000, 0.000000}
\pgfsetstrokecolor{dialinecolor}
\pgfpathmoveto{\pgfpoint{35.250000\du}{60.000000\du}}
\pgfpathcurveto{\pgfpoint{35.250000\du}{60.125000\du}}{\pgfpoint{35.125000\du}{60.250000\du}}{\pgfpoint{35.000000\du}{60.250000\du}}
\pgfpathcurveto{\pgfpoint{34.875000\du}{60.250000\du}}{\pgfpoint{34.750000\du}{60.125000\du}}{\pgfpoint{34.750000\du}{60.000000\du}}
\pgfpathcurveto{\pgfpoint{34.750000\du}{59.875000\du}}{\pgfpoint{34.875000\du}{59.750000\du}}{\pgfpoint{35.000000\du}{59.750000\du}}
\pgfpathcurveto{\pgfpoint{35.125000\du}{59.750000\du}}{\pgfpoint{35.250000\du}{59.875000\du}}{\pgfpoint{35.250000\du}{60.000000\du}}
\pgfusepath{stroke}
\pgfsetlinewidth{0.100000\du}
\pgfsetdash{}{0pt}
\pgfsetdash{}{0pt}
\pgfsetbuttcap
{
\definecolor{dialinecolor}{rgb}{0.000000, 0.000000, 0.000000}
\pgfsetfillcolor{dialinecolor}
% was here!!!
\pgfsetarrowsend{latex}
\definecolor{dialinecolor}{rgb}{0.000000, 0.000000, 0.000000}
\pgfsetstrokecolor{dialinecolor}
\draw (31.000000\du,64.000000\du)--(31.000000\du,61.750000\du);
}
\pgfsetlinewidth{0.100000\du}
\pgfsetdash{}{0pt}
\pgfsetdash{}{0pt}
\pgfsetbuttcap
{
\definecolor{dialinecolor}{rgb}{0.000000, 0.000000, 0.000000}
\pgfsetfillcolor{dialinecolor}
% was here!!!
\definecolor{dialinecolor}{rgb}{0.000000, 0.000000, 0.000000}
\pgfsetstrokecolor{dialinecolor}
\draw (31.000000\du,60.000000\du)--(31.000000\du,62.000000\du);
}
\pgfsetlinewidth{0.100000\du}
\pgfsetdash{}{0pt}
\pgfsetdash{}{0pt}
\pgfsetbuttcap
{
\definecolor{dialinecolor}{rgb}{0.000000, 0.000000, 0.000000}
\pgfsetfillcolor{dialinecolor}
% was here!!!
\pgfsetarrowsend{latex}
\definecolor{dialinecolor}{rgb}{0.000000, 0.000000, 0.000000}
\pgfsetstrokecolor{dialinecolor}
\draw (31.250000\du,64.000000\du)--(28.750000\du,64.000000\du);
}
\definecolor{dialinecolor}{rgb}{0.000000, 0.000000, 0.000000}
\pgfsetstrokecolor{dialinecolor}
\draw (31.250000\du,64.000000\du)--(28.750000\du,64.000000\du);
\pgfsetlinewidth{0.100000\du}
\pgfsetdash{}{0pt}
\pgfsetmiterjoin
\pgfsetbuttcap
\definecolor{dialinecolor}{rgb}{0.000000, 0.000000, 0.000000}
\pgfsetfillcolor{dialinecolor}
\pgfpathmoveto{\pgfpoint{31.250000\du}{64.000000\du}}
\pgfpathcurveto{\pgfpoint{31.250000\du}{64.125000\du}}{\pgfpoint{31.125000\du}{64.250000\du}}{\pgfpoint{31.000000\du}{64.250000\du}}
\pgfpathcurveto{\pgfpoint{30.875000\du}{64.250000\du}}{\pgfpoint{30.750000\du}{64.125000\du}}{\pgfpoint{30.750000\du}{64.000000\du}}
\pgfpathcurveto{\pgfpoint{30.750000\du}{63.875000\du}}{\pgfpoint{30.875000\du}{63.750000\du}}{\pgfpoint{31.000000\du}{63.750000\du}}
\pgfpathcurveto{\pgfpoint{31.125000\du}{63.750000\du}}{\pgfpoint{31.250000\du}{63.875000\du}}{\pgfpoint{31.250000\du}{64.000000\du}}
\pgfusepath{fill}
\definecolor{dialinecolor}{rgb}{0.000000, 0.000000, 0.000000}
\pgfsetstrokecolor{dialinecolor}
\pgfpathmoveto{\pgfpoint{31.250000\du}{64.000000\du}}
\pgfpathcurveto{\pgfpoint{31.250000\du}{64.125000\du}}{\pgfpoint{31.125000\du}{64.250000\du}}{\pgfpoint{31.000000\du}{64.250000\du}}
\pgfpathcurveto{\pgfpoint{30.875000\du}{64.250000\du}}{\pgfpoint{30.750000\du}{64.125000\du}}{\pgfpoint{30.750000\du}{64.000000\du}}
\pgfpathcurveto{\pgfpoint{30.750000\du}{63.875000\du}}{\pgfpoint{30.875000\du}{63.750000\du}}{\pgfpoint{31.000000\du}{63.750000\du}}
\pgfpathcurveto{\pgfpoint{31.125000\du}{63.750000\du}}{\pgfpoint{31.250000\du}{63.875000\du}}{\pgfpoint{31.250000\du}{64.000000\du}}
\pgfusepath{stroke}
\pgfsetlinewidth{0.100000\du}
\pgfsetdash{}{0pt}
\pgfsetdash{}{0pt}
\pgfsetbuttcap
{
\definecolor{dialinecolor}{rgb}{0.000000, 0.000000, 0.000000}
\pgfsetfillcolor{dialinecolor}
% was here!!!
\definecolor{dialinecolor}{rgb}{0.000000, 0.000000, 0.000000}
\pgfsetstrokecolor{dialinecolor}
\draw (27.000000\du,64.000000\du)--(29.000000\du,64.000000\du);
}
\pgfsetlinewidth{0.100000\du}
\pgfsetdash{}{0pt}
\pgfsetdash{}{0pt}
\pgfsetbuttcap
{
\definecolor{dialinecolor}{rgb}{0.000000, 0.000000, 0.000000}
\pgfsetfillcolor{dialinecolor}
% was here!!!
\pgfsetarrowsend{latex}
\definecolor{dialinecolor}{rgb}{0.000000, 0.000000, 0.000000}
\pgfsetstrokecolor{dialinecolor}
\draw (35.250000\du,64.000000\du)--(32.750000\du,64.000000\du);
}
\definecolor{dialinecolor}{rgb}{0.000000, 0.000000, 0.000000}
\pgfsetstrokecolor{dialinecolor}
\draw (35.250000\du,64.000000\du)--(32.750000\du,64.000000\du);
\pgfsetlinewidth{0.100000\du}
\pgfsetdash{}{0pt}
\pgfsetmiterjoin
\pgfsetbuttcap
\definecolor{dialinecolor}{rgb}{0.000000, 0.000000, 0.000000}
\pgfsetfillcolor{dialinecolor}
\pgfpathmoveto{\pgfpoint{35.250000\du}{64.000000\du}}
\pgfpathcurveto{\pgfpoint{35.250000\du}{64.125000\du}}{\pgfpoint{35.125000\du}{64.250000\du}}{\pgfpoint{35.000000\du}{64.250000\du}}
\pgfpathcurveto{\pgfpoint{34.875000\du}{64.250000\du}}{\pgfpoint{34.750000\du}{64.125000\du}}{\pgfpoint{34.750000\du}{64.000000\du}}
\pgfpathcurveto{\pgfpoint{34.750000\du}{63.875000\du}}{\pgfpoint{34.875000\du}{63.750000\du}}{\pgfpoint{35.000000\du}{63.750000\du}}
\pgfpathcurveto{\pgfpoint{35.125000\du}{63.750000\du}}{\pgfpoint{35.250000\du}{63.875000\du}}{\pgfpoint{35.250000\du}{64.000000\du}}
\pgfusepath{fill}
\definecolor{dialinecolor}{rgb}{0.000000, 0.000000, 0.000000}
\pgfsetstrokecolor{dialinecolor}
\pgfpathmoveto{\pgfpoint{35.250000\du}{64.000000\du}}
\pgfpathcurveto{\pgfpoint{35.250000\du}{64.125000\du}}{\pgfpoint{35.125000\du}{64.250000\du}}{\pgfpoint{35.000000\du}{64.250000\du}}
\pgfpathcurveto{\pgfpoint{34.875000\du}{64.250000\du}}{\pgfpoint{34.750000\du}{64.125000\du}}{\pgfpoint{34.750000\du}{64.000000\du}}
\pgfpathcurveto{\pgfpoint{34.750000\du}{63.875000\du}}{\pgfpoint{34.875000\du}{63.750000\du}}{\pgfpoint{35.000000\du}{63.750000\du}}
\pgfpathcurveto{\pgfpoint{35.125000\du}{63.750000\du}}{\pgfpoint{35.250000\du}{63.875000\du}}{\pgfpoint{35.250000\du}{64.000000\du}}
\pgfusepath{stroke}
\pgfsetlinewidth{0.100000\du}
\pgfsetdash{}{0pt}
\pgfsetdash{}{0pt}
\pgfsetbuttcap
{
\definecolor{dialinecolor}{rgb}{0.000000, 0.000000, 0.000000}
\pgfsetfillcolor{dialinecolor}
% was here!!!
\definecolor{dialinecolor}{rgb}{0.000000, 0.000000, 0.000000}
\pgfsetstrokecolor{dialinecolor}
\draw (31.000000\du,64.000000\du)--(33.000000\du,64.000000\du);
}
\pgfsetlinewidth{0.100000\du}
\pgfsetdash{}{0pt}
\pgfsetdash{}{0pt}
\pgfsetbuttcap
{
\definecolor{dialinecolor}{rgb}{0.000000, 0.000000, 0.000000}
\pgfsetfillcolor{dialinecolor}
% was here!!!
\pgfsetarrowsend{latex}
\definecolor{dialinecolor}{rgb}{0.000000, 0.000000, 0.000000}
\pgfsetstrokecolor{dialinecolor}
\draw (35.000000\du,60.000000\du)--(35.000000\du,62.250000\du);
}
\pgfsetlinewidth{0.100000\du}
\pgfsetdash{}{0pt}
\pgfsetdash{}{0pt}
\pgfsetbuttcap
{
\definecolor{dialinecolor}{rgb}{0.000000, 0.000000, 0.000000}
\pgfsetfillcolor{dialinecolor}
% was here!!!
\definecolor{dialinecolor}{rgb}{0.000000, 0.000000, 0.000000}
\pgfsetstrokecolor{dialinecolor}
\draw (35.000000\du,62.000000\du)--(35.000000\du,64.000000\du);
}
% setfont left to latex
\definecolor{dialinecolor}{rgb}{0.000000, 0.000000, 0.000000}
\pgfsetstrokecolor{dialinecolor}
\node[anchor=west] at (23.000000\du,59.000000\du){};
% setfont left to latex
\definecolor{dialinecolor}{rgb}{0.000000, 0.000000, 0.000000}
\pgfsetstrokecolor{dialinecolor}
\node at (22.000000\du,60.213188\du){$a_{0}$};
% setfont left to latex
\definecolor{dialinecolor}{rgb}{0.000000, 0.000000, 0.000000}
\pgfsetstrokecolor{dialinecolor}
\node at (22.000000\du,64.213188\du){$b_{0}$};
% setfont left to latex
\definecolor{dialinecolor}{rgb}{0.000000, 0.000000, 0.000000}
\pgfsetstrokecolor{dialinecolor}
\node at (29.000000\du,61.222500\du){$e_{10}$};
% setfont left to latex
\definecolor{dialinecolor}{rgb}{0.000000, 0.000000, 0.000000}
\pgfsetstrokecolor{dialinecolor}
\node at (29.000000\du,63.222500\du){$e_{01}$};
% setfont left to latex
\definecolor{dialinecolor}{rgb}{0.000000, 0.000000, 0.000000}
\pgfsetstrokecolor{dialinecolor}
\node at (28.000000\du,62.222500\du){$e_{00}$};
% setfont left to latex
\definecolor{dialinecolor}{rgb}{0.000000, 0.000000, 0.000000}
\pgfsetstrokecolor{dialinecolor}
\node at (30.000000\du,62.213188\du){$e_{11}$};
% setfont left to latex
\definecolor{dialinecolor}{rgb}{0.000000, 0.000000, 0.000000}
\pgfsetstrokecolor{dialinecolor}
\node at (36.000000\du,60.222500\du){$a_{1}$};
% setfont left to latex
\definecolor{dialinecolor}{rgb}{0.000000, 0.000000, 0.000000}
\pgfsetstrokecolor{dialinecolor}
\node at (36.000000\du,64.222500\du){$b_{1}$};
% setfont left to latex
\definecolor{dialinecolor}{rgb}{0.000000, 0.000000, 0.000000}
\pgfsetstrokecolor{dialinecolor}
\node at (36.000000\du,62.222500\du){$\Gamma$};
% setfont left to latex
\definecolor{dialinecolor}{rgb}{0.000000, 0.000000, 0.000000}
\pgfsetstrokecolor{dialinecolor}
\node at (35.000000\du,58.222500\du){DUT};
% setfont left to latex
\definecolor{dialinecolor}{rgb}{0.000000, 0.000000, 0.000000}
\pgfsetstrokecolor{dialinecolor}
\node at (31.000000\du,59.222500\du){Port 1};
% setfont left to latex
\definecolor{dialinecolor}{rgb}{0.000000, 0.000000, 0.000000}
\pgfsetstrokecolor{dialinecolor}
\node at (23.000000\du,62.213188\du){$\Gamma_{M}$};
\end{tikzpicture}

	\caption{One port model}
	\label{fig:oneportmodel}
\end{figure}


The measured S-parameter data can be de-embedded to the actual DUT with the lengths of the cables and the error coefficients of the network analyzer taken out of the data as if the DUT was measured directly.
	\begin{equation}
	\label{eqn:solcal}
	S11=\frac{S11_{M}-e_{00}}{(S11_{M}e_{11})-\Delta_{e}}
	\end{equation}
	\begin{itemize}
		\item $e_{00}$ is the Directivity
		\item $e_{11}$ is the port match
		\item $\Delta_{e} = e_{00}e_{11}-(e_{10}e_{01})$, of which $(e_{10}e_{01})$ is the tracking.
		\item $S11$ is the one port S-parameter that you want to display (De-embedded)
		\item 	$S11_M$ is the measured S-parameter including the cable and the errors of the port
	\end{itemize}

	The 3 error coefficients can be obtained from 3 independent measurements of known standards. The commonly used standards are Short, Open and Load, but any known standard can be used instead. (see section \ref{sec:obtainingerrorcoefsSOL})
\newpage
\subsubsection{Obtaining error coefficients for SOL calibration}
\label{sec:obtainingerrorcoefsSOL}

Equation \ref{eqn:solcal} contains 3 error coefficients $e_{00}$, $e_{11}$ and $\Delta_{e}$. From \cite{agilent_calibration} the equations \ref{eqn:obtaining1}, \ref{eqn:obtaining2} and \ref{eqn:obtaining3} are obtained. We see 3 times the same equations, but with different measurements. $\Gamma_1$, $\Gamma_2$ and $\Gamma_3$ are the known independent calibration standards, in this case Short, Open and Load. The standards don't have to be perfect though, a short can for instance have some series inductance or loss (see \ref{sec:calstds}). $\Gamma_{M1}$, $\Gamma_{M2}$ and $\Gamma_{M3}$ are the measured traces, this data is obtained by connecting the well-defined calibration standard to the network analyzer, through the cable that is also used in the measurement and measure the reflection (S11). Equations \ref{eqn:obtaining1}, \ref{eqn:obtaining2} and \ref{eqn:obtaining3} still contain our 3 unknown error coefficients $e_{00}$, $e_{11}$ and $\Delta_{e}$ that we need to solve equation \ref{eqn:solcal}.

	
	\small
	\begin{equation}
	\label{eqn:obtaining1}
	\Gamma_{M{1}} = e_{00}+\Gamma_{1}\Gamma_{M{1}}e{11}-\Gamma_{1}\Delta_{e}
	\end{equation}

	\begin{equation}
	\label{eqn:obtaining2}
	\Gamma_{M{2}} = e_{00}+\Gamma_{2}\Gamma_{M{2}}e{11}-\Gamma_{2}\Delta_{e}
	\end{equation}

	\begin{equation}
	\label{eqn:obtaining3}
	\Gamma_{M{3}} = e_{00}+\Gamma_{3}\Gamma_{M{3}}e{11}-\Gamma_{3}\Delta_{e}
	\end{equation}
	\normalsize

In order to solve $e_{00}$, $e_{11}$ and $\Delta_{e}$ we need to substitute equations  \ref{eqn:obtaining1}, \ref{eqn:obtaining2} and \ref{eqn:obtaining3} into one equation and extract the 3 error coefficients. The result is 3 lengthy equations, but with modern computers they can easily be computed for all the data points in our measurement.

	\small
	\begin{equation}
	\label{eqn:e00}
		e_{00} = -\frac{{\left(\Gamma_{2} \Gamma_{M_{3}} - \Gamma_{3} \Gamma_{M_{3}}\right)} \Gamma_{1} \Gamma_{M_{2}} - {\left(\Gamma_{2} \Gamma_{3} \Gamma_{M_{2}} - \Gamma_{2} \Gamma_{3} \Gamma_{M_{3}} - {\left(\Gamma_{3} \Gamma_{M_{2}} - \Gamma_{2} \Gamma_{M_{3}}\right)} \Gamma_{1}\right)} \Gamma_{M_{1}}}{\Gamma_{1} {\left(\Gamma_{2} - \Gamma_{3}\right)} \Gamma_{M_{1}} + \Gamma_{2} \Gamma_{3} \Gamma_{M_{2}} - \Gamma_{2} \Gamma_{3} \Gamma_{M_{3}} - {\left(\Gamma_{2} \Gamma_{M_{2}} - \Gamma_{3} \Gamma_{M_{3}}\right)} \Gamma_{1}}		
	\end{equation}
	\begin{equation}
	\label{eqn:e11}
	e_{11} = \frac{{\left(\Gamma_{2} - \Gamma_{3}\right)} \Gamma_{M_{1}} - \Gamma_{1} {\left(\Gamma_{M_{2}} - \Gamma_{M_{3}}\right)} + \Gamma_{3} \Gamma_{M_{2}} - \Gamma_{2} \Gamma_{M_{3}}}{\Gamma_{1} {\left(\Gamma_{2} - \Gamma_{3}\right)} \Gamma_{M_{1}} + \Gamma_{2} \Gamma_{3} \Gamma_{M_{2}} - \Gamma_{2} \Gamma_{3} \Gamma_{M_{3}} - {\left(\Gamma_{2} \Gamma_{M_{2}} - \Gamma_{3} \Gamma_{M_{3}}\right)} \Gamma_{1}}
	\end{equation}
	\begin{equation}
	\label{eqn:deltae}
	 \Delta_{e} = -\frac{{\left(\Gamma_{1} {\left(\Gamma_{M_{2}} - \Gamma_{M_{3}}\right)} - \Gamma_{2} \Gamma_{M_{2}} + \Gamma_{3} \Gamma_{M_{3}}\right)} \Gamma_{M_{1}} + {\left(\Gamma_{2} \Gamma_{M_{3}} - \Gamma_{3} \Gamma_{M_{3}}\right)} \Gamma_{M_{2}}}{\Gamma_{1} {\left(\Gamma_{2} - \Gamma_{3}\right)} \Gamma_{M_{1}} + \Gamma_{2} \Gamma_{3} \Gamma_{M_{2}} - \Gamma_{2} \Gamma_{3} \Gamma_{M_{3}} - {\left(\Gamma_{2} \Gamma_{M_{2}} - \Gamma_{3} \Gamma_{M_{3}}\right)} \Gamma_{1}}
	\end{equation}
	\normalsize
	
\newpage
\subsection{Full two port calibration}
\label{sec:soltical}
\begin{figure}[H]
	\centering
	% Graphic for TeX using PGF
% Title: /home/frans/DeEmbed/deembed/doc/figures/TwoPortModel.dia
% Creator: Dia v0.97.3
% CreationDate: Wed Feb 15 21:11:42 2017
% For: frans
% \usepackage{tikz}
% The following commands are not supported in PSTricks at present
% We define them conditionally, so when they are implemented,
% this pgf file will use them.
\ifx\du\undefined
  \newlength{\du}
\fi
\setlength{\du}{15\unitlength}
\begin{tikzpicture}[scale=0.75]
\pgftransformxscale{1.000000}
\pgftransformyscale{-1.000000}
\definecolor{dialinecolor}{rgb}{0.000000, 0.000000, 0.000000}
\pgfsetstrokecolor{dialinecolor}
\definecolor{dialinecolor}{rgb}{1.000000, 1.000000, 1.000000}
\pgfsetfillcolor{dialinecolor}
\definecolor{dialinecolor}{rgb}{1.000000, 0.996078, 0.870588}
\pgfsetfillcolor{dialinecolor}
\fill (34.000000\du,59.000000\du)--(34.000000\du,65.000000\du)--(40.000000\du,65.000000\du)--(40.000000\du,59.000000\du)--cycle;
\pgfsetlinewidth{0.100000\du}
\pgfsetdash{}{0pt}
\pgfsetdash{}{0pt}
\pgfsetmiterjoin
\definecolor{dialinecolor}{rgb}{0.000000, 0.000000, 0.000000}
\pgfsetstrokecolor{dialinecolor}
\draw (34.000000\du,59.000000\du)--(34.000000\du,65.000000\du)--(40.000000\du,65.000000\du)--(40.000000\du,59.000000\du)--cycle;
% setfont left to latex
\definecolor{dialinecolor}{rgb}{0.000000, 0.000000, 0.000000}
\pgfsetstrokecolor{dialinecolor}
\node at (37.000000\du,62.195000\du){};
\pgfsetlinewidth{0.100000\du}
\pgfsetdash{}{0pt}
\pgfsetdash{}{0pt}
\pgfsetbuttcap
{
\definecolor{dialinecolor}{rgb}{0.000000, 0.000000, 0.000000}
\pgfsetfillcolor{dialinecolor}
% was here!!!
\pgfsetarrowsend{latex}
\definecolor{dialinecolor}{rgb}{0.000000, 0.000000, 0.000000}
\pgfsetstrokecolor{dialinecolor}
\draw (22.750000\du,60.000000\du)--(25.250000\du,60.000000\du);
}
\definecolor{dialinecolor}{rgb}{0.000000, 0.000000, 0.000000}
\pgfsetstrokecolor{dialinecolor}
\draw (23.300000\du,60.000000\du)--(25.250000\du,60.000000\du);
\pgfsetlinewidth{0.100000\du}
\pgfsetdash{}{0pt}
\pgfsetmiterjoin
\pgfsetbuttcap
\definecolor{dialinecolor}{rgb}{1.000000, 1.000000, 1.000000}
\pgfsetfillcolor{dialinecolor}
\pgfpathmoveto{\pgfpoint{22.800000\du}{60.000000\du}}
\pgfpathcurveto{\pgfpoint{22.800000\du}{59.875000\du}}{\pgfpoint{22.925000\du}{59.750000\du}}{\pgfpoint{23.050000\du}{59.750000\du}}
\pgfpathcurveto{\pgfpoint{23.175000\du}{59.750000\du}}{\pgfpoint{23.300000\du}{59.875000\du}}{\pgfpoint{23.300000\du}{60.000000\du}}
\pgfpathcurveto{\pgfpoint{23.300000\du}{60.125000\du}}{\pgfpoint{23.175000\du}{60.250000\du}}{\pgfpoint{23.050000\du}{60.250000\du}}
\pgfpathcurveto{\pgfpoint{22.925000\du}{60.250000\du}}{\pgfpoint{22.800000\du}{60.125000\du}}{\pgfpoint{22.800000\du}{60.000000\du}}
\pgfusepath{fill}
\definecolor{dialinecolor}{rgb}{0.000000, 0.000000, 0.000000}
\pgfsetstrokecolor{dialinecolor}
\pgfpathmoveto{\pgfpoint{22.800000\du}{60.000000\du}}
\pgfpathcurveto{\pgfpoint{22.800000\du}{59.875000\du}}{\pgfpoint{22.925000\du}{59.750000\du}}{\pgfpoint{23.050000\du}{59.750000\du}}
\pgfpathcurveto{\pgfpoint{23.175000\du}{59.750000\du}}{\pgfpoint{23.300000\du}{59.875000\du}}{\pgfpoint{23.300000\du}{60.000000\du}}
\pgfpathcurveto{\pgfpoint{23.300000\du}{60.125000\du}}{\pgfpoint{23.175000\du}{60.250000\du}}{\pgfpoint{23.050000\du}{60.250000\du}}
\pgfpathcurveto{\pgfpoint{22.925000\du}{60.250000\du}}{\pgfpoint{22.800000\du}{60.125000\du}}{\pgfpoint{22.800000\du}{60.000000\du}}
\pgfusepath{stroke}
\pgfsetlinewidth{0.100000\du}
\pgfsetdash{}{0pt}
\pgfsetdash{}{0pt}
\pgfsetbuttcap
{
\definecolor{dialinecolor}{rgb}{0.000000, 0.000000, 0.000000}
\pgfsetfillcolor{dialinecolor}
% was here!!!
\definecolor{dialinecolor}{rgb}{0.000000, 0.000000, 0.000000}
\pgfsetstrokecolor{dialinecolor}
\draw (25.000000\du,60.000000\du)--(27.250000\du,60.000000\du);
}
\definecolor{dialinecolor}{rgb}{0.000000, 0.000000, 0.000000}
\pgfsetstrokecolor{dialinecolor}
\draw (25.000000\du,60.000000\du)--(27.250000\du,60.000000\du);
\pgfsetlinewidth{0.100000\du}
\pgfsetdash{}{0pt}
\pgfsetmiterjoin
\pgfsetbuttcap
\definecolor{dialinecolor}{rgb}{0.000000, 0.000000, 0.000000}
\pgfsetfillcolor{dialinecolor}
\pgfpathmoveto{\pgfpoint{27.250000\du}{60.000000\du}}
\pgfpathcurveto{\pgfpoint{27.250000\du}{60.125000\du}}{\pgfpoint{27.125000\du}{60.250000\du}}{\pgfpoint{27.000000\du}{60.250000\du}}
\pgfpathcurveto{\pgfpoint{26.875000\du}{60.250000\du}}{\pgfpoint{26.750000\du}{60.125000\du}}{\pgfpoint{26.750000\du}{60.000000\du}}
\pgfpathcurveto{\pgfpoint{26.750000\du}{59.875000\du}}{\pgfpoint{26.875000\du}{59.750000\du}}{\pgfpoint{27.000000\du}{59.750000\du}}
\pgfpathcurveto{\pgfpoint{27.125000\du}{59.750000\du}}{\pgfpoint{27.250000\du}{59.875000\du}}{\pgfpoint{27.250000\du}{60.000000\du}}
\pgfusepath{fill}
\definecolor{dialinecolor}{rgb}{0.000000, 0.000000, 0.000000}
\pgfsetstrokecolor{dialinecolor}
\pgfpathmoveto{\pgfpoint{27.250000\du}{60.000000\du}}
\pgfpathcurveto{\pgfpoint{27.250000\du}{60.125000\du}}{\pgfpoint{27.125000\du}{60.250000\du}}{\pgfpoint{27.000000\du}{60.250000\du}}
\pgfpathcurveto{\pgfpoint{26.875000\du}{60.250000\du}}{\pgfpoint{26.750000\du}{60.125000\du}}{\pgfpoint{26.750000\du}{60.000000\du}}
\pgfpathcurveto{\pgfpoint{26.750000\du}{59.875000\du}}{\pgfpoint{26.875000\du}{59.750000\du}}{\pgfpoint{27.000000\du}{59.750000\du}}
\pgfpathcurveto{\pgfpoint{27.125000\du}{59.750000\du}}{\pgfpoint{27.250000\du}{59.875000\du}}{\pgfpoint{27.250000\du}{60.000000\du}}
\pgfusepath{stroke}
\pgfsetlinewidth{0.100000\du}
\pgfsetdash{}{0pt}
\pgfsetdash{}{0pt}
\pgfsetbuttcap
{
\definecolor{dialinecolor}{rgb}{0.000000, 0.000000, 0.000000}
\pgfsetfillcolor{dialinecolor}
% was here!!!
\definecolor{dialinecolor}{rgb}{0.000000, 0.000000, 0.000000}
\pgfsetstrokecolor{dialinecolor}
\draw (22.750000\du,64.000000\du)--(25.000000\du,64.000000\du);
}
\definecolor{dialinecolor}{rgb}{0.000000, 0.000000, 0.000000}
\pgfsetstrokecolor{dialinecolor}
\draw (23.300000\du,64.000000\du)--(25.000000\du,64.000000\du);
\pgfsetlinewidth{0.100000\du}
\pgfsetdash{}{0pt}
\pgfsetmiterjoin
\pgfsetbuttcap
\definecolor{dialinecolor}{rgb}{1.000000, 1.000000, 1.000000}
\pgfsetfillcolor{dialinecolor}
\pgfpathmoveto{\pgfpoint{22.800000\du}{64.000000\du}}
\pgfpathcurveto{\pgfpoint{22.800000\du}{63.875000\du}}{\pgfpoint{22.925000\du}{63.750000\du}}{\pgfpoint{23.050000\du}{63.750000\du}}
\pgfpathcurveto{\pgfpoint{23.175000\du}{63.750000\du}}{\pgfpoint{23.300000\du}{63.875000\du}}{\pgfpoint{23.300000\du}{64.000000\du}}
\pgfpathcurveto{\pgfpoint{23.300000\du}{64.125000\du}}{\pgfpoint{23.175000\du}{64.250000\du}}{\pgfpoint{23.050000\du}{64.250000\du}}
\pgfpathcurveto{\pgfpoint{22.925000\du}{64.250000\du}}{\pgfpoint{22.800000\du}{64.125000\du}}{\pgfpoint{22.800000\du}{64.000000\du}}
\pgfusepath{fill}
\definecolor{dialinecolor}{rgb}{0.000000, 0.000000, 0.000000}
\pgfsetstrokecolor{dialinecolor}
\pgfpathmoveto{\pgfpoint{22.800000\du}{64.000000\du}}
\pgfpathcurveto{\pgfpoint{22.800000\du}{63.875000\du}}{\pgfpoint{22.925000\du}{63.750000\du}}{\pgfpoint{23.050000\du}{63.750000\du}}
\pgfpathcurveto{\pgfpoint{23.175000\du}{63.750000\du}}{\pgfpoint{23.300000\du}{63.875000\du}}{\pgfpoint{23.300000\du}{64.000000\du}}
\pgfpathcurveto{\pgfpoint{23.300000\du}{64.125000\du}}{\pgfpoint{23.175000\du}{64.250000\du}}{\pgfpoint{23.050000\du}{64.250000\du}}
\pgfpathcurveto{\pgfpoint{22.925000\du}{64.250000\du}}{\pgfpoint{22.800000\du}{64.125000\du}}{\pgfpoint{22.800000\du}{64.000000\du}}
\pgfusepath{stroke}
\pgfsetlinewidth{0.100000\du}
\pgfsetdash{}{0pt}
\pgfsetdash{}{0pt}
\pgfsetbuttcap
{
\definecolor{dialinecolor}{rgb}{0.000000, 0.000000, 0.000000}
\pgfsetfillcolor{dialinecolor}
% was here!!!
\pgfsetarrowsstart{latex}
\definecolor{dialinecolor}{rgb}{0.000000, 0.000000, 0.000000}
\pgfsetstrokecolor{dialinecolor}
\draw (24.750000\du,64.000000\du)--(27.250000\du,64.000000\du);
}
\definecolor{dialinecolor}{rgb}{0.000000, 0.000000, 0.000000}
\pgfsetstrokecolor{dialinecolor}
\draw (24.750000\du,64.000000\du)--(27.250000\du,64.000000\du);
\pgfsetlinewidth{0.100000\du}
\pgfsetdash{}{0pt}
\pgfsetmiterjoin
\pgfsetbuttcap
\definecolor{dialinecolor}{rgb}{0.000000, 0.000000, 0.000000}
\pgfsetfillcolor{dialinecolor}
\pgfpathmoveto{\pgfpoint{27.250000\du}{64.000000\du}}
\pgfpathcurveto{\pgfpoint{27.250000\du}{64.125000\du}}{\pgfpoint{27.125000\du}{64.250000\du}}{\pgfpoint{27.000000\du}{64.250000\du}}
\pgfpathcurveto{\pgfpoint{26.875000\du}{64.250000\du}}{\pgfpoint{26.750000\du}{64.125000\du}}{\pgfpoint{26.750000\du}{64.000000\du}}
\pgfpathcurveto{\pgfpoint{26.750000\du}{63.875000\du}}{\pgfpoint{26.875000\du}{63.750000\du}}{\pgfpoint{27.000000\du}{63.750000\du}}
\pgfpathcurveto{\pgfpoint{27.125000\du}{63.750000\du}}{\pgfpoint{27.250000\du}{63.875000\du}}{\pgfpoint{27.250000\du}{64.000000\du}}
\pgfusepath{fill}
\definecolor{dialinecolor}{rgb}{0.000000, 0.000000, 0.000000}
\pgfsetstrokecolor{dialinecolor}
\pgfpathmoveto{\pgfpoint{27.250000\du}{64.000000\du}}
\pgfpathcurveto{\pgfpoint{27.250000\du}{64.125000\du}}{\pgfpoint{27.125000\du}{64.250000\du}}{\pgfpoint{27.000000\du}{64.250000\du}}
\pgfpathcurveto{\pgfpoint{26.875000\du}{64.250000\du}}{\pgfpoint{26.750000\du}{64.125000\du}}{\pgfpoint{26.750000\du}{64.000000\du}}
\pgfpathcurveto{\pgfpoint{26.750000\du}{63.875000\du}}{\pgfpoint{26.875000\du}{63.750000\du}}{\pgfpoint{27.000000\du}{63.750000\du}}
\pgfpathcurveto{\pgfpoint{27.125000\du}{63.750000\du}}{\pgfpoint{27.250000\du}{63.875000\du}}{\pgfpoint{27.250000\du}{64.000000\du}}
\pgfusepath{stroke}
\pgfsetlinewidth{0.100000\du}
\pgfsetdash{}{0pt}
\pgfsetdash{}{0pt}
\pgfsetbuttcap
{
\definecolor{dialinecolor}{rgb}{0.000000, 0.000000, 0.000000}
\pgfsetfillcolor{dialinecolor}
% was here!!!
\pgfsetarrowsend{latex}
\definecolor{dialinecolor}{rgb}{0.000000, 0.000000, 0.000000}
\pgfsetstrokecolor{dialinecolor}
\draw (27.000000\du,60.000000\du)--(27.000000\du,62.250000\du);
}
\pgfsetlinewidth{0.100000\du}
\pgfsetdash{}{0pt}
\pgfsetdash{}{0pt}
\pgfsetbuttcap
{
\definecolor{dialinecolor}{rgb}{0.000000, 0.000000, 0.000000}
\pgfsetfillcolor{dialinecolor}
% was here!!!
\definecolor{dialinecolor}{rgb}{0.000000, 0.000000, 0.000000}
\pgfsetstrokecolor{dialinecolor}
\draw (27.000000\du,62.000000\du)--(27.000000\du,64.000000\du);
}
\pgfsetlinewidth{0.100000\du}
\pgfsetdash{}{0pt}
\pgfsetdash{}{0pt}
\pgfsetbuttcap
{
\definecolor{dialinecolor}{rgb}{0.000000, 0.000000, 0.000000}
\pgfsetfillcolor{dialinecolor}
% was here!!!
\pgfsetarrowsend{latex}
\definecolor{dialinecolor}{rgb}{0.000000, 0.000000, 0.000000}
\pgfsetstrokecolor{dialinecolor}
\draw (27.000000\du,60.000000\du)--(29.250000\du,60.000000\du);
}
\pgfsetlinewidth{0.100000\du}
\pgfsetdash{}{0pt}
\pgfsetdash{}{0pt}
\pgfsetbuttcap
{
\definecolor{dialinecolor}{rgb}{0.000000, 0.000000, 0.000000}
\pgfsetfillcolor{dialinecolor}
% was here!!!
\definecolor{dialinecolor}{rgb}{0.000000, 0.000000, 0.000000}
\pgfsetstrokecolor{dialinecolor}
\draw (29.000000\du,60.000000\du)--(31.250000\du,60.000000\du);
}
\definecolor{dialinecolor}{rgb}{0.000000, 0.000000, 0.000000}
\pgfsetstrokecolor{dialinecolor}
\draw (29.000000\du,60.000000\du)--(31.250000\du,60.000000\du);
\pgfsetlinewidth{0.100000\du}
\pgfsetdash{}{0pt}
\pgfsetmiterjoin
\pgfsetbuttcap
\definecolor{dialinecolor}{rgb}{0.000000, 0.000000, 0.000000}
\pgfsetfillcolor{dialinecolor}
\pgfpathmoveto{\pgfpoint{31.250000\du}{60.000000\du}}
\pgfpathcurveto{\pgfpoint{31.250000\du}{60.125000\du}}{\pgfpoint{31.125000\du}{60.250000\du}}{\pgfpoint{31.000000\du}{60.250000\du}}
\pgfpathcurveto{\pgfpoint{30.875000\du}{60.250000\du}}{\pgfpoint{30.750000\du}{60.125000\du}}{\pgfpoint{30.750000\du}{60.000000\du}}
\pgfpathcurveto{\pgfpoint{30.750000\du}{59.875000\du}}{\pgfpoint{30.875000\du}{59.750000\du}}{\pgfpoint{31.000000\du}{59.750000\du}}
\pgfpathcurveto{\pgfpoint{31.125000\du}{59.750000\du}}{\pgfpoint{31.250000\du}{59.875000\du}}{\pgfpoint{31.250000\du}{60.000000\du}}
\pgfusepath{fill}
\definecolor{dialinecolor}{rgb}{0.000000, 0.000000, 0.000000}
\pgfsetstrokecolor{dialinecolor}
\pgfpathmoveto{\pgfpoint{31.250000\du}{60.000000\du}}
\pgfpathcurveto{\pgfpoint{31.250000\du}{60.125000\du}}{\pgfpoint{31.125000\du}{60.250000\du}}{\pgfpoint{31.000000\du}{60.250000\du}}
\pgfpathcurveto{\pgfpoint{30.875000\du}{60.250000\du}}{\pgfpoint{30.750000\du}{60.125000\du}}{\pgfpoint{30.750000\du}{60.000000\du}}
\pgfpathcurveto{\pgfpoint{30.750000\du}{59.875000\du}}{\pgfpoint{30.875000\du}{59.750000\du}}{\pgfpoint{31.000000\du}{59.750000\du}}
\pgfpathcurveto{\pgfpoint{31.125000\du}{59.750000\du}}{\pgfpoint{31.250000\du}{59.875000\du}}{\pgfpoint{31.250000\du}{60.000000\du}}
\pgfusepath{stroke}
\pgfsetlinewidth{0.100000\du}
\pgfsetdash{}{0pt}
\pgfsetdash{}{0pt}
\pgfsetbuttcap
{
\definecolor{dialinecolor}{rgb}{0.000000, 0.000000, 0.000000}
\pgfsetfillcolor{dialinecolor}
% was here!!!
\pgfsetarrowsend{latex}
\definecolor{dialinecolor}{rgb}{0.000000, 0.000000, 0.000000}
\pgfsetstrokecolor{dialinecolor}
\draw (31.000000\du,60.000000\du)--(33.250000\du,60.000000\du);
}
\pgfsetlinewidth{0.100000\du}
\pgfsetdash{}{0pt}
\pgfsetdash{}{0pt}
\pgfsetbuttcap
{
\definecolor{dialinecolor}{rgb}{0.000000, 0.000000, 0.000000}
\pgfsetfillcolor{dialinecolor}
% was here!!!
\definecolor{dialinecolor}{rgb}{0.000000, 0.000000, 0.000000}
\pgfsetstrokecolor{dialinecolor}
\draw (33.000000\du,60.000000\du)--(35.250000\du,60.000000\du);
}
\definecolor{dialinecolor}{rgb}{0.000000, 0.000000, 0.000000}
\pgfsetstrokecolor{dialinecolor}
\draw (33.000000\du,60.000000\du)--(35.250000\du,60.000000\du);
\pgfsetlinewidth{0.100000\du}
\pgfsetdash{}{0pt}
\pgfsetmiterjoin
\pgfsetbuttcap
\definecolor{dialinecolor}{rgb}{0.000000, 0.000000, 0.000000}
\pgfsetfillcolor{dialinecolor}
\pgfpathmoveto{\pgfpoint{35.250000\du}{60.000000\du}}
\pgfpathcurveto{\pgfpoint{35.250000\du}{60.125000\du}}{\pgfpoint{35.125000\du}{60.250000\du}}{\pgfpoint{35.000000\du}{60.250000\du}}
\pgfpathcurveto{\pgfpoint{34.875000\du}{60.250000\du}}{\pgfpoint{34.750000\du}{60.125000\du}}{\pgfpoint{34.750000\du}{60.000000\du}}
\pgfpathcurveto{\pgfpoint{34.750000\du}{59.875000\du}}{\pgfpoint{34.875000\du}{59.750000\du}}{\pgfpoint{35.000000\du}{59.750000\du}}
\pgfpathcurveto{\pgfpoint{35.125000\du}{59.750000\du}}{\pgfpoint{35.250000\du}{59.875000\du}}{\pgfpoint{35.250000\du}{60.000000\du}}
\pgfusepath{fill}
\definecolor{dialinecolor}{rgb}{0.000000, 0.000000, 0.000000}
\pgfsetstrokecolor{dialinecolor}
\pgfpathmoveto{\pgfpoint{35.250000\du}{60.000000\du}}
\pgfpathcurveto{\pgfpoint{35.250000\du}{60.125000\du}}{\pgfpoint{35.125000\du}{60.250000\du}}{\pgfpoint{35.000000\du}{60.250000\du}}
\pgfpathcurveto{\pgfpoint{34.875000\du}{60.250000\du}}{\pgfpoint{34.750000\du}{60.125000\du}}{\pgfpoint{34.750000\du}{60.000000\du}}
\pgfpathcurveto{\pgfpoint{34.750000\du}{59.875000\du}}{\pgfpoint{34.875000\du}{59.750000\du}}{\pgfpoint{35.000000\du}{59.750000\du}}
\pgfpathcurveto{\pgfpoint{35.125000\du}{59.750000\du}}{\pgfpoint{35.250000\du}{59.875000\du}}{\pgfpoint{35.250000\du}{60.000000\du}}
\pgfusepath{stroke}
\pgfsetlinewidth{0.100000\du}
\pgfsetdash{}{0pt}
\pgfsetdash{}{0pt}
\pgfsetbuttcap
{
\definecolor{dialinecolor}{rgb}{0.000000, 0.000000, 0.000000}
\pgfsetfillcolor{dialinecolor}
% was here!!!
\pgfsetarrowsend{latex}
\definecolor{dialinecolor}{rgb}{0.000000, 0.000000, 0.000000}
\pgfsetstrokecolor{dialinecolor}
\draw (31.000000\du,64.000000\du)--(31.000000\du,61.750000\du);
}
\pgfsetlinewidth{0.100000\du}
\pgfsetdash{}{0pt}
\pgfsetdash{}{0pt}
\pgfsetbuttcap
{
\definecolor{dialinecolor}{rgb}{0.000000, 0.000000, 0.000000}
\pgfsetfillcolor{dialinecolor}
% was here!!!
\definecolor{dialinecolor}{rgb}{0.000000, 0.000000, 0.000000}
\pgfsetstrokecolor{dialinecolor}
\draw (31.000000\du,60.000000\du)--(31.000000\du,62.000000\du);
}
\pgfsetlinewidth{0.100000\du}
\pgfsetdash{}{0pt}
\pgfsetdash{}{0pt}
\pgfsetbuttcap
{
\definecolor{dialinecolor}{rgb}{0.000000, 0.000000, 0.000000}
\pgfsetfillcolor{dialinecolor}
% was here!!!
\pgfsetarrowsend{latex}
\definecolor{dialinecolor}{rgb}{0.000000, 0.000000, 0.000000}
\pgfsetstrokecolor{dialinecolor}
\draw (31.250000\du,64.000000\du)--(28.750000\du,64.000000\du);
}
\definecolor{dialinecolor}{rgb}{0.000000, 0.000000, 0.000000}
\pgfsetstrokecolor{dialinecolor}
\draw (31.250000\du,64.000000\du)--(28.750000\du,64.000000\du);
\pgfsetlinewidth{0.100000\du}
\pgfsetdash{}{0pt}
\pgfsetmiterjoin
\pgfsetbuttcap
\definecolor{dialinecolor}{rgb}{0.000000, 0.000000, 0.000000}
\pgfsetfillcolor{dialinecolor}
\pgfpathmoveto{\pgfpoint{31.250000\du}{64.000000\du}}
\pgfpathcurveto{\pgfpoint{31.250000\du}{64.125000\du}}{\pgfpoint{31.125000\du}{64.250000\du}}{\pgfpoint{31.000000\du}{64.250000\du}}
\pgfpathcurveto{\pgfpoint{30.875000\du}{64.250000\du}}{\pgfpoint{30.750000\du}{64.125000\du}}{\pgfpoint{30.750000\du}{64.000000\du}}
\pgfpathcurveto{\pgfpoint{30.750000\du}{63.875000\du}}{\pgfpoint{30.875000\du}{63.750000\du}}{\pgfpoint{31.000000\du}{63.750000\du}}
\pgfpathcurveto{\pgfpoint{31.125000\du}{63.750000\du}}{\pgfpoint{31.250000\du}{63.875000\du}}{\pgfpoint{31.250000\du}{64.000000\du}}
\pgfusepath{fill}
\definecolor{dialinecolor}{rgb}{0.000000, 0.000000, 0.000000}
\pgfsetstrokecolor{dialinecolor}
\pgfpathmoveto{\pgfpoint{31.250000\du}{64.000000\du}}
\pgfpathcurveto{\pgfpoint{31.250000\du}{64.125000\du}}{\pgfpoint{31.125000\du}{64.250000\du}}{\pgfpoint{31.000000\du}{64.250000\du}}
\pgfpathcurveto{\pgfpoint{30.875000\du}{64.250000\du}}{\pgfpoint{30.750000\du}{64.125000\du}}{\pgfpoint{30.750000\du}{64.000000\du}}
\pgfpathcurveto{\pgfpoint{30.750000\du}{63.875000\du}}{\pgfpoint{30.875000\du}{63.750000\du}}{\pgfpoint{31.000000\du}{63.750000\du}}
\pgfpathcurveto{\pgfpoint{31.125000\du}{63.750000\du}}{\pgfpoint{31.250000\du}{63.875000\du}}{\pgfpoint{31.250000\du}{64.000000\du}}
\pgfusepath{stroke}
\pgfsetlinewidth{0.100000\du}
\pgfsetdash{}{0pt}
\pgfsetdash{}{0pt}
\pgfsetbuttcap
{
\definecolor{dialinecolor}{rgb}{0.000000, 0.000000, 0.000000}
\pgfsetfillcolor{dialinecolor}
% was here!!!
\definecolor{dialinecolor}{rgb}{0.000000, 0.000000, 0.000000}
\pgfsetstrokecolor{dialinecolor}
\draw (27.000000\du,64.000000\du)--(29.000000\du,64.000000\du);
}
\pgfsetlinewidth{0.100000\du}
\pgfsetdash{}{0pt}
\pgfsetdash{}{0pt}
\pgfsetbuttcap
{
\definecolor{dialinecolor}{rgb}{0.000000, 0.000000, 0.000000}
\pgfsetfillcolor{dialinecolor}
% was here!!!
\pgfsetarrowsend{latex}
\definecolor{dialinecolor}{rgb}{0.000000, 0.000000, 0.000000}
\pgfsetstrokecolor{dialinecolor}
\draw (35.250000\du,64.000000\du)--(32.750000\du,64.000000\du);
}
\definecolor{dialinecolor}{rgb}{0.000000, 0.000000, 0.000000}
\pgfsetstrokecolor{dialinecolor}
\draw (35.250000\du,64.000000\du)--(32.750000\du,64.000000\du);
\pgfsetlinewidth{0.100000\du}
\pgfsetdash{}{0pt}
\pgfsetmiterjoin
\pgfsetbuttcap
\definecolor{dialinecolor}{rgb}{0.000000, 0.000000, 0.000000}
\pgfsetfillcolor{dialinecolor}
\pgfpathmoveto{\pgfpoint{35.250000\du}{64.000000\du}}
\pgfpathcurveto{\pgfpoint{35.250000\du}{64.125000\du}}{\pgfpoint{35.125000\du}{64.250000\du}}{\pgfpoint{35.000000\du}{64.250000\du}}
\pgfpathcurveto{\pgfpoint{34.875000\du}{64.250000\du}}{\pgfpoint{34.750000\du}{64.125000\du}}{\pgfpoint{34.750000\du}{64.000000\du}}
\pgfpathcurveto{\pgfpoint{34.750000\du}{63.875000\du}}{\pgfpoint{34.875000\du}{63.750000\du}}{\pgfpoint{35.000000\du}{63.750000\du}}
\pgfpathcurveto{\pgfpoint{35.125000\du}{63.750000\du}}{\pgfpoint{35.250000\du}{63.875000\du}}{\pgfpoint{35.250000\du}{64.000000\du}}
\pgfusepath{fill}
\definecolor{dialinecolor}{rgb}{0.000000, 0.000000, 0.000000}
\pgfsetstrokecolor{dialinecolor}
\pgfpathmoveto{\pgfpoint{35.250000\du}{64.000000\du}}
\pgfpathcurveto{\pgfpoint{35.250000\du}{64.125000\du}}{\pgfpoint{35.125000\du}{64.250000\du}}{\pgfpoint{35.000000\du}{64.250000\du}}
\pgfpathcurveto{\pgfpoint{34.875000\du}{64.250000\du}}{\pgfpoint{34.750000\du}{64.125000\du}}{\pgfpoint{34.750000\du}{64.000000\du}}
\pgfpathcurveto{\pgfpoint{34.750000\du}{63.875000\du}}{\pgfpoint{34.875000\du}{63.750000\du}}{\pgfpoint{35.000000\du}{63.750000\du}}
\pgfpathcurveto{\pgfpoint{35.125000\du}{63.750000\du}}{\pgfpoint{35.250000\du}{63.875000\du}}{\pgfpoint{35.250000\du}{64.000000\du}}
\pgfusepath{stroke}
\pgfsetlinewidth{0.100000\du}
\pgfsetdash{}{0pt}
\pgfsetdash{}{0pt}
\pgfsetbuttcap
{
\definecolor{dialinecolor}{rgb}{0.000000, 0.000000, 0.000000}
\pgfsetfillcolor{dialinecolor}
% was here!!!
\definecolor{dialinecolor}{rgb}{0.000000, 0.000000, 0.000000}
\pgfsetstrokecolor{dialinecolor}
\draw (31.000000\du,64.000000\du)--(33.000000\du,64.000000\du);
}
\pgfsetlinewidth{0.100000\du}
\pgfsetdash{}{0pt}
\pgfsetdash{}{0pt}
\pgfsetbuttcap
{
\definecolor{dialinecolor}{rgb}{0.000000, 0.000000, 0.000000}
\pgfsetfillcolor{dialinecolor}
% was here!!!
\pgfsetarrowsend{latex}
\definecolor{dialinecolor}{rgb}{0.000000, 0.000000, 0.000000}
\pgfsetstrokecolor{dialinecolor}
\draw (35.000000\du,60.000000\du)--(35.000000\du,62.250000\du);
}
\pgfsetlinewidth{0.100000\du}
\pgfsetdash{}{0pt}
\pgfsetdash{}{0pt}
\pgfsetbuttcap
{
\definecolor{dialinecolor}{rgb}{0.000000, 0.000000, 0.000000}
\pgfsetfillcolor{dialinecolor}
% was here!!!
\definecolor{dialinecolor}{rgb}{0.000000, 0.000000, 0.000000}
\pgfsetstrokecolor{dialinecolor}
\draw (35.000000\du,62.000000\du)--(35.000000\du,64.000000\du);
}
% setfont left to latex
\definecolor{dialinecolor}{rgb}{0.000000, 0.000000, 0.000000}
\pgfsetstrokecolor{dialinecolor}
\node[anchor=west] at (23.000000\du,59.000000\du){};
% setfont left to latex
\definecolor{dialinecolor}{rgb}{0.000000, 0.000000, 0.000000}
\pgfsetstrokecolor{dialinecolor}
\node at (22.000000\du,60.222500\du){$a_{0}$};
% setfont left to latex
\definecolor{dialinecolor}{rgb}{0.000000, 0.000000, 0.000000}
\pgfsetstrokecolor{dialinecolor}
\node at (22.000000\du,64.222500\du){$b_{0}$};
% setfont left to latex
\definecolor{dialinecolor}{rgb}{0.000000, 0.000000, 0.000000}
\pgfsetstrokecolor{dialinecolor}
\node at (29.000000\du,60.949900\du){$e_{10}$};
% setfont left to latex
\definecolor{dialinecolor}{rgb}{0.000000, 0.000000, 0.000000}
\pgfsetstrokecolor{dialinecolor}
\node at (29.000000\du,63.495100\du){$e_{01}$};
% setfont left to latex
\definecolor{dialinecolor}{rgb}{0.000000, 0.000000, 0.000000}
\pgfsetstrokecolor{dialinecolor}
\node at (27.809200\du,62.222500\du){$e_{00}$};
% setfont left to latex
\definecolor{dialinecolor}{rgb}{0.000000, 0.000000, 0.000000}
\pgfsetstrokecolor{dialinecolor}
\node at (30.190800\du,62.222500\du){$e_{11}$};
% setfont left to latex
\definecolor{dialinecolor}{rgb}{0.000000, 0.000000, 0.000000}
\pgfsetstrokecolor{dialinecolor}
\node at (35.000000\du,59.444800\du){$a_{1}$};
% setfont left to latex
\definecolor{dialinecolor}{rgb}{0.000000, 0.000000, 0.000000}
\pgfsetstrokecolor{dialinecolor}
\node at (35.000000\du,64.712500\du){$b_{1}$};
% setfont left to latex
\definecolor{dialinecolor}{rgb}{0.000000, 0.000000, 0.000000}
\pgfsetstrokecolor{dialinecolor}
\node at (36.000000\du,62.222500\du){$S_{11}$};
% setfont left to latex
\definecolor{dialinecolor}{rgb}{0.000000, 0.000000, 0.000000}
\pgfsetstrokecolor{dialinecolor}
\node at (37.000000\du,58.222500\du){DUT};
% setfont left to latex
\definecolor{dialinecolor}{rgb}{0.000000, 0.000000, 0.000000}
\pgfsetstrokecolor{dialinecolor}
\node at (31.000000\du,59.222500\du){Port 1};
\pgfsetlinewidth{0.100000\du}
\pgfsetdash{}{0pt}
\pgfsetdash{}{0pt}
\pgfsetbuttcap
{
\definecolor{dialinecolor}{rgb}{0.000000, 0.000000, 0.000000}
\pgfsetfillcolor{dialinecolor}
% was here!!!
\pgfsetarrowsend{latex}
\definecolor{dialinecolor}{rgb}{0.000000, 0.000000, 0.000000}
\pgfsetstrokecolor{dialinecolor}
\draw (35.000000\du,60.000000\du)--(37.250000\du,60.000000\du);
}
\pgfsetlinewidth{0.100000\du}
\pgfsetdash{}{0pt}
\pgfsetdash{}{0pt}
\pgfsetbuttcap
{
\definecolor{dialinecolor}{rgb}{0.000000, 0.000000, 0.000000}
\pgfsetfillcolor{dialinecolor}
% was here!!!
\definecolor{dialinecolor}{rgb}{0.000000, 0.000000, 0.000000}
\pgfsetstrokecolor{dialinecolor}
\draw (37.000000\du,60.000000\du)--(39.250000\du,60.000000\du);
}
\definecolor{dialinecolor}{rgb}{0.000000, 0.000000, 0.000000}
\pgfsetstrokecolor{dialinecolor}
\draw (37.000000\du,60.000000\du)--(39.250000\du,60.000000\du);
\pgfsetlinewidth{0.100000\du}
\pgfsetdash{}{0pt}
\pgfsetmiterjoin
\pgfsetbuttcap
\definecolor{dialinecolor}{rgb}{0.000000, 0.000000, 0.000000}
\pgfsetfillcolor{dialinecolor}
\pgfpathmoveto{\pgfpoint{39.250000\du}{60.000000\du}}
\pgfpathcurveto{\pgfpoint{39.250000\du}{60.125000\du}}{\pgfpoint{39.125000\du}{60.250000\du}}{\pgfpoint{39.000000\du}{60.250000\du}}
\pgfpathcurveto{\pgfpoint{38.875000\du}{60.250000\du}}{\pgfpoint{38.750000\du}{60.125000\du}}{\pgfpoint{38.750000\du}{60.000000\du}}
\pgfpathcurveto{\pgfpoint{38.750000\du}{59.875000\du}}{\pgfpoint{38.875000\du}{59.750000\du}}{\pgfpoint{39.000000\du}{59.750000\du}}
\pgfpathcurveto{\pgfpoint{39.125000\du}{59.750000\du}}{\pgfpoint{39.250000\du}{59.875000\du}}{\pgfpoint{39.250000\du}{60.000000\du}}
\pgfusepath{fill}
\definecolor{dialinecolor}{rgb}{0.000000, 0.000000, 0.000000}
\pgfsetstrokecolor{dialinecolor}
\pgfpathmoveto{\pgfpoint{39.250000\du}{60.000000\du}}
\pgfpathcurveto{\pgfpoint{39.250000\du}{60.125000\du}}{\pgfpoint{39.125000\du}{60.250000\du}}{\pgfpoint{39.000000\du}{60.250000\du}}
\pgfpathcurveto{\pgfpoint{38.875000\du}{60.250000\du}}{\pgfpoint{38.750000\du}{60.125000\du}}{\pgfpoint{38.750000\du}{60.000000\du}}
\pgfpathcurveto{\pgfpoint{38.750000\du}{59.875000\du}}{\pgfpoint{38.875000\du}{59.750000\du}}{\pgfpoint{39.000000\du}{59.750000\du}}
\pgfpathcurveto{\pgfpoint{39.125000\du}{59.750000\du}}{\pgfpoint{39.250000\du}{59.875000\du}}{\pgfpoint{39.250000\du}{60.000000\du}}
\pgfusepath{stroke}
\pgfsetlinewidth{0.100000\du}
\pgfsetdash{}{0pt}
\pgfsetdash{}{0pt}
\pgfsetbuttcap
{
\definecolor{dialinecolor}{rgb}{0.000000, 0.000000, 0.000000}
\pgfsetfillcolor{dialinecolor}
% was here!!!
\pgfsetarrowsend{latex}
\definecolor{dialinecolor}{rgb}{0.000000, 0.000000, 0.000000}
\pgfsetstrokecolor{dialinecolor}
\draw (39.000000\du,64.000000\du)--(39.000000\du,61.750000\du);
}
\pgfsetlinewidth{0.100000\du}
\pgfsetdash{}{0pt}
\pgfsetdash{}{0pt}
\pgfsetbuttcap
{
\definecolor{dialinecolor}{rgb}{0.000000, 0.000000, 0.000000}
\pgfsetfillcolor{dialinecolor}
% was here!!!
\definecolor{dialinecolor}{rgb}{0.000000, 0.000000, 0.000000}
\pgfsetstrokecolor{dialinecolor}
\draw (39.000000\du,60.000000\du)--(39.000000\du,62.000000\du);
}
\pgfsetlinewidth{0.100000\du}
\pgfsetdash{}{0pt}
\pgfsetdash{}{0pt}
\pgfsetbuttcap
{
\definecolor{dialinecolor}{rgb}{0.000000, 0.000000, 0.000000}
\pgfsetfillcolor{dialinecolor}
% was here!!!
\pgfsetarrowsend{latex}
\definecolor{dialinecolor}{rgb}{0.000000, 0.000000, 0.000000}
\pgfsetstrokecolor{dialinecolor}
\draw (39.250000\du,64.000000\du)--(36.750000\du,64.000000\du);
}
\definecolor{dialinecolor}{rgb}{0.000000, 0.000000, 0.000000}
\pgfsetstrokecolor{dialinecolor}
\draw (39.250000\du,64.000000\du)--(36.750000\du,64.000000\du);
\pgfsetlinewidth{0.100000\du}
\pgfsetdash{}{0pt}
\pgfsetmiterjoin
\pgfsetbuttcap
\definecolor{dialinecolor}{rgb}{0.000000, 0.000000, 0.000000}
\pgfsetfillcolor{dialinecolor}
\pgfpathmoveto{\pgfpoint{39.250000\du}{64.000000\du}}
\pgfpathcurveto{\pgfpoint{39.250000\du}{64.125000\du}}{\pgfpoint{39.125000\du}{64.250000\du}}{\pgfpoint{39.000000\du}{64.250000\du}}
\pgfpathcurveto{\pgfpoint{38.875000\du}{64.250000\du}}{\pgfpoint{38.750000\du}{64.125000\du}}{\pgfpoint{38.750000\du}{64.000000\du}}
\pgfpathcurveto{\pgfpoint{38.750000\du}{63.875000\du}}{\pgfpoint{38.875000\du}{63.750000\du}}{\pgfpoint{39.000000\du}{63.750000\du}}
\pgfpathcurveto{\pgfpoint{39.125000\du}{63.750000\du}}{\pgfpoint{39.250000\du}{63.875000\du}}{\pgfpoint{39.250000\du}{64.000000\du}}
\pgfusepath{fill}
\definecolor{dialinecolor}{rgb}{0.000000, 0.000000, 0.000000}
\pgfsetstrokecolor{dialinecolor}
\pgfpathmoveto{\pgfpoint{39.250000\du}{64.000000\du}}
\pgfpathcurveto{\pgfpoint{39.250000\du}{64.125000\du}}{\pgfpoint{39.125000\du}{64.250000\du}}{\pgfpoint{39.000000\du}{64.250000\du}}
\pgfpathcurveto{\pgfpoint{38.875000\du}{64.250000\du}}{\pgfpoint{38.750000\du}{64.125000\du}}{\pgfpoint{38.750000\du}{64.000000\du}}
\pgfpathcurveto{\pgfpoint{38.750000\du}{63.875000\du}}{\pgfpoint{38.875000\du}{63.750000\du}}{\pgfpoint{39.000000\du}{63.750000\du}}
\pgfpathcurveto{\pgfpoint{39.125000\du}{63.750000\du}}{\pgfpoint{39.250000\du}{63.875000\du}}{\pgfpoint{39.250000\du}{64.000000\du}}
\pgfusepath{stroke}
\pgfsetlinewidth{0.100000\du}
\pgfsetdash{}{0pt}
\pgfsetdash{}{0pt}
\pgfsetbuttcap
{
\definecolor{dialinecolor}{rgb}{0.000000, 0.000000, 0.000000}
\pgfsetfillcolor{dialinecolor}
% was here!!!
\definecolor{dialinecolor}{rgb}{0.000000, 0.000000, 0.000000}
\pgfsetstrokecolor{dialinecolor}
\draw (35.000000\du,64.000000\du)--(37.000000\du,64.000000\du);
}
% setfont left to latex
\definecolor{dialinecolor}{rgb}{0.000000, 0.000000, 0.000000}
\pgfsetstrokecolor{dialinecolor}
\node at (39.000000\du,64.714400\du){$a_{2}$};
% setfont left to latex
\definecolor{dialinecolor}{rgb}{0.000000, 0.000000, 0.000000}
\pgfsetstrokecolor{dialinecolor}
\node at (39.000000\du,59.440700\du){$b_{2}$};
% setfont left to latex
\definecolor{dialinecolor}{rgb}{0.000000, 0.000000, 0.000000}
\pgfsetstrokecolor{dialinecolor}
\node at (37.000000\du,61.222500\du){$S_{21}$};
\pgfsetlinewidth{0.100000\du}
\pgfsetdash{}{0pt}
\pgfsetdash{}{0pt}
\pgfsetbuttcap
{
\definecolor{dialinecolor}{rgb}{0.000000, 0.000000, 0.000000}
\pgfsetfillcolor{dialinecolor}
% was here!!!
\pgfsetarrowsend{latex}
\definecolor{dialinecolor}{rgb}{0.000000, 0.000000, 0.000000}
\pgfsetstrokecolor{dialinecolor}
\draw (39.000000\du,60.000000\du)--(41.250000\du,60.000000\du);
}
\pgfsetlinewidth{0.100000\du}
\pgfsetdash{}{0pt}
\pgfsetdash{}{0pt}
\pgfsetbuttcap
{
\definecolor{dialinecolor}{rgb}{0.000000, 0.000000, 0.000000}
\pgfsetfillcolor{dialinecolor}
% was here!!!
\definecolor{dialinecolor}{rgb}{0.000000, 0.000000, 0.000000}
\pgfsetstrokecolor{dialinecolor}
\draw (41.000000\du,60.000000\du)--(43.250000\du,60.000000\du);
}
\definecolor{dialinecolor}{rgb}{0.000000, 0.000000, 0.000000}
\pgfsetstrokecolor{dialinecolor}
\draw (41.000000\du,60.000000\du)--(43.250000\du,60.000000\du);
\pgfsetlinewidth{0.100000\du}
\pgfsetdash{}{0pt}
\pgfsetmiterjoin
\pgfsetbuttcap
\definecolor{dialinecolor}{rgb}{0.000000, 0.000000, 0.000000}
\pgfsetfillcolor{dialinecolor}
\pgfpathmoveto{\pgfpoint{43.250000\du}{60.000000\du}}
\pgfpathcurveto{\pgfpoint{43.250000\du}{60.125000\du}}{\pgfpoint{43.125000\du}{60.250000\du}}{\pgfpoint{43.000000\du}{60.250000\du}}
\pgfpathcurveto{\pgfpoint{42.875000\du}{60.250000\du}}{\pgfpoint{42.750000\du}{60.125000\du}}{\pgfpoint{42.750000\du}{60.000000\du}}
\pgfpathcurveto{\pgfpoint{42.750000\du}{59.875000\du}}{\pgfpoint{42.875000\du}{59.750000\du}}{\pgfpoint{43.000000\du}{59.750000\du}}
\pgfpathcurveto{\pgfpoint{43.125000\du}{59.750000\du}}{\pgfpoint{43.250000\du}{59.875000\du}}{\pgfpoint{43.250000\du}{60.000000\du}}
\pgfusepath{fill}
\definecolor{dialinecolor}{rgb}{0.000000, 0.000000, 0.000000}
\pgfsetstrokecolor{dialinecolor}
\pgfpathmoveto{\pgfpoint{43.250000\du}{60.000000\du}}
\pgfpathcurveto{\pgfpoint{43.250000\du}{60.125000\du}}{\pgfpoint{43.125000\du}{60.250000\du}}{\pgfpoint{43.000000\du}{60.250000\du}}
\pgfpathcurveto{\pgfpoint{42.875000\du}{60.250000\du}}{\pgfpoint{42.750000\du}{60.125000\du}}{\pgfpoint{42.750000\du}{60.000000\du}}
\pgfpathcurveto{\pgfpoint{42.750000\du}{59.875000\du}}{\pgfpoint{42.875000\du}{59.750000\du}}{\pgfpoint{43.000000\du}{59.750000\du}}
\pgfpathcurveto{\pgfpoint{43.125000\du}{59.750000\du}}{\pgfpoint{43.250000\du}{59.875000\du}}{\pgfpoint{43.250000\du}{60.000000\du}}
\pgfusepath{stroke}
\pgfsetlinewidth{0.100000\du}
\pgfsetdash{}{0pt}
\pgfsetdash{}{0pt}
\pgfsetbuttcap
{
\definecolor{dialinecolor}{rgb}{0.000000, 0.000000, 0.000000}
\pgfsetfillcolor{dialinecolor}
% was here!!!
\pgfsetarrowsend{latex}
\definecolor{dialinecolor}{rgb}{0.000000, 0.000000, 0.000000}
\pgfsetstrokecolor{dialinecolor}
\draw (43.000000\du,64.000000\du)--(43.000000\du,61.750000\du);
}
\pgfsetlinewidth{0.100000\du}
\pgfsetdash{}{0pt}
\pgfsetdash{}{0pt}
\pgfsetbuttcap
{
\definecolor{dialinecolor}{rgb}{0.000000, 0.000000, 0.000000}
\pgfsetfillcolor{dialinecolor}
% was here!!!
\definecolor{dialinecolor}{rgb}{0.000000, 0.000000, 0.000000}
\pgfsetstrokecolor{dialinecolor}
\draw (43.000000\du,60.000000\du)--(43.000000\du,62.000000\du);
}
\pgfsetlinewidth{0.100000\du}
\pgfsetdash{}{0pt}
\pgfsetdash{}{0pt}
\pgfsetbuttcap
{
\definecolor{dialinecolor}{rgb}{0.000000, 0.000000, 0.000000}
\pgfsetfillcolor{dialinecolor}
% was here!!!
\pgfsetarrowsend{latex}
\definecolor{dialinecolor}{rgb}{0.000000, 0.000000, 0.000000}
\pgfsetstrokecolor{dialinecolor}
\draw (43.250000\du,64.000000\du)--(40.750000\du,64.000000\du);
}
\definecolor{dialinecolor}{rgb}{0.000000, 0.000000, 0.000000}
\pgfsetstrokecolor{dialinecolor}
\draw (43.250000\du,64.000000\du)--(40.750000\du,64.000000\du);
\pgfsetlinewidth{0.100000\du}
\pgfsetdash{}{0pt}
\pgfsetmiterjoin
\pgfsetbuttcap
\definecolor{dialinecolor}{rgb}{0.000000, 0.000000, 0.000000}
\pgfsetfillcolor{dialinecolor}
\pgfpathmoveto{\pgfpoint{43.250000\du}{64.000000\du}}
\pgfpathcurveto{\pgfpoint{43.250000\du}{64.125000\du}}{\pgfpoint{43.125000\du}{64.250000\du}}{\pgfpoint{43.000000\du}{64.250000\du}}
\pgfpathcurveto{\pgfpoint{42.875000\du}{64.250000\du}}{\pgfpoint{42.750000\du}{64.125000\du}}{\pgfpoint{42.750000\du}{64.000000\du}}
\pgfpathcurveto{\pgfpoint{42.750000\du}{63.875000\du}}{\pgfpoint{42.875000\du}{63.750000\du}}{\pgfpoint{43.000000\du}{63.750000\du}}
\pgfpathcurveto{\pgfpoint{43.125000\du}{63.750000\du}}{\pgfpoint{43.250000\du}{63.875000\du}}{\pgfpoint{43.250000\du}{64.000000\du}}
\pgfusepath{fill}
\definecolor{dialinecolor}{rgb}{0.000000, 0.000000, 0.000000}
\pgfsetstrokecolor{dialinecolor}
\pgfpathmoveto{\pgfpoint{43.250000\du}{64.000000\du}}
\pgfpathcurveto{\pgfpoint{43.250000\du}{64.125000\du}}{\pgfpoint{43.125000\du}{64.250000\du}}{\pgfpoint{43.000000\du}{64.250000\du}}
\pgfpathcurveto{\pgfpoint{42.875000\du}{64.250000\du}}{\pgfpoint{42.750000\du}{64.125000\du}}{\pgfpoint{42.750000\du}{64.000000\du}}
\pgfpathcurveto{\pgfpoint{42.750000\du}{63.875000\du}}{\pgfpoint{42.875000\du}{63.750000\du}}{\pgfpoint{43.000000\du}{63.750000\du}}
\pgfpathcurveto{\pgfpoint{43.125000\du}{63.750000\du}}{\pgfpoint{43.250000\du}{63.875000\du}}{\pgfpoint{43.250000\du}{64.000000\du}}
\pgfusepath{stroke}
\pgfsetlinewidth{0.100000\du}
\pgfsetdash{}{0pt}
\pgfsetdash{}{0pt}
\pgfsetbuttcap
{
\definecolor{dialinecolor}{rgb}{0.000000, 0.000000, 0.000000}
\pgfsetfillcolor{dialinecolor}
% was here!!!
\definecolor{dialinecolor}{rgb}{0.000000, 0.000000, 0.000000}
\pgfsetstrokecolor{dialinecolor}
\draw (39.000000\du,64.000000\du)--(41.000000\du,64.000000\du);
}
\pgfsetlinewidth{0.100000\du}
\pgfsetdash{}{0pt}
\pgfsetdash{}{0pt}
\pgfsetbuttcap
{
\definecolor{dialinecolor}{rgb}{0.000000, 0.000000, 0.000000}
\pgfsetfillcolor{dialinecolor}
% was here!!!
\pgfsetarrowsend{latex}
\definecolor{dialinecolor}{rgb}{0.000000, 0.000000, 0.000000}
\pgfsetstrokecolor{dialinecolor}
\draw (43.000000\du,60.000000\du)--(45.250000\du,60.000000\du);
}
\pgfsetlinewidth{0.100000\du}
\pgfsetdash{}{0pt}
\pgfsetdash{}{0pt}
\pgfsetbuttcap
{
\definecolor{dialinecolor}{rgb}{0.000000, 0.000000, 0.000000}
\pgfsetfillcolor{dialinecolor}
% was here!!!
\definecolor{dialinecolor}{rgb}{0.000000, 0.000000, 0.000000}
\pgfsetstrokecolor{dialinecolor}
\draw (45.000000\du,60.000000\du)--(47.250000\du,60.000000\du);
}
\definecolor{dialinecolor}{rgb}{0.000000, 0.000000, 0.000000}
\pgfsetstrokecolor{dialinecolor}
\draw (45.000000\du,60.000000\du)--(47.250000\du,60.000000\du);
\pgfsetlinewidth{0.100000\du}
\pgfsetdash{}{0pt}
\pgfsetmiterjoin
\pgfsetbuttcap
\definecolor{dialinecolor}{rgb}{0.000000, 0.000000, 0.000000}
\pgfsetfillcolor{dialinecolor}
\pgfpathmoveto{\pgfpoint{47.250000\du}{60.000000\du}}
\pgfpathcurveto{\pgfpoint{47.250000\du}{60.125000\du}}{\pgfpoint{47.125000\du}{60.250000\du}}{\pgfpoint{47.000000\du}{60.250000\du}}
\pgfpathcurveto{\pgfpoint{46.875000\du}{60.250000\du}}{\pgfpoint{46.750000\du}{60.125000\du}}{\pgfpoint{46.750000\du}{60.000000\du}}
\pgfpathcurveto{\pgfpoint{46.750000\du}{59.875000\du}}{\pgfpoint{46.875000\du}{59.750000\du}}{\pgfpoint{47.000000\du}{59.750000\du}}
\pgfpathcurveto{\pgfpoint{47.125000\du}{59.750000\du}}{\pgfpoint{47.250000\du}{59.875000\du}}{\pgfpoint{47.250000\du}{60.000000\du}}
\pgfusepath{fill}
\definecolor{dialinecolor}{rgb}{0.000000, 0.000000, 0.000000}
\pgfsetstrokecolor{dialinecolor}
\pgfpathmoveto{\pgfpoint{47.250000\du}{60.000000\du}}
\pgfpathcurveto{\pgfpoint{47.250000\du}{60.125000\du}}{\pgfpoint{47.125000\du}{60.250000\du}}{\pgfpoint{47.000000\du}{60.250000\du}}
\pgfpathcurveto{\pgfpoint{46.875000\du}{60.250000\du}}{\pgfpoint{46.750000\du}{60.125000\du}}{\pgfpoint{46.750000\du}{60.000000\du}}
\pgfpathcurveto{\pgfpoint{46.750000\du}{59.875000\du}}{\pgfpoint{46.875000\du}{59.750000\du}}{\pgfpoint{47.000000\du}{59.750000\du}}
\pgfpathcurveto{\pgfpoint{47.125000\du}{59.750000\du}}{\pgfpoint{47.250000\du}{59.875000\du}}{\pgfpoint{47.250000\du}{60.000000\du}}
\pgfusepath{stroke}
\pgfsetlinewidth{0.100000\du}
\pgfsetdash{}{0pt}
\pgfsetdash{}{0pt}
\pgfsetbuttcap
{
\definecolor{dialinecolor}{rgb}{0.000000, 0.000000, 0.000000}
\pgfsetfillcolor{dialinecolor}
% was here!!!
\pgfsetarrowsend{latex}
\definecolor{dialinecolor}{rgb}{0.000000, 0.000000, 0.000000}
\pgfsetstrokecolor{dialinecolor}
\draw (47.000000\du,64.000000\du)--(47.000000\du,61.750000\du);
}
\pgfsetlinewidth{0.100000\du}
\pgfsetdash{}{0pt}
\pgfsetdash{}{0pt}
\pgfsetbuttcap
{
\definecolor{dialinecolor}{rgb}{0.000000, 0.000000, 0.000000}
\pgfsetfillcolor{dialinecolor}
% was here!!!
\definecolor{dialinecolor}{rgb}{0.000000, 0.000000, 0.000000}
\pgfsetstrokecolor{dialinecolor}
\draw (47.000000\du,60.000000\du)--(47.000000\du,62.000000\du);
}
\pgfsetlinewidth{0.100000\du}
\pgfsetdash{}{0pt}
\pgfsetdash{}{0pt}
\pgfsetbuttcap
{
\definecolor{dialinecolor}{rgb}{0.000000, 0.000000, 0.000000}
\pgfsetfillcolor{dialinecolor}
% was here!!!
\pgfsetarrowsend{latex}
\definecolor{dialinecolor}{rgb}{0.000000, 0.000000, 0.000000}
\pgfsetstrokecolor{dialinecolor}
\draw (47.250000\du,64.000000\du)--(44.750000\du,64.000000\du);
}
\definecolor{dialinecolor}{rgb}{0.000000, 0.000000, 0.000000}
\pgfsetstrokecolor{dialinecolor}
\draw (47.250000\du,64.000000\du)--(44.750000\du,64.000000\du);
\pgfsetlinewidth{0.100000\du}
\pgfsetdash{}{0pt}
\pgfsetmiterjoin
\pgfsetbuttcap
\definecolor{dialinecolor}{rgb}{0.000000, 0.000000, 0.000000}
\pgfsetfillcolor{dialinecolor}
\pgfpathmoveto{\pgfpoint{47.250000\du}{64.000000\du}}
\pgfpathcurveto{\pgfpoint{47.250000\du}{64.125000\du}}{\pgfpoint{47.125000\du}{64.250000\du}}{\pgfpoint{47.000000\du}{64.250000\du}}
\pgfpathcurveto{\pgfpoint{46.875000\du}{64.250000\du}}{\pgfpoint{46.750000\du}{64.125000\du}}{\pgfpoint{46.750000\du}{64.000000\du}}
\pgfpathcurveto{\pgfpoint{46.750000\du}{63.875000\du}}{\pgfpoint{46.875000\du}{63.750000\du}}{\pgfpoint{47.000000\du}{63.750000\du}}
\pgfpathcurveto{\pgfpoint{47.125000\du}{63.750000\du}}{\pgfpoint{47.250000\du}{63.875000\du}}{\pgfpoint{47.250000\du}{64.000000\du}}
\pgfusepath{fill}
\definecolor{dialinecolor}{rgb}{0.000000, 0.000000, 0.000000}
\pgfsetstrokecolor{dialinecolor}
\pgfpathmoveto{\pgfpoint{47.250000\du}{64.000000\du}}
\pgfpathcurveto{\pgfpoint{47.250000\du}{64.125000\du}}{\pgfpoint{47.125000\du}{64.250000\du}}{\pgfpoint{47.000000\du}{64.250000\du}}
\pgfpathcurveto{\pgfpoint{46.875000\du}{64.250000\du}}{\pgfpoint{46.750000\du}{64.125000\du}}{\pgfpoint{46.750000\du}{64.000000\du}}
\pgfpathcurveto{\pgfpoint{46.750000\du}{63.875000\du}}{\pgfpoint{46.875000\du}{63.750000\du}}{\pgfpoint{47.000000\du}{63.750000\du}}
\pgfpathcurveto{\pgfpoint{47.125000\du}{63.750000\du}}{\pgfpoint{47.250000\du}{63.875000\du}}{\pgfpoint{47.250000\du}{64.000000\du}}
\pgfusepath{stroke}
\pgfsetlinewidth{0.100000\du}
\pgfsetdash{}{0pt}
\pgfsetdash{}{0pt}
\pgfsetbuttcap
{
\definecolor{dialinecolor}{rgb}{0.000000, 0.000000, 0.000000}
\pgfsetfillcolor{dialinecolor}
% was here!!!
\definecolor{dialinecolor}{rgb}{0.000000, 0.000000, 0.000000}
\pgfsetstrokecolor{dialinecolor}
\draw (43.000000\du,64.000000\du)--(45.000000\du,64.000000\du);
}
% setfont left to latex
\definecolor{dialinecolor}{rgb}{0.000000, 0.000000, 0.000000}
\pgfsetstrokecolor{dialinecolor}
\node at (43.000000\du,59.222500\du){Port 2};
% setfont left to latex
\definecolor{dialinecolor}{rgb}{0.000000, 0.000000, 0.000000}
\pgfsetstrokecolor{dialinecolor}
\node at (45.000000\du,60.949900\du){$e_{32}$};
% setfont left to latex
\definecolor{dialinecolor}{rgb}{0.000000, 0.000000, 0.000000}
\pgfsetstrokecolor{dialinecolor}
\node at (45.000000\du,63.495100\du){$e_{23}$};
% setfont left to latex
\definecolor{dialinecolor}{rgb}{0.000000, 0.000000, 0.000000}
\pgfsetstrokecolor{dialinecolor}
\node at (43.809200\du,62.222500\du){$e_{22}$};
% setfont left to latex
\definecolor{dialinecolor}{rgb}{0.000000, 0.000000, 0.000000}
\pgfsetstrokecolor{dialinecolor}
\node at (46.190800\du,62.222500\du){$e_{33}$};
% setfont left to latex
\definecolor{dialinecolor}{rgb}{0.000000, 0.000000, 0.000000}
\pgfsetstrokecolor{dialinecolor}
\node at (37.000000\du,63.222500\du){$S_{12}$};
% setfont left to latex
\definecolor{dialinecolor}{rgb}{0.000000, 0.000000, 0.000000}
\pgfsetstrokecolor{dialinecolor}
\node at (38.000000\du,62.222500\du){$S_{22}$};
\pgfsetlinewidth{0.100000\du}
\pgfsetdash{}{0pt}
\pgfsetdash{}{0pt}
\pgfsetbuttcap
{
\definecolor{dialinecolor}{rgb}{0.000000, 0.000000, 0.000000}
\pgfsetfillcolor{dialinecolor}
% was here!!!
\pgfsetarrowsend{latex}
\definecolor{dialinecolor}{rgb}{0.000000, 0.000000, 0.000000}
\pgfsetstrokecolor{dialinecolor}
\draw (51.250000\du,64.000000\du)--(48.750000\du,64.000000\du);
}
\definecolor{dialinecolor}{rgb}{0.000000, 0.000000, 0.000000}
\pgfsetstrokecolor{dialinecolor}
\draw (50.700000\du,64.000000\du)--(48.750000\du,64.000000\du);
\pgfsetlinewidth{0.100000\du}
\pgfsetdash{}{0pt}
\pgfsetmiterjoin
\pgfsetbuttcap
\definecolor{dialinecolor}{rgb}{1.000000, 1.000000, 1.000000}
\pgfsetfillcolor{dialinecolor}
\pgfpathmoveto{\pgfpoint{51.200000\du}{64.000000\du}}
\pgfpathcurveto{\pgfpoint{51.200000\du}{64.125000\du}}{\pgfpoint{51.075000\du}{64.250000\du}}{\pgfpoint{50.950000\du}{64.250000\du}}
\pgfpathcurveto{\pgfpoint{50.825000\du}{64.250000\du}}{\pgfpoint{50.700000\du}{64.125000\du}}{\pgfpoint{50.700000\du}{64.000000\du}}
\pgfpathcurveto{\pgfpoint{50.700000\du}{63.875000\du}}{\pgfpoint{50.825000\du}{63.750000\du}}{\pgfpoint{50.950000\du}{63.750000\du}}
\pgfpathcurveto{\pgfpoint{51.075000\du}{63.750000\du}}{\pgfpoint{51.200000\du}{63.875000\du}}{\pgfpoint{51.200000\du}{64.000000\du}}
\pgfusepath{fill}
\definecolor{dialinecolor}{rgb}{0.000000, 0.000000, 0.000000}
\pgfsetstrokecolor{dialinecolor}
\pgfpathmoveto{\pgfpoint{51.200000\du}{64.000000\du}}
\pgfpathcurveto{\pgfpoint{51.200000\du}{64.125000\du}}{\pgfpoint{51.075000\du}{64.250000\du}}{\pgfpoint{50.950000\du}{64.250000\du}}
\pgfpathcurveto{\pgfpoint{50.825000\du}{64.250000\du}}{\pgfpoint{50.700000\du}{64.125000\du}}{\pgfpoint{50.700000\du}{64.000000\du}}
\pgfpathcurveto{\pgfpoint{50.700000\du}{63.875000\du}}{\pgfpoint{50.825000\du}{63.750000\du}}{\pgfpoint{50.950000\du}{63.750000\du}}
\pgfpathcurveto{\pgfpoint{51.075000\du}{63.750000\du}}{\pgfpoint{51.200000\du}{63.875000\du}}{\pgfpoint{51.200000\du}{64.000000\du}}
\pgfusepath{stroke}
\pgfsetlinewidth{0.100000\du}
\pgfsetdash{}{0pt}
\pgfsetdash{}{0pt}
\pgfsetbuttcap
{
\definecolor{dialinecolor}{rgb}{0.000000, 0.000000, 0.000000}
\pgfsetfillcolor{dialinecolor}
% was here!!!
\definecolor{dialinecolor}{rgb}{0.000000, 0.000000, 0.000000}
\pgfsetstrokecolor{dialinecolor}
\draw (47.000000\du,64.000000\du)--(49.000000\du,64.000000\du);
}
\pgfsetlinewidth{0.100000\du}
\pgfsetdash{}{0pt}
\pgfsetdash{}{0pt}
\pgfsetbuttcap
{
\definecolor{dialinecolor}{rgb}{0.000000, 0.000000, 0.000000}
\pgfsetfillcolor{dialinecolor}
% was here!!!
\definecolor{dialinecolor}{rgb}{0.000000, 0.000000, 0.000000}
\pgfsetstrokecolor{dialinecolor}
\draw (51.250000\du,60.000000\du)--(49.000000\du,60.000000\du);
}
\definecolor{dialinecolor}{rgb}{0.000000, 0.000000, 0.000000}
\pgfsetstrokecolor{dialinecolor}
\draw (50.700000\du,60.000000\du)--(49.000000\du,60.000000\du);
\pgfsetlinewidth{0.100000\du}
\pgfsetdash{}{0pt}
\pgfsetmiterjoin
\pgfsetbuttcap
\definecolor{dialinecolor}{rgb}{1.000000, 1.000000, 1.000000}
\pgfsetfillcolor{dialinecolor}
\pgfpathmoveto{\pgfpoint{51.200000\du}{60.000000\du}}
\pgfpathcurveto{\pgfpoint{51.200000\du}{60.125000\du}}{\pgfpoint{51.075000\du}{60.250000\du}}{\pgfpoint{50.950000\du}{60.250000\du}}
\pgfpathcurveto{\pgfpoint{50.825000\du}{60.250000\du}}{\pgfpoint{50.700000\du}{60.125000\du}}{\pgfpoint{50.700000\du}{60.000000\du}}
\pgfpathcurveto{\pgfpoint{50.700000\du}{59.875000\du}}{\pgfpoint{50.825000\du}{59.750000\du}}{\pgfpoint{50.950000\du}{59.750000\du}}
\pgfpathcurveto{\pgfpoint{51.075000\du}{59.750000\du}}{\pgfpoint{51.200000\du}{59.875000\du}}{\pgfpoint{51.200000\du}{60.000000\du}}
\pgfusepath{fill}
\definecolor{dialinecolor}{rgb}{0.000000, 0.000000, 0.000000}
\pgfsetstrokecolor{dialinecolor}
\pgfpathmoveto{\pgfpoint{51.200000\du}{60.000000\du}}
\pgfpathcurveto{\pgfpoint{51.200000\du}{60.125000\du}}{\pgfpoint{51.075000\du}{60.250000\du}}{\pgfpoint{50.950000\du}{60.250000\du}}
\pgfpathcurveto{\pgfpoint{50.825000\du}{60.250000\du}}{\pgfpoint{50.700000\du}{60.125000\du}}{\pgfpoint{50.700000\du}{60.000000\du}}
\pgfpathcurveto{\pgfpoint{50.700000\du}{59.875000\du}}{\pgfpoint{50.825000\du}{59.750000\du}}{\pgfpoint{50.950000\du}{59.750000\du}}
\pgfpathcurveto{\pgfpoint{51.075000\du}{59.750000\du}}{\pgfpoint{51.200000\du}{59.875000\du}}{\pgfpoint{51.200000\du}{60.000000\du}}
\pgfusepath{stroke}
\pgfsetlinewidth{0.100000\du}
\pgfsetdash{}{0pt}
\pgfsetdash{}{0pt}
\pgfsetbuttcap
{
\definecolor{dialinecolor}{rgb}{0.000000, 0.000000, 0.000000}
\pgfsetfillcolor{dialinecolor}
% was here!!!
\pgfsetarrowsend{latex}
\definecolor{dialinecolor}{rgb}{0.000000, 0.000000, 0.000000}
\pgfsetstrokecolor{dialinecolor}
\draw (47.000000\du,60.000000\du)--(49.250000\du,60.000000\du);
}
% setfont left to latex
\definecolor{dialinecolor}{rgb}{0.000000, 0.000000, 0.000000}
\pgfsetstrokecolor{dialinecolor}
\node at (52.000000\du,60.213188\du){$b_{3}$};
% setfont left to latex
\definecolor{dialinecolor}{rgb}{0.000000, 0.000000, 0.000000}
\pgfsetstrokecolor{dialinecolor}
\node at (52.000000\du,64.213188\du){$a_{3}$};
\pgfsetlinewidth{0.100000\du}
\pgfsetdash{{\pgflinewidth}{0.200000\du}}{0cm}
\pgfsetdash{{\pgflinewidth}{0.200000\du}}{0cm}
\pgfsetbuttcap
{
\definecolor{dialinecolor}{rgb}{0.000000, 0.000000, 0.000000}
\pgfsetfillcolor{dialinecolor}
% was here!!!
\definecolor{dialinecolor}{rgb}{0.000000, 0.000000, 0.000000}
\pgfsetstrokecolor{dialinecolor}
\draw (27.000000\du,60.000000\du)--(27.000000\du,56.000000\du);
}
\pgfsetlinewidth{0.100000\du}
\pgfsetdash{{\pgflinewidth}{0.200000\du}}{0cm}
\pgfsetdash{{\pgflinewidth}{0.200000\du}}{0cm}
\pgfsetbuttcap
{
\definecolor{dialinecolor}{rgb}{0.000000, 0.000000, 0.000000}
\pgfsetfillcolor{dialinecolor}
% was here!!!
\pgfsetarrowsend{latex}
\definecolor{dialinecolor}{rgb}{0.000000, 0.000000, 0.000000}
\pgfsetstrokecolor{dialinecolor}
\draw (27.000000\du,56.000000\du)--(37.250000\du,56.000000\du);
}
\pgfsetlinewidth{0.100000\du}
\pgfsetdash{{\pgflinewidth}{0.200000\du}}{0cm}
\pgfsetdash{{\pgflinewidth}{0.200000\du}}{0cm}
\pgfsetbuttcap
{
\definecolor{dialinecolor}{rgb}{0.000000, 0.000000, 0.000000}
\pgfsetfillcolor{dialinecolor}
% was here!!!
\definecolor{dialinecolor}{rgb}{0.000000, 0.000000, 0.000000}
\pgfsetstrokecolor{dialinecolor}
\draw (37.000000\du,56.000000\du)--(47.000000\du,56.000000\du);
}
\pgfsetlinewidth{0.100000\du}
\pgfsetdash{{\pgflinewidth}{0.200000\du}}{0cm}
\pgfsetdash{{\pgflinewidth}{0.200000\du}}{0cm}
\pgfsetbuttcap
{
\definecolor{dialinecolor}{rgb}{0.000000, 0.000000, 0.000000}
\pgfsetfillcolor{dialinecolor}
% was here!!!
\definecolor{dialinecolor}{rgb}{0.000000, 0.000000, 0.000000}
\pgfsetstrokecolor{dialinecolor}
\draw (47.000000\du,56.000000\du)--(47.000000\du,60.000000\du);
}
\pgfsetlinewidth{0.100000\du}
\pgfsetdash{{\pgflinewidth}{0.200000\du}}{0cm}
\pgfsetdash{{\pgflinewidth}{0.200000\du}}{0cm}
\pgfsetbuttcap
{
\definecolor{dialinecolor}{rgb}{0.000000, 0.000000, 0.000000}
\pgfsetfillcolor{dialinecolor}
% was here!!!
\definecolor{dialinecolor}{rgb}{0.000000, 0.000000, 0.000000}
\pgfsetstrokecolor{dialinecolor}
\draw (47.000000\du,64.000000\du)--(47.000000\du,68.000000\du);
}
\pgfsetlinewidth{0.100000\du}
\pgfsetdash{{\pgflinewidth}{0.200000\du}}{0cm}
\pgfsetdash{{\pgflinewidth}{0.200000\du}}{0cm}
\pgfsetbuttcap
{
\definecolor{dialinecolor}{rgb}{0.000000, 0.000000, 0.000000}
\pgfsetfillcolor{dialinecolor}
% was here!!!
\pgfsetarrowsend{latex}
\definecolor{dialinecolor}{rgb}{0.000000, 0.000000, 0.000000}
\pgfsetstrokecolor{dialinecolor}
\draw (47.000000\du,68.000000\du)--(36.750000\du,68.000000\du);
}
\pgfsetlinewidth{0.100000\du}
\pgfsetdash{{\pgflinewidth}{0.200000\du}}{0cm}
\pgfsetdash{{\pgflinewidth}{0.200000\du}}{0cm}
\pgfsetbuttcap
{
\definecolor{dialinecolor}{rgb}{0.000000, 0.000000, 0.000000}
\pgfsetfillcolor{dialinecolor}
% was here!!!
\definecolor{dialinecolor}{rgb}{0.000000, 0.000000, 0.000000}
\pgfsetstrokecolor{dialinecolor}
\draw (37.000000\du,68.000000\du)--(27.000000\du,68.000000\du);
}
\pgfsetlinewidth{0.100000\du}
\pgfsetdash{{\pgflinewidth}{0.200000\du}}{0cm}
\pgfsetdash{{\pgflinewidth}{0.200000\du}}{0cm}
\pgfsetbuttcap
{
\definecolor{dialinecolor}{rgb}{0.000000, 0.000000, 0.000000}
\pgfsetfillcolor{dialinecolor}
% was here!!!
\definecolor{dialinecolor}{rgb}{0.000000, 0.000000, 0.000000}
\pgfsetstrokecolor{dialinecolor}
\draw (27.000000\du,68.000000\du)--(27.000000\du,64.000000\du);
}
% setfont left to latex
\definecolor{dialinecolor}{rgb}{0.000000, 0.000000, 0.000000}
\pgfsetstrokecolor{dialinecolor}
\node at (37.000000\du,55.222500\du){$e_{30}$};
% setfont left to latex
\definecolor{dialinecolor}{rgb}{0.000000, 0.000000, 0.000000}
\pgfsetstrokecolor{dialinecolor}
\node at (37.000000\du,69.222500\du){$e_{03}$};
\end{tikzpicture}

	\caption{Two port model}
	\label{fig:twoportmodel}
\end{figure}

Where the SOL calibration is using 3 error coefficients, the full two port calibration needs 12 for SOLTI calibration (Short - Open - Load - Through - Isolation). In order to obtain the 12 error terms, we need Short Open Load measurements for every port as performed in section \ref{sec:solcal}. Apart from the SOL measurements we also need a Through measurement (two cables connected) and an isolation measurement. An additional standard is also introduced, a non-perfect through standard (see section \ref{sec:calstds}). The 12 error coefficients needed for SOLTI calibration are obtained in section \ref{sec:obtainingsolti}.

\begin{equation}
D = \left[1+\frac{S_{11M}-e_{00}}{e_{10}e_{01}}e_{11}\right]
\left[1+\frac{S_{22M}-e_{33}}{e_{23}e_{32}}e_{22}\right]-
\frac{S_{21M}-e_{30}}{e_{10}e_{32}}
\frac{S_{12M}-e_{03}}{e_{23}e_{01}}
e'_{22}e'_{11}
\end{equation}
\begin{equation}
S_{11}= \frac{\frac{S_{11M}-e_{00}}{e_{10}e_{01}}\left[ 1+\frac{S_{22M}-e_{33}}{e_{23}e_{32}}e_{22}\right]-e'_{22}\frac{S_{21M}-e_{30}}{e_{10}e_{32}}\frac{S_{12M}-e_{03}}{e_{23}e_{01}}}{D}
\end{equation}
\begin{equation}
S_{21} = \frac{\frac{S_{21M}-e_{30}}{e_{10}e_{32}}\left[1+\frac{S_{22M}-e_{33}}{e_{23}e_{32}}(e_{22}-e'_{22})\right]}{D}
\end{equation}
\begin{equation}
S_{12} = \frac{\frac{S_{12M}-e_{03}}{e_{23}e_{01}}\left[1+\frac{S_{11M}-e_{00}}{e_{10}e_{01}}(e_{11}-e'_{11})\right]}{D}
\end{equation}
\begin{equation}
S_{22}= \frac{\frac{S_{22M}-e_{33}}{e_{23}e_{32}}\left[ 1+\frac{S_{11M}-e_{00}}{e_{10}e_{01}}e_{11}\right]-e'_{11}\frac{S_{12M}-e_{03}}{e_{23}e_{01}}\frac{S_{21M}-e_{30}}{e_{10}e_{32}}}{D}
\end{equation}
\newpage
\begin{itemize}
	\item $e_{00}$: Port 1 directivity
	\item $e_{11}$: Port 1 match
	\item $e_{10}e_{01}$: Port 1 reflection tracking
	\item $e_{10}e_{32}$: Forward transmission tracking
	\item $e'_{22}$: Port 2 match seen from port 1
	\item $e_{30}$: Forward transmission leakage
	\item $e_{33}$: Port 2 directivity
	\item $e_{22}$: Port 2 match
	\item $e_{23}e_{32}$: Port 2 reflection tracking
	\item $e_{23}e_{01}$: Reverse transmission tracking
	\item $e'_{11}$: Port 1 match seen from port 2
	\item $e_{03}$: Reverse transmission leakage
\end{itemize}



\subsubsection{obtaining the error coefficients for SOLTI Calibration}
\label{sec:obtainingsolti}
The Through - Isolation calibration depends on the SOL calibration, this calibration has to be done first. 
$e_{00}$, $e_{11}$ and $\Delta_{eP1}$ are obtained from the measurements in \ref{sec:obtainingerrorcoefsSOL}, $e_{33}$, $e_{22}$ and $\Delta_{eP2}$ use the same equations but with Short / Open / Load measurements taken from port 2.
\begin{itemize}
	\item {$e_{00}$ and $e_{33}$ are obtained from equation \ref{eqn:e00}}
	\item {$e_{11}$ and $e_{22}$ are obtained from equation \ref{eqn:e11}}	
	\item {$\Delta_{eP1}$ and $\Delta_{eP2}$ are obtained from equation \ref{eqn:deltae}}
\end{itemize}

\begin{equation}
(e_{10}e_{01}) = \frac{\Delta_{eP1}}{e_{11}e_{00}}
\end{equation}
\begin{equation}
(e_{23}e_{32}) = \frac{\Delta_{eP2}}{e_{22}e_{33}}
\end{equation}

For the next set of equations, we determine 5 sets of 2-port s-parameter matrices.
\begin{itemize}
	\item {$S_i$: The isolation measurement executed with the two measurement cables connected to $50\Omega$ loads. From this matrix only $S_{21i}$ and $S_{12i}$ are used.}
	\item {$S_t$: The through measurement, this s-parameter contains data of the measured through standard with the uncalibrated network analyzer.}
	\item {$S_s$: The through standard (see paragraph \ref{sec:calstds})}
	\item {$S_M$: The measured S-parameter matrix of the 2-port DUT}
	\item {$S$: This is the resulting calibrated S-parameter matrix containing $S_{11}$, $S_{21}$, $S_{12}$ and $S_{22}$}
	
\end{itemize}
The two port isolation measurement file is created with the measurement cables connected to a 50$\Omega$ load. From this file the following error terms are obtained:
\begin{itemize}
	\item $e_{30} = S_{21i}$, this is the forward isolation
	\item $e_{03} = S_{12i}$, this is the reverse isolation
\end{itemize}



$e'_{22}$ is in fact the same error term as $e_{22}$, but it is seen from port 1, for that reason we use a different calculation. Instead we use the SOL calibration method to de-embed the reflection measured at the through measurement $S_{11t}$, normalized to the standard $S_{21s}$. The same is true for $e'_{11}$. 

\begin{equation}
e'_{22} = \frac{\frac{S_{11t}}{S_{21s}}-e_{00}}{\frac{S_{11t}}{S_{21s}}e_{11}-\Delta_{eP1}}
\end{equation}
\begin{equation}
e'_{11} = \frac{\frac{S_{22t}}{S_{12s}}-e_{33}}{\frac{S_{22t}}{S_{12s}}e_{22}-\Delta_{eP2}}
\end{equation}

\begin{equation}
e_{10}e_{32} = \left(\frac{S_{21t}}{S_{21s}} - e_{30}\right)\left(1-e_{11}e'_{22}\right);
\end{equation}
\begin{equation}
e_{23}e_{01} = \left(\frac{S_{12t}}{S_{12s}} - e_{03}\right)\left(1-e_{22}e'_{11}\right);
\end{equation}


\subsection{Calibration Standards}
\label{sec:calstds}

The calibrations used in \ref{sec:solcal} and \ref{sec:soltical} contain measurements of known standards, but apart from the measured trace we also need to know what the actual standard looks like. In order to use the standard in the calibration methods, we need to know the $\Gamma$ (S-Parameter) of the standard.
\subsubsection{Short}
The short standard like other standards is never $0\Omega$, instead it usually has a series inductance that can be frequency dependent. The connector of the standard can also have electrical loss and an electrical length. The parameters of the Short standard in DeEmbed are:
\begin{itemize}
	\item $L_0$: Constant series inductance.
	\item $L_1 \cdot f$: Series inductance with linear frequency dependency
	\item $L_2 \cdot f^2$: Series inductance with quadratic frequency dependency
	\item $L_3 \cdot f^3$: Series inductance with cubic frequency dependency
	\item Loss(dB): Connector or trace loss of the standard
	\item Loss(dB/Hz): Frequency dependent connector or trace loss of the standard
	\item Length (m): Electrical (not physical) length of the connector / trace towards the standard.
\end{itemize}
$\Gamma_{sp}$ is calculated in 3 steps, first the complex impedance of the inductive part is calculated:
\begin{equation}
Z_{sp} = i 2 \pi f (L_0 + L_1 f + L_2 f^2 + L_3 f^3)
\end{equation}
Convert $Z_{sp}$ into $\Gamma_{sp}$ with equation \ref{eqn:stoz}.
\begin{equation}
\label{eqn:stoz}
\Gamma=\frac{\frac{Z}{50\Omega}-1}{\frac{Z}{50\Omega}+1}
\end{equation}
At last we apply electrical length and loss:
\begin{equation}
\Gamma_{sp} = \Gamma_{sp} e^{2 \pi length\cdot f\cdot c^{-1}} 10^{\frac{-Loss_{dB}}{20}} 10^{\frac{-Loss_{dBHz} f }{20}}
\end{equation}
where $c$ is the speed of light (299792458)


\subsubsection{Open}
The open standard like other standards is never $\infty\Omega$, instead it usually has a capacitance that can be frequency dependent. The connector of the standard can also have electrical loss and an electrical length. The parameters of the Open standard in DeEmbed are:
\begin{itemize}
	\item $C_0$: Constant series capacitance.
	\item $C_1 \cdot f$: Series capacitance with linear frequency dependency
	\item $C_2 \cdot f^2$: Series capacitance with quadratic frequency dependency
	\item $C_3 \cdot f^3$: Series capacitance with cubic frequency dependency
	\item Loss(dB): Connector or trace loss of the standard
	\item Loss(dB/Hz): Frequency dependent connector or trace loss of the standard
	\item Length (m): Electrical (not physical) length of the connector / trace towards the standard.
\end{itemize}
$\Gamma_{op}$ is calculated in 3 steps, first the complex impedance of the capacitive part is calculated:
\begin{equation}
Z_{op} = \frac{-i}{2 \pi f (C_0 + C_1 f + C_2 f^2 + C_3 f^3)}
\end{equation}
Convert $Z_{op}$ into $\Gamma_{op}$ with equation \ref{eqn:stoz}. 
At last we apply electrical length and loss:
\begin{equation}
\Gamma_{op} = \Gamma_{op} e^{2 \pi length\cdot f\cdot c^{-1}} 10^{\frac{-Loss_{dB}}{20}} 10^{\frac{-Loss_{dBHz} f }{20}}
\end{equation}
where $c$ is the speed of light (299792458)
\subsubsection{Load}
The load standard in DeEmbed can be defined as a perfect resistance ($R_l$) with a series inductance ($L_l$).
\begin{equation}
Z_{lp} = R_l + 2 \pi f \cdot L_l
\end{equation}
Convert $Z_{lp}$ into $\Gamma_{lp}$ with equation \ref{eqn:stoz}.
\subsubsection{Through}
The through standard is only a connector between the two cables, we define the standard with only length and loss:
\begin{itemize}
	\item Loss(dB): Connector or trace loss of the standard
	\item Loss(dB/Hz): Frequency dependent connector or trace loss of the standard
	\item Length (m): Electrical (not physical) length of the connector / trace towards the standard.
\end{itemize}
\begin{equation}
\Gamma_{th} = 1 \cdot e^{2 \pi length\cdot f\cdot c^{-1}} 10^{\frac{-Loss_{dB}}{20}} 10^{\frac{-Loss_{dBHz} f }{20}}
\end{equation}

\newpage

